% Author: Grayson Orr
% Course: ID607001: Introductory Application Development Concepts

\documentclass{article}
\author{}

\usepackage{graphicx}
\usepackage{wrapfig}
\usepackage{enumerate}
\usepackage{hyperref}
\usepackage[margin = 2.25cm]{geometry}
\usepackage[table]{xcolor}
\usepackage{soul}
\usepackage{fancyhdr}
\hypersetup{
  colorlinks = true,
  urlcolor = blue
}
\setlength\parindent{0pt}
\pagestyle{fancy}
\fancyhf{}
\rhead{College of Engineering, Construction and Living Sciences\\Bachelor of Information Technology}
\lfoot{Project 1: Node.js REST API Marks Breakdown\\Version 1, Semester One, 2022}
\rfoot{\thepage} 
 
\begin{document}

\begin{figure}
  \centering
  \includegraphics[width=50mm]{../img/logo.png}
\end{figure}

\title{College of Engineering, Construction and Living Sciences\\Bachelor of Information Technology\\ID607001: Introductory Application Development Concepts\\Level 6, Credits 15\\\textbf{Project 1: Node.js REST API Marks Breakdown}}
\date{}
\maketitle

\subsection*{Functionality (Features) - Learning Outcome 1 (40\%)}
\begin{itemize}
  \item \textbf{REST API} is developed using \textbf{Node.js}. \hl{0.5\%}
  \item \textbf{REST API} can run locally without modification. \hl{0.5\%}
  \item \textbf{Five} \textbf{collections} containing at least \textbf{three fields} of data which you can interact with. \textbf{Note:} It includes a \textbf{user collection}. \hl{4\%}
  \item A range of different data types, i.e., all \textbf{fields} of data can not be of a single type. \hl{2\%}
  \item \textbf{Two relationships} between \textbf{collections}. \hl{2\%}
  \item Each \textbf{collection} has a separate \textbf{controller} \& \textbf{route} file. \hl{1\%}
  \item A \textbf{controller} contains \textbf{CRUD} (Create, Read, Update \& Delete) operations. \hl{3\%}
  \item Each \textbf{field} of data has custom validation when creating \& updating a \textbf{document}.  \hl{3\%}
  \item Each \textbf{collection} is seeded with a \textbf{JSON} file. \textbf{Note:} This is \textbf{only} for testing purposes. \hl{3\%}
  \item \textbf{Independent Research:} \textbf{REST API} version is set to \textbf{v1}. For example, an endpoint should look like \textbf{/api/v1/items} \hl{0.5\%}
  \item Return success, i.e., true or false \& data when performing \textbf{CRUD} operations. \hl{2\%}
  \item \textbf{Independent Research:} Return an appropriate message if a query does not return any \textbf{REST API data}. \hl{2\%}
  \item \textbf{Independent Research:} Return an appropriate message if an endpoint does not exist. \hl{2\%}
  \item \textbf{Independent Research:} Filter \& sort \textbf{REST API data} using query parameters. A consumer should be able to filter all \textbf{fields} of data \& sort \textbf{fields} of data in ascending/descending order. \hl{5\%}
  \item \textbf{Independent Research:} Paginate the \textbf{REST API data} so that any number of records can be displayed per page. \hl{3\%}
  \item \textbf{GET}, \textbf{POST}, \textbf{PUT} \& \textbf{DELETE} routes are protected using \textbf{JSON Web Tokens (JWT)}. \hl{2\%}
  \item \textbf{REST API} rate limit is set to 25 requests per minute. \hl{0.5\%}
  \item Secure \textbf{HTTP} headers using \textbf{Helmet}. \hl{1\%}
  \item \textbf{REST API} is deployed to \textbf{Heroku}. The \textbf{REST API} \textbf{should} be usable i.e., a consumer should be able to perform operations on your \textbf{REST API}. \hl{2\%}
  \item \textbf{REST API data} is stored in a \textbf{MongoDB Atlas} database. \hl{1\%}
\end{itemize}

\subsection*{Code Elegance - Learning Outcome 1 (45\%)}

A deduction of \textbf{0.5\%} is given for every \textbf{second} occurrence to which the \textbf{Code Elegance} criterion is \textbf{not} met.

\begin{itemize}
  \item Use of intermediate variables. No method calls as arguments. \hl{4\%}
  \item Idiomatic use of control flow, data structures \& in-built functions. \hl{4\%}
  \item Sufficient modularity, i.e., reusable base URL. \hl{4\%}
  \item Functions \& variables are named appropriately. \hl{4\%}
  \item Efficient algorithmic approach, i.e., using the appropriate function(s) when querying your \textbf{collections}. \hl{4\%}
  \item \textbf{REST API} resource groups named with a plural noun instead of a verb, i.e., \textbf{/api/v1/items} not \textbf{/api/v1/item}. \hl{4\%}
  \item File header comments using \textbf{JSDoc}. You \textbf{need} to explain the purpose of each \textbf{controller} \& \textbf{route} file. \hl{5\%}
  \item In-line comments using \textbf{JSDoc}. You \textbf{need} to explain complex logic that is not obvious. \hl{4\%}
  \item \textbf{Independent Research:} Code files are formatted using \textbf{Prettier} \& a \textbf{.prettierrc} file. You \textbf{need} to declare a \textbf{npm} script in your application's \textbf{package.json} file which automates this process \hl{1\%}. Rules \textbf{should} include:
        \begin{itemize}
          \item Single quote is set to \textbf{true}. \hl{1\%}
          \item Semi-colon is set to \textbf{false}. \hl{1\%}
          \item Tab-width is set to \textbf{2}. \hl{1\%}
        \end{itemize}
  \item \textbf{Independent Research:} \textbf{Prettier} is installed as a development dependency. \hl{1\%}
  \item No dead or unused code. \hl{3\%}
  \item Database configured for production environment. \hl{3\%}
  \item Application's environment variables are stored in a \textbf{.env} file. \hl{1\%}
\end{itemize}

\subsection*{Documentation \& Git Usage - Learning Outcome 1 (15\%)}
\begin{itemize}
  \item Project board to help you organise \& prioritise your work. \hl{2\%}
  \item \textbf{REST API} is documented using \textbf{Postman}. You \textbf{should} provide an example for each route. Each example \textbf{should} contain a description, request \& response. \hl{3\%}
  \item Provide the following in your repository \textbf{README.md} file:
        \begin{itemize}
          \item URL to the documented \textbf{REST API} on \textbf{Postman}. \hl{0.5\%}
          \item URL to the \textbf{REST API} on \textbf{Heroku}. \hl{0.5\%}
          \item How do you setup the environment for development, i.e., after the repository is cloned, what do you need to run the \textbf{REST API} locally? \hl{2\%}
          \item How do you deploy the \textbf{REST API} to \textbf{Heroku}? \hl{2\%}
        \end{itemize}
  \item Use of \textbf{Markdown}, i.e., bold text, code blocks, etc. \hl{1\%}
  \item Correct spelling \& grammar. \hl{1\%}
  \item Your \textbf{Git commit messages} should:
        \begin{itemize}
          \item Reflect the context of each functional requirement change. \hl{2\%}
          \item Be formatted using the naming conventions outlined in the following: \hl{1\%}
                \begin{itemize}
                  \item \textbf{Resource:} \small\href{https://dev.to/i5han3/git-commit-message-convention-that-you-can-follow-1709}{https://dev.to/i5han3/git-commit-message-convention-that-you-can-follow-1709}
                \end{itemize}
        \end{itemize}
\end{itemize}

\end{document}