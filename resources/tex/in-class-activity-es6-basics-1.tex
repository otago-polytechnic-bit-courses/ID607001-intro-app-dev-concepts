% Author: Grayson Orr
% Course: ID607001: Introductory Application Development Concept

\documentclass{article}
\author{}

\usepackage{graphicx}
\usepackage{wrapfig}
\usepackage{enumerate}
\usepackage{hyperref}
\usepackage[margin = 2.25cm]{geometry}
\usepackage[table]{xcolor} 
\usepackage{fancyhdr}
\hypersetup{
  colorlinks = true, 
  urlcolor = blue
}
\setlength\parindent{0pt}
\pagestyle{fancy}
\fancyhf{}
\rhead{College of Engineering, Construction \& Living Sciences\\Bachelor of Information Technology}
\lfoot{In-Class Activity: ES6 Basics 1\\Version 1, Semester One, 2022}
\rfoot{\thepage}

\begin{document}

\begin{figure}
	\centering
	\includegraphics[width=50mm]{../img/logo.png}
\end{figure}

\title{College of Engineering, Construction \& Living Sciences\\Bachelor of Information Technology\\ID607001: Introductory Application Development Concepts\\Level 6, Credits 15\\\textbf{In-Class Activity: ES6 Basics 1}}
\date{}
\maketitle
 
\section*{Instructions}
The purpose of this in-class activity is to familiarise yourself with the \textbf{ES6} syntax as well as develop your problem-solving skills. The following 15 problems are commonly asked in coding interviews. You may come across one or two of these when you apply for software development/engineering positions in the future. \textbf{Note:} do not use functional programming constructs such as \textbf{map}, \textbf{filter} \& \textbf{reduce} to solve some of these problems.

\section*{Code Review}
You must submit all program files via \textbf{GitHub Classroom}. Here is the URL to the repository you will use for your code review – \href{https://classroom.github.com/a/\_6KSahyX}{https://classroom.github.com/a/\_6KSahyX}. If you wish to have your code reviewed, message me on \textbf{Microsoft Teams}.

\section*{Getting Started}
Open your repository in \textbf{Visual Studio Code}. Create a new file called \textbf{01-in-class-activity.js}. In \textbf{01-in-class-activity.js}, add the following:

\begin{verbatim}
  console.log('Hello, World!')
\end{verbatim}

Open a \textbf{terminal} \& run the following command:

\begin{verbatim}
  node 01-in-class-activity.js
\end{verbatim}

If the output is \textbf{Hello, World!}, then you are ready to start coding.

\subsection*{Problem 1:} 
Declare two \textbf{immutable} variables called \textbf{name} \& \textbf{age} with the values Jane \& 45. Use the two variables \& \textbf{string interpolation} to display the expected output. 

\begin{verbatim}
  // Write your solution here

  // Expected output:
  // Hello my name is Jane & I am 45 years old.
\end{verbatim}

\subsection*{Problem 2:} 
Calculate the \textbf{sum} of the given \textbf{integers} \& use \textbf{string interpolation} to display the expected output.

\begin{verbatim}
  const x = 1957452
  const y = 2975635

  // Write your solution here

  // Expected output:
  // The sum of 1957452 & 2975635 is 4933087
\end{verbatim}

\subsection*{Problem 3:} 
Calculate the \textbf{average} of the given \textbf{array} of \textbf{doubles} called \textbf{nums} \& use \textbf{string interpolation} to display the expected output.

\begin{verbatim}
  const nums = [45.3, 67.5, -45.6, 20.34, -33.0, 45.6]

  // Write your solution here

  // Expected output:
  // Average: 16.69 
\end{verbatim}

\subsection*{Problem 4:}
Write an \textbf{arrow function} called \textbf{fizzBuzz} which accepts an \textbf{integer} \textbf{num}. If \textbf{num} is a multiple of three, return \textbf{Fizz}, if \textbf{num} is a multiple of five, return \textbf{Buzz} \& if \textbf{num} is a multiple of three \& five, return \textbf{FizzBuzz}. Call the \textbf{fizzBuzz} function in the for loop to display the expected output.

\begin{verbatim}
  // Write your fizzBuzz function here

  for (let i = 1; i <= 15; i += 2) {
      // Write your solution here
  }

  // Expected output:
  // 1
  // Fizz 
  // Buzz
  // 7
  // Fizz
  // 11
  // 13
  // FizzBuzz
\end{verbatim}

\subsection*{Problem 5:}
You have been given an \textbf{array} of \textbf{integers} called \textbf{nums}. Display \textbf{only} the odd numbers in \textbf{nums}. Sort from lowest to highest.

\begin{verbatim}
  const nums = [21, 19, 68, 55, 42, 12]
        
  // Write your solution here

  // Expected output:
  // 19
  // 21
  // 55
\end{verbatim}

\subsection*{Problem 6:}
Write an \textbf{arrow function} called \textbf{isAnagram} which accepts two parameters called \textbf{someStrOne} \& \textbf{someStrTwo}. In the function block, write some code that checks whether or not \textbf{someStrOne} \& \textbf{someStrTwo} are an anagram. \textbf{Note:} An anagram is a word or phrase that made by arranging the letters of another word or phrase in a different order. If you are still unsure what an anagram is, here is an example:

\begin{verbatim}
  Input: isAnagram('elvis', 'lives')
  Output: true

  Input: isAnagram('cat', 'sat')
  Output : false
\end{verbatim}

Call the \textbf{isAnagram} function to display the expected output.

\begin{verbatim}
  // Write your solution here

  // Expected output:
  // true
  // false
\end{verbatim}

\subsection*{Problem 7:}
Write an \textbf{arrow function} called \textbf{convert} which accepts two parameters called \textbf{hours} \& \textbf{minutes}. In the function block, write some code that converts both \textbf{hours} \& \textbf{minutes} to seconds, then adds them together.

\begin{verbatim}
  // Write your solution here

  console.log(convert(1, 3))

  // Expected output:
  // 3780
\end{verbatim}

\subsection*{Problem 8:}
Write an \textbf{arrow function} called \textbf{palindrome} which accepts a single parameter called \textbf{someStr}. In the function block, determine whether or not \textbf{someStr} is a palindrome. The function should return a \textbf{boolean}.

\begin{verbatim}
  // Write your solution here

  console.log(palindrome('A man, a plan, a canal - Panama'))
  console.log(palindrome('Hello, World!'))

  // Expected output:
  // true
  // false
\end{verbatim}
 
\subsection*{Problem 9:}
Write an \textbf{arrow function} called \textbf{isLessThanFiveLetters} which accepts an \textbf{array} of \textbf{strings}. In the function block, return all words that are less than \textbf{five} letters. Sort from A to Z.

\begin{verbatim} 
  // Write your solution here

  const transport = ['car', 'bike', 'scooter', 'skateboard', 'truck', 'walk']

  // Expected output:
  // bike
  // car
  // walk
\end{verbatim}

\subsection*{Problem 10:} 
Write an \textbf{arrow function} called \textbf{findBreed} which accepts an unsorted \textbf{array} of \textbf{strings} called \textbf{breeds}. Your code needs to search \textbf{breeds} for 'Afghan Hound' \& return its location in the \textbf{array}, i.e., index. If 'Afghan Hound' is not in \textbf{breeds}, return -1.

\begin{verbatim}
  // Write your solution here

  const breeds = ['Afghan Hound', /** Add your other breeds here */]
  console.log(findBreed(breeds)) 

  // Expected output:
  // 1
\end{verbatim}

\subsection*{Problem 11:} 
Write an \textbf{arrow function} called \textbf{removeVowels} which accepts a \textbf{string} called \textbf{word} \& returns a new \textbf{string} with all vowels removed. Also, how would you handle the edge case where \textbf{word} does not contain vowels.

\begin{verbatim}
  // Write your solution here

  const word = // Add your word here
  console.log(removeVowels(word))
\end{verbatim}

\subsection*{Problem 12:} 
Write an \textbf{arrow function} function called \textbf{missingNum} which accepts an unsorted \textbf{array} of \textbf{integers} called \textbf{nums} \& return the missing number.

\begin{verbatim}
  // Write your solution here

  const nums = [10, 3, 4, 8, 1, 7, 6, 9, 5]
  console.log(missingNum(nums))

  // Expected output:
  // 2
\end{verbatim}

\subsection*{Problem 13:}
Write an \textbf{arrow function} called \textbf{fileExtensions} which accepts an \textbf{array} of \textbf{objects} called \textbf{files} \& returns their extension names.

\begin{verbatim}
  // Write your solution here

  const files = [
      { 'name': 'index', 'extension': 'html' },
      { 'name': 'main', 'extension': 'js' },
      { 'name': 'sample', 'extension': 'txt' },
      { 'name': 'data', 'extension': 'json' }
  ]
  console.log(fileExtensions(files))

  // Expected output:
  // html
  // js
  // txt
  // json
\end{verbatim}

\subsection*{Problem 14:}
What is a substring? It is a portion of a \textbf{string}, i.e., 'Hello' is a substring of 'Hello, World!' and 'el' is a substring of 'Hello'. String manipulation is commonly used \& working with substrings is something you will often do. \\

You have been given the following sentence as a \textbf{string}: \\

\textbf{'The anemone, the wild violet, the hepatica, and the funny little curled-up.'} \\

Write code that returns the number of occurrences of the word \textbf{'the'} in the sentence above.

\begin{verbatim}
  const sentence = 'The anemone, the wild violet, the hepatica, and the funny little curled-up.'

  // Write your solution here

  // Expected output:
  // 4
\end{verbatim}

\subsection*{Problem 15:}
In this problem you are going to use the \textbf{abs()} function. Write an \textbf{arrow function} called \textbf{calcDist} which calculates the distance between two \textbf{integers}. It does not matter which order the parameters are given; it should still return the same result. 

\begin{verbatim}
  // Write your solution here

  console.log(calcDist(-1, 4))
  console.log(calcDist(4, -1))

  // Expected output:
  // 5
  // 5
\end{verbatim}

\end{document}
