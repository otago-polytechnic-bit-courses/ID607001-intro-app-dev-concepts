% Author: Grayson Orr
% Course: IN607: Introductory Application Development Concepts

\documentclass{article}
\author{}

\usepackage{graphicx}
\usepackage{wrapfig}
\usepackage{enumerate}
\usepackage{hyperref}
\usepackage[margin = 2.25cm]{geometry}
\usepackage[table]{xcolor}
\usepackage{fancyhdr}
\hypersetup{
  colorlinks = true,
  urlcolor = blue
}
\setlength\parindent{0pt}
\pagestyle{fancy}
\fancyhf{}
\rhead{College of Engineering, Construction and Living Sciences\\Bachelor of Information Technology}
\lfoot{Project 2: React CRUD\\Version 2, Semester Two, 2021}
\rfoot{\thepage}
 
\begin{document}

\begin{figure}
	\centering
	\includegraphics[width=50mm]{../img/logo.png}
\end{figure}

\title{College of Engineering, Construction and Living Sciences\\Bachelor of Information Technology\\IN607: Introductory Application Development Concepts\\Level 6, Credits 15\\\textbf{Project 2: React CRUD}}
\date{}
\maketitle

\section*{Assessment Overview}
In this assessment, you will develop a \textbf{CRUD} application using \textbf{React} \& deploy it to \textbf{Heroku}. This application will consume the \textbf{Laravel API} you created in the \textbf{Project 1: Laravel API} assessment. The main purpose of this assessment is not just to build a full-stack application, rather to demonstrate an ability to decouple the presentation layer (\textbf{frontend}) from the business logic (\textbf{backend}). Also, you will be required to independently research and implement pagination, deployment \& automated code formatting. In addition, marks will be allocated for code elegance, documentation \& \textbf{Git} usage.

\section*{Learning Outcomes}
At the successful completion of this course, learners will be able to:
\begin{enumerate}
	\item Design \& build usable, secure \& attractive applications with dynamic database functionality following an appropriate software development methodology.
\end{enumerate}

\section*{Assessment Table}
\renewcommand{\arraystretch}{1.5}
\begin{tabular}{|l|l|l|l|l|}
	\hline
	\vtop{\hbox{\strut \textbf{Assessment}}\hbox{\strut \textbf{Activity}}} & \textbf{Weighting} & \vtop{\hbox{\strut \textbf{Learning}}\hbox{\strut \textbf{Outcomes}}} & \vtop{\hbox{\strut \textbf{Assessment}}\hbox{\strut \textbf{Grading Scheme}}} & \vtop{\hbox{\strut \textbf{Completion}}\hbox{\strut \textbf{Requirements}}} \\

	\hline

	\small Practical: API Testing Research                                                      & \small 20\%        & \small 1                                                           & \small CRA                                                                    & \small Cumulative                                                           \\ \hline
	\small Project 1: Laravel API                                                        & \small 30\%        & \small 1                                                        & \small CRA                                                                    & \small Cumulative                                                           \\ \hline
	\small Project 2: React CRUD                                                        & \small 50\%        & \small 1                                                        & \small CRA                                                                    & \small Cumulative                                                           \\ \hline
\end{tabular}

\section*{Conditions of Assessment}
You will complete this assessment during your learner managed time, however, there will be availability during the teaching sessions to discuss the requirements \& your progress of this assessment. This assessment will need to be completed by \textbf{Friday, 19 November 2021 at 5:00 PM}.

\section*{Pass Criteria}
This assessment is criterion-referenced (CRA) with a cumulative pass mark of \textbf{50\%} over all assessments in \textbf{IN607: Introductory Application Development Concepts}.

\section*{Authenticity}
All parts of your submitted assessment must be completely your work \& any references must be cited appropriately. Provide your references in a \textbf{README.md} file. Failure to do this will result in a mark of \textbf{zero} for this assessment.

\section*{Policy on Submissions, Extensions, Resubmissions \& Resits}
The school's process concerning submissions, extensions, resubmissions \& resits complies with \textbf{Otago Polytechnic} policies. Learners can view policies on the \textbf{Otago Polytechnic} website located at \href{https://www.op.ac.nz/about-us/governance-and-management/policies}{https://www.op.ac.nz/about-us/governance-and-management/policies}.

\section*{Submissions}
You must submit all program files via \textbf{GitHub Classroom}. Here is the URL to the repository you will use for your submission – \href{https://classroom.github.com/a/PZJYGNeP}{https://classroom.github.com/a/PZJYGNeP}. The latest program files in the \textbf{main} branch will be used to run your application. Late submissions will incur a \textbf{10\% penalty per day}, rolling over at \textbf{5:00 PM}.

\section*{Extensions}
Familiarise yourself with the assessment due date. If you need an extension, contact the course lecturer before the due date. If you require more than a week's extension, a medical certificate or support letter from your manager may be needed.

\section*{Resubmissions}
Learners may be requested to resubmit an assessment following a rework of part/s of the original assessment. Resubmissions are to be completed within a negotiable short time frame \& usually must be completed within the timing of the course to which the assessment relates. Resubmissions will be available to learners who have made a genuine attempt at the first assessment opportunity \& achieved a \textbf{D grade (40-49\%)}. The maximum grade awarded for resubmission will be \textbf{C-}.

\section*{Resits}
Resits \& reassessments are not applicable in \textbf{IN607: Introductory Application Development Concepts}. 

\newpage

\section*{Instructions}
You will need to submit an application \& documentation that meet the following requirements:

\subsection*{Functionality - Learning Outcomes 1 (40\%)}

\begin{itemize}
		\item Request \textbf{API} data from at least three \textbf{API} resource groups using \textbf{Axios}.
		\item Create new \textbf{API} data via a form. You \textbf{must} display the form in a modal. 
		\item Incorrectly formatted form field values handled gracefully using validation error messages, i.e., \textbf{first name} form field is required.
		\item View \textbf{API} data in a table using an id \& a variety of query parameters. 
		\item Update \textbf{API} data via a form. Similar to creating \textbf{API} data, you \textbf{must} display the form in a modal. 
		\item Delete \textbf{API} data. Prompt the user for deletion. You \textbf{can not} use the in-built \textbf{confirm() JavaScript} function. 
    \item Data should automatically re-render, i.e., the user should not have to refresh the browser to see the created, updated and deleted data.
    \item \textbf{Independent Research:} Paginate \textbf{API} data across several pages with \textbf{next} \& \textbf{previous} buttons or links.
		\item Visually attractive user-interface with a coherent graphical theme \& style using \textbf{Reactstrap}.
	  \item \textbf{Independent Research:} Application deployed to \textbf{Heroku}. 
	  \item Five component tests using \textbf{React Testing Library}.
\end{itemize}

\subsection*{Code Elegance - Learning Outcomes 1 (45\%)}
\begin{itemize}
	\item Idiomatic use of control flow, data structures \& in-built functions.
	\item Sufficient code modularity, i.e., UI split into independent reusable pieces.
	\item Components written as functional, not class.
	\item Adheres to a client-server architecture, i.e., the presentation layer (frontend application) is separate from the business logic (backend application).
	\item Header \& in-line comments explaining complex logic.
	\item \textbf{Independent Research:} Code files are formatted using \textbf{Prettier}. You \textbf{must} declare a \textbf{npm} script in your application's \textbf{package.json} file which automates this process.
	\item No dead or unused code.
\end{itemize}

\subsection*{Documentation \& Git Usage - Learning Outcomes 1 (15\%)}
\begin{itemize}
	\item Provide the following in your repository \textbf{README.md} file:
	      \begin{itemize}
		      \item URL to your application on \textbf{Heroku}.
		      \item How do you setup the environment for development, i.e., after the repository is cloned, what do you need to run the application locally?
					\item How do you deploy the \textbf{React} application to \textbf{Heroku}?
	      \end{itemize}
			\end{itemize}
			\begin{itemize}
	\item At least 10 feature branches excluding the \textbf{main} branch.
	\begin{itemize}
			\item Your branches must be prefix with \textbf{feature}, for example, \textbf{feature-$<$name of functional requirement$>$}.
			\item For each branch, merge your own pull request to the \textbf{main} branch.
	\end{itemize}
	\item Commit messages must reflect the context of each functional requirement change. \textbf{Do not} rewrite your \textbf{Git} history. It is important that the course lecturer can see how you worked on your assessment over time.
\end{itemize}
\end{document}