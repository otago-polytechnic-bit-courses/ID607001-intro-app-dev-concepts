% Author: Grayson Orr
% Course: ID607001: Introductory Application Development Concepts

\documentclass{article}
\author{}

\usepackage{graphicx}
\usepackage{wrapfig}
\usepackage{enumerate}
\usepackage{hyperref}
\usepackage[margin = 2.25cm]{geometry}
\usepackage[table]{xcolor}
\usepackage{fancyhdr}
\hypersetup{
  colorlinks = true,
  urlcolor = blue
}
\setlength\parindent{0pt}
\pagestyle{fancy}
\fancyhf{}
\rhead{College of Engineering, Construction and Living Sciences\\Bachelor of Information Technology}
\lfoot{Project 2: React CRUD\\Version 3, Semester One, 2022}
\rfoot{\thepage}
 
\begin{document}

\begin{figure}
	\centering
	\includegraphics[width=50mm]{../img/logo.png}
\end{figure}

\title{College of Engineering, Construction and Living Sciences\\Bachelor of Information Technology\\ID607001: Introductory Application Development Concepts\\Level 6, Credits 15\\\textbf{Project 2: React CRUD}}
\date{}
\maketitle

\section*{Assessment Overview}
In this assessment, you will develop a \textbf{CRUD} application using \textbf{React} \& deploy it to \textbf{Heroku}. This application will consume an API from either the \textbf{Project 1: Node.js REST API} assessment or the \textbf{Practical: REST API Testing Research}. The main purpose of this assessment is not just to build a full-stack application, rather to demonstrate an ability to decouple the presentation layer (\textbf{frontend}) from the business logic (\textbf{backend}). Also, you will be required to independently research and implement pagination, deployment \& automated code formatting. In addition, marks will be allocated for code elegance, documentation \& \textbf{Git} usage.

\section*{Learning Outcome}
At the successful completion of this course, learners will be able to:
\begin{enumerate}
	\item Design \& build usable, secure \& attractive applications with dynamic database functionality following an appropriate software development methodology.
\end{enumerate}

\section*{Assessment Table}
\renewcommand{\arraystretch}{1.5}
\begin{tabular}{|l|l|l|l|l|}
	\hline
	\vtop{\hbox{\strut \textbf{Assessment}}\hbox{\strut \textbf{Activity}}} & \textbf{Weighting} & \vtop{\hbox{\strut \textbf{Learning}}\hbox{\strut \textbf{Outcome}}} & \vtop{\hbox{\strut \textbf{Assessment}}\hbox{\strut \textbf{Grading Scheme}}} & \vtop{\hbox{\strut \textbf{Completion}}\hbox{\strut \textbf{Requirements}}} \\
	
	\hline
	
	\small Practical: Node.js REST API Testing Research                     & \small 20\%        & \small 1                                                             & \small CRA                                                                    & \small Cumulative                                                           \\ \hline
	\small Project 1: Node.js REST API                                      & \small 30\%        & \small 1                                                             & \small CRA                                                                    & \small Cumulative                                                           \\ \hline
	\small Project 2: React CRUD                                            & \small 50\%        & \small 1                                                             & \small CRA                                                                    & \small Cumulative                                                           \\ \hline
\end{tabular}

\section*{Conditions of Assessment}
You will complete this assessment during your learner-managed time. However, there will be time to discuss the requirements \& your assessment progress during the teaching sessions. This assessment will need to be completed by \textbf{Tuesday, 21 June 2022 at 4.59 PM}.

\section*{Pass Criteria}
This assessment is criterion-referenced (CRA) with a cumulative pass mark of \textbf{50\%} across all assessments in \textbf{ID607001: Introductory Application Development Concepts}.

\section*{Submission}
You must submit all program files via \textbf{GitHub Classroom}. Here is the URL to the repository you will use for your submission – \href{https://classroom.github.com/a/Vq7T0W6E}{https://classroom.github.com/a/Vq7T0W6E}. The latest program files in the \textbf{master} or \textbf{main} branch will be used to mark against the \textbf{Functionality} criterion. Please test your \textbf{master} or \textbf{main} branch application before you submit. Partial marks \textbf{will not} be given for incomplete functionality. Late submissions will incur a \textbf{10\% penalty per day}, rolling over at \textbf{5:00 PM}.

\section*{Authenticity}
All parts of your submitted assessment must be completely your work. If you use code snippets from \textbf{GitHub}, \textbf{StackOverflow} or other online resource, you \textbf{must} reference it appropriately using \textbf{APA 7th edition}. Provide your references in the \textbf{README.md} file in your repository. Failure to do this will result in a mark of \textbf{zero} for this assessment.

\section*{Policy on Submissions, Extensions, Resubmissions \& Resits}
The school's process concerning submissions, extensions, resubmissions \& resits complies with \textbf{Otago Polytechnic} policies. Learners can view policies on the \textbf{Otago Polytechnic} website located at \href{https://www.op.ac.nz/about-us/governance-and-management/policies}{https://www.op.ac.nz/about-us/governance-and-management/policies}.

\section*{Extensions}
Familiarise yourself with the assessment due date. If you need an extension, contact the course lecturer before the due date. If you require more than a \textbf{seven days} extension, a medical certificate or support letter from your manager may be needed.

\section*{Resubmissions}
Learners may be requested to resubmit an assessment following a rework of part/s of the original assessment. Resubmissions are to be completed within a negotiable short time frame \& usually must be completed within the timing of the course to which the assessment relates. Resubmissions will be available to learners who have made a genuine attempt at the first assessment opportunity \& achieved a \textbf{D grade (40-49\%)}. The maximum grade awarded for resubmission will be \textbf{C-}.

\section*{Resits}
Resits \& reassessments are not applicable in \textbf{ID607001: Introductory Application Development Concepts}. 

\newpage

\section*{Exemplar}
You can find an exemplar here - \small\href{https://id607001-graysono-frontend.herokuapp.com}{https://id607001-graysono-frontend.herokuapp.com}.
 
\begin{itemize}
	\item Email: graysono@op.ac.nz
	\item Password: P@ssw0rd123
\end{itemize}

\section*{Instructions}
You will need to submit an application \& documentation that meet the following requirements:

\subsection*{Functionality (Features) - Learning Outcome 1 (40\%)}
\begin{itemize}
	\item Authentication
	      \begin{itemize}
	      	\item \textbf{Independent Research:} Register a new user via a form.
	      	\item Login an existing user via a form.
	      	\item Log out of the application.
	      \end{itemize}
	\item CRUD
	      \begin{itemize}
	      	\item Request \textbf{REST API data} from at least three \textbf{API} resource groups using \textbf{Axios}.
	      	\item Create new \textbf{REST API data} via a form. You can display the form on the page or in a modal. 
	      	\item View \textbf{REST API data} in a table.
	      	\item \textbf{Independent Research:} View \textbf{REST API data} in a table using an id. For example, \textbf{/institutions/1} would return \textbf{REST API data} for that specific \textbf{Institutions} object.
	      	\item \textbf{Independent Research:} Update \textbf{REST API data} via a form. Similar to creating \textbf{REST API data}, you can display the form on the page or in a modal. 
	      	\item \textbf{Independent Research:} Delete \textbf{REST API data}. Prompt the user for deletion. You \textbf{can} use the in-built \textbf{confirm() JavaScript} function. 
	      	\item \textbf{Independent Research:} Incorrectly formatted form field values handled gracefully using validation error messages, i.e., \textbf{first name} form field is required.
	      \end{itemize}
	\item \textbf{Independent Research:} Paginate \textbf{REST API data} across several pages with \textbf{next} \& \textbf{previous} buttons or links. You can choose the number of \textbf{REST API data} per page.
	\item \textbf{Independent Research:} Search \textbf{REST API data} via a search bar.
	\item User-interface is visually attractive with a coherent graphical theme \& style using \textbf{Reactstrap}.
	\item Application deployed to \textbf{Heroku}. 
	\item End-to-end \textbf{Cypress} tests that ensures the register, login and logout functionality is working as expected. You \textbf{need} to declare a \textbf{npm} script in your application's \textbf{package.json} file that automates this process.
\end{itemize}

\subsection*{Code Elegance - Learning Outcome 1 (45\%)}
\begin{itemize}
    \item Use of intermediate variables. No method calls as arguments.
    \item Idiomatic use of control flow, data structures \& in-built functions.
	\item Sufficient modularity, i.e., UI split into independent reusable pieces.
	\item Functions \& variables are named appropriately.
	\item Components written as functional, not class.
	\item Adheres to a client-server architecture, i.e., the frontend is separate from the backend.
	\item File header comment explaining the purpose of each \textbf{component} file.
	\item In-line comments explaining complex logic.
	\item Code files are formatted using \textbf{Prettier}. You \textbf{need} to declare a \textbf{npm} script in your application's \textbf{package.json} file that automates this process. Rules \textbf{must} include:
	      \begin{itemize}
	      	\item Single quote is set to \textbf{true}.
	      	\item Semi-colon is set to \textbf{false}.
	      	\item Tab-width is set to \textbf{2}.
	      \end{itemize}
	\item \textbf{Prettier} \& \textbf{Cypress} are installed as development dependencies.
	\item No dead or unused code.
\end{itemize}

\subsection*{Documentation \& Git Usage - Learning Outcome 1 (15\%)}
\begin{itemize}
	\item Provide the following in your repository \textbf{README.md} file:
	      \begin{itemize}
	      	\item URL to the application on \textbf{Heroku}.
	      	\item How do you setup the environment for development, i.e., after the repository is cloned, what do you need to run the application locally?
	      	\item How do you run the end-to-end \textbf{Cypress} tests?
	      	\item How do you deploy the \textbf{React} application to \textbf{Heroku}?
	      \end{itemize}
\end{itemize}
\begin{itemize}
	\item Your \textbf{Git commit messages} should:
	      \begin{itemize}
	      	\item Reflect the context of each functional requirement change. 
	      	\item Be formatted using the naming conventions outlined in the following:
	      	      \begin{itemize}
	      	      	\item \textbf{Resource:} \small\href{https://dev.to/i5han3/git-commit-message-convention-that-you-can-follow-1709}{https://dev.to/i5han3/git-commit-message-convention-that-you-can-follow-1709}
	      	      \end{itemize} 
	      \end{itemize}
\end{itemize}
          
\subsection*{Additional Information}
\begin{itemize}
    \item Attempt to commit at least \textbf{10} times per week.
    \item \textbf{Do not} rewrite your \textbf{Git} history. It is important that the course lecturer can see how you worked on your assessment over time. 
\end{itemize} 
\end{document}