% Author: Grayson Orr
% Course: ID607001: Introductory Application Development Concepts

\documentclass{article}
\author{}

\usepackage{graphicx}
\usepackage{wrapfig}
\usepackage{enumerate}
\usepackage{hyperref}
\usepackage[margin = 2.25cm]{geometry}
\usepackage[table]{xcolor}
\usepackage{fancyhdr}
\hypersetup{
  colorlinks = true,
  urlcolor = blue
}
\setlength\parindent{0pt}
\pagestyle{fancy}
\fancyhf{}
\rhead{College of Engineering, Construction \& Living Sciences\\Bachelor of Information Technology}
\lfoot{Practical: Node.js REST API Testing\\Version 3, Semester One, 2023}
\rfoot{\thepage}
 
\begin{document}

\begin{figure}
    \centering
    \includegraphics[width=50mm]{../img/logo.png}
\end{figure}

\title{College of Engineering, Construction \& Living Sciences\\Bachelor of Information Technology\\ID607001: Introductory Application Development Concepts\\Level 6, Credits 15\\\textbf{Practical: Node.js REST API Testing}}
\date{}
\maketitle

\section*{Assessment Overview}
In this \textbf{individual} assessment, you will be given a \textbf{Node.js REST API} to \textbf{API test}. You will be required to independently research \& write at least \textbf{40 API tests} using \textbf{Chai} \& \textbf{Mocha}. You will use these dependencies to verify the correctness of the given \textbf{Node.js REST API}. It includes \textbf{CRUD} operations query parameters, status codes \& the shape of response data. In addition, marks will be allocated for code elegance, documentation \& \textbf{Git} usage. 

\section*{Learning Outcome}
At the successful completion of this course, learners will be able to:
\begin{enumerate}
    \item Design \& build secure applications with dynamic database functionality following an appropriate software development methodology.
\end{enumerate}

\section*{Assessments}
\renewcommand{\arraystretch}{1.5}
\begin{tabular}{|c|c|c|c|}
	\hline
	\textbf{Assessment}                                 & \textbf{Weighting} & \textbf{Due Date}            & \textbf{Learning Outcomes} \\ \hline
	\small Practical: Node.js REST API Testing & \small 20\%        & \small 05-05-2023 (Friday at 4.59 PM)   & \small 1                   \\ \hline
	\small Project 1: Node.js REST API                  & \small 30\%        & \small \small 05-05-2023 (Friday at 4.59 PM) & \small 1                   \\ \hline
	\small Project 2: React CRUD                        & \small 50\%        & \small 16-06-2023 (Friday at 4.59 PM)  & \small 1                   \\ \hline
\end{tabular}

\section*{Conditions of Assessment}
You will complete this assessment during your learner-managed time. However, there will be time to discuss the requirements \& your assessment progress during the teaching sessions. This assessment will need to be completed by \textbf{Friday, 05 May 2022 at 4.59 PM}.

\section*{Pass Criteria}
This assessment is criterion-referenced (CRA) with a cumulative pass mark of \textbf{50\%} across all assessments in \textbf{ID607001: Introductory Application Development Concepts}.

\section*{Submission}
You must submit all program files via \textbf{GitHub Classroom}. Here is the URL to the repository you will use for your submission – \href{https://classroom.github.com/a/fZCB58Sl}{https://classroom.github.com/a/fZCB58Sl}. Create a \textbf{.gitignore} and add the ignored files in this resource - \href{https://raw.githubusercontent.com/github/gitignore/main/Node.gitignore}{https://raw.githubusercontent.com/github/gitignore/main/Node.gitignore}. The latest program files in the \textbf{master} or \textbf{main} branch will be used to mark against the \textbf{Functionality} criterion. Please test your \textbf{master} or \textbf{main} branch application before you submit. Partial marks \textbf{will not} be given for incomplete functionality. Late submissions will incur a \textbf{10\% penalty per day}, rolling over at \textbf{5:00 PM}.

\section*{Authenticity}
All parts of your submitted assessment \textbf{must} be completely your work. Do your best to complete this assessment without \textbf{ChatGPT}. You need to demonstrate to the course lecturer that you can meet the learning outcome for this assessment. \\
 
 However, if you get stuck, you can use \textbf{ChatGPT} to help you get unstuck, permitting you acknowledge that you have used \textbf{ChatGPT}. In the assessment's repository \textbf{README.md} file, please include what prompt(s) you provided to \textbf{ChatGPT} \& how you used the response(s) to help you with your work. It also applies to code snippets retrieved from \textbf{StackOverflow} \& \textbf{GitHub}. Failure to do this will result in a mark of \textbf{zero} for this assessment.

\section*{Policy on Submissions, Extensions, Resubmissions \& Resits}
The school's process concerning submissions, extensions, resubmissions \& resits complies with \textbf{Otago Polytechnic | Te Pūkenga} policies. Learners can view policies on the \textbf{Otago Polytechnic | Te Pūkenga} website located at \href{https://www.op.ac.nz/about-us/governance-and-management/policies}{https://www.op.ac.nz/about-us/governance-and-management/policies}. 

\section*{Extensions}
Familiarise yourself with the assessment due date. If you need an extension, contact the course lecturer before the due date. If you require more than a week's extension, a medical certificate or support letter from your manager may be needed.

\section*{Resubmissions}
Learners may be requested to resubmit an assessment following a rework of part/s of the original assessment. Resubmissions are to be completed within a negotiable short time frame \& usually \textbf{must} be completed within the timing of the course to which the assessment relates. Resubmissions will be available to learners who have made a genuine attempt at the first assessment opportunity \& achieved a \textbf{D grade (40-49\%)}. The maximum grade awarded for resubmission will be \textbf{C-}.

\section*{Resits}
Resits \& reassessments are not applicable in \textbf{ID607001: Introductory Application Development Concepts}. 

\newpage

\section*{Instructions}
You will need to submit a \textbf{suite of API tests} \& documentation that meet the following requirements:

\subsection*{Functionality - Learning Outcome 1 (60\%)}
\begin{itemize}
    \item \textbf{API tests} are written using \textbf{Mocha} \& \textbf{Chai}.
    \item At least \textbf{40 API tests} verifying the correctness of the following:
          \begin{itemize}
            \item CRUD (create, read, update \& delete) operations.
            \item Validation rules, i.e., checking if field is required, etc.
            \item Query parameters, i.e., filtering, sorting \& paging data.
            \item Status codes, i.e., checking if a response returns 200, 404, etc.
            \item Shape of the data, i.e., does the response data contain a specific field?
          \end{itemize}
\end{itemize}

\subsection*{Code Elegance - Learning Outcome 1 (30\%)}
\begin{itemize}
    \item Use of intermediate variables. No method calls as arguments.
    \item Idiomatic use of control flow, data structures \& in-built functions.
    \item Sufficient modularity, i.e., \textbf{before()} \& \textbf{after()} functions.
    \item Functions \& variables are named appropriately.
    \item File header comments using \textbf{JSDoc}. You \textbf{need} to explain the purpose of each each \textbf{API test} file.
    \item In-line comments using \textbf{JSDoc}. You \textbf{need} to explain complex logic that is not obvious.
    \item \textbf{API test files} are stored in a directory called \textbf{test} located in the root directory. Each file has the file extension - \textbf{.test.js}
    \item Code files are formatted using \textbf{Prettier} \& a \textbf{.prettierrc} file. You \textbf{need} to declare a \textbf{npm} script in your application's \textbf{package.json} file which automates this process. Rules \textbf{should} include:
    \begin{itemize}
      \item Single quote is set to \textbf{true}.
      \item Semi-colon is set to \textbf{false}.
      \item Tab-width is set to \textbf{2}.
    \end{itemize}
    \item Declare a \textbf{npm} script in your application's \textbf{package.json} file that runs the \textbf{API tests} in the testing environment.
    \item \textbf{Prettier}, \textbf{Chai}, \textbf{Chai HTTP} \& \textbf{Mocha} are installed as development dependencies.
    \item No dead or unused code.
    \item Database configured for the testing environment.
    \begin{itemize}
      \item Create a new database in \textbf{MongoDB Atlas} specifically for the testing environment. 
      \item Do \textbf{not} use the database from \textbf{Project 1: Node.js REST API}.
    \end{itemize}
    \item Application's environment variables are stored in a \textbf{.env} file.
    \begin{itemize}
      \item Create a \textbf{example.env} file containing all of the application's environment variables' key. 
      \item Do not include the environment variables' value.
    \end{itemize}
\end{itemize}

\subsection*{Documentation \& Git Usage - Learning Outcome 1 (10\%)}
\begin{itemize}
    \item \textbf{GitHub} repository contains a \textbf{Node.js .gitignore}.
    \item Provide the following in your repository \textbf{README.md} file:
    \begin{itemize} 
      \item What is API testing \& why is it important?
      \item How do you setup the testing environment, i.e., after the repository is cloned, what do you need to do \textbf{before} you run the \textbf{API tests}?
      \item How do you run the \textbf{API tests}?
      \item How do you format the code using \textbf{Prettier}?
    \end{itemize}
    \item Use of \textbf{Markdown}, i.e., bold text, code blocks, etc.
    \item Correct spelling \& grammar.
    \item Your \textbf{Git commit messages} should:
    \begin{itemize}
      \item Reflect the context of each functional requirement change. 
      \item Be formatted using the naming conventions outlined in the following:
            \begin{itemize}
              \item \textbf{Resource:} \small\href{https://dev.to/i5han3/git-commit-message-convention-that-you-can-follow-1709}{https://dev.to/i5han3/git-commit-message-convention-that-you-can-follow-1709}
            \end{itemize} 
    \end{itemize}
\end{itemize}
          
\subsection*{Additional Information}
\begin{itemize}
    \item Attempt to commit at least \textbf{10} times per week.
    \item \textbf{Do not} rewrite your \textbf{Git} history. It is important that the course lecturer can see how you worked on your assessment over time. 
\end{itemize} 

\end{document}
