% Author: Grayson Orr
% Course: IN607: Introductory Application Development Concepts

\documentclass{article}
\author{}

\usepackage{graphicx}
\usepackage{wrapfig}
\usepackage{enumerate}
\usepackage{hyperref}
\usepackage[margin = 2.25cm]{geometry}
\usepackage[table]{xcolor}
\usepackage{fancyhdr}
\hypersetup{
  colorlinks = true,
  urlcolor = blue
}
\setlength\parindent{0pt}
\pagestyle{fancy}
\fancyhf{}
\rhead{College of Engineering, Construction and Living Sciences\\Bachelor of Information Technology}
\lfoot{Project 2: React\\Version 1, Summer School, 2021-2022}
\rfoot{\thepage}
 
\begin{document}

\begin{figure}
    \centering
    \includegraphics[width=50mm]{../img/logo.png}
\end{figure}

\title{College of Engineering, Construction and Living Sciences\\Bachelor of Information Technology\\Year Two - Special Topic\\Level 6, Credits 15\\\textbf{Project 2: React}}
\date{}
\maketitle

\section*{Assessment Overview}
In this assessment, you will develop a \textbf{CRUD} application using \textbf{React} \& deploy it to \textbf{AWS Amplify}. This application will consume either your \textbf{REST} or \textbf{GraphQL} API from the \textbf{Project 1: REST/GraphQL APIs} assessment. The main purpose of this assessment is not just to build a full-stack application, rather demonstrate an ability to decouple the presentation layer (\textbf{frontend}) from the business logic (\textbf{backend}). Also, you will be required to research and implement pagination, deployment, automated code formatting \& end-to-end testing. In addition, marks will be allocated for code elegance, documentation \& \textbf{Git} usage.

\section*{Learning Outcome}
At the successful completion of this course, learners will be able to:
\begin{enumerate}
    \item Design, create \& deploy microservices using a range of industry-relevant technologies. 
    \item Critically reflect on \& evaluate own learning to identify ways of further personal development.  
\end{enumerate}

\section*{Assessment Table}
\renewcommand{\arraystretch}{1.5}
\begin{tabular}{|l|l|l|l|l|}
    \hline
    \vtop{\hbox{\strut \textbf{Assessment}}\hbox{\strut \textbf{Activity}}} & \textbf{Weighting} & \vtop{\hbox{\strut \textbf{Learning}}\hbox{\strut \textbf{Outcome}}} & \vtop{\hbox{\strut \textbf{Assessment}}\hbox{\strut \textbf{Grading Scheme}}} & \vtop{\hbox{\strut \textbf{Completion}}\hbox{\strut \textbf{Requirements}}} \\

    \hline

    \small Project 1: REST/GraphQL APIs                                                        & \small 40\%        & \small 1                                                           & \small CRA                                                                    & \small Cumulative                                                           \\ \hline
    \small Project 2: React                                                       & \small 40\%        & \small 1                                                        & \small CRA                                                                    & \small Cumulative                                                           \\ \hline
    \small Evaluative Conversation                                                       & \small 20\%        & \small 2                                                        & \small CRA                                                                    & \small Cumulative                                                           \\ \hline
\end{tabular}

\section*{Conditions of Assessment}
You will complete this assessment during your learner-managed time, however, there will be availability during the weekly meetings to discuss the requirements. This assessment will need to be completed by \textbf{Friday, 11 February 2022 at 5:00 PM}.

\section*{Pass Criteria}
This assessment is criterion-referenced (CRA) with a cumulative pass mark of \textbf{50\%} across all assessments in \textbf{Year Two - Special Topic}.

\section*{Authenticity}
All parts of your submitted assessment must be completely your work \& any references must be cited appropriately. Provide your references in a \textbf{README.md} file. Failure to do this will result in a mark of \textbf{zero} for this assessment.

\section*{Policy on Submissions, Extensions, Resubmissions \& Resits}
The school's process concerning submissions, extensions, resubmissions \& resits complies with \textbf{Otago Polytechnic} policies. Learners can view policies on the \textbf{Otago Polytechnic} website located at \href{https://www.op.ac.nz/about-us/governance-and-management/policies}{https://www.op.ac.nz/about-us/governance-and-management/policies}.

\section*{Submissions}
You must submit all program files via \textbf{GitHub}. You will need to create a new repository \& add \textbf{grayson-orr} as a collaborator. The latest program files in the \textbf{main} branch will be used to mark against the \textbf{Functionality} criterion. Please test your \textbf{main} branch application before you submit. Partial marks \textbf{will not} be given for functionality in other branches. Late submissions will incur a \textbf{10\% penalty per day}, rolling over at \textbf{5:00 PM}.

\section*{Extensions}
Familiarise yourself with the assessment due date. If you need an extension, contact the course lecturer before the due date. If you require more than a week's extension, a medical certificate or support letter from your manager may be needed.

\section*{Resubmissions}
Learners may be requested to resubmit an assessment following a rework of part/s of the original assessment. Resubmissions are to be completed within a negotiable short time frame \& usually must be completed within the timing of the course to which the assessment relates. Resubmissions will be available to learners who have made a genuine attempt at the first assessment opportunity \& achieved a \textbf{D grade (40-49\%)}. The maximum grade awarded for resubmission will be \textbf{C-}.

\section*{Resits}
Resits \& reassessments are not applicable in \textbf{Year Two - Special Topic}. 

\newpage

\section*{Instructions}
You will need to submit an application \& documentation that meet the following requirements:

\subsection*{Functionality - Learning Outcome 1 (45\%)}
\begin{itemize}
        \item \textbf{Authentication}
        \begin{itemize}
            \item Register a new user via a form.
            \item User can login and logout. 
        \end{itemize}
        \item \textbf{CRUD}
        \begin{itemize}
            \item Request \textbf{API} data from at least three \textbf{API} resource groups using \textbf{Axios}.
            \item Create new \textbf{API} data via a form. You can display the form on the page or in a modal. 
            \item View \textbf{API} data in a table.
            \item View \textbf{API} data in a table using an id. For example, \textbf{/institutions/1} would return \textbf{API} data for that specific \textbf{Institutions} object.
            \item Update \textbf{API} data via a form. Similar to creating \textbf{API} data, you can display the form on the page or in a modal. 
            \item Delete \textbf{API} data. Prompt the user for deletion. You \textbf{can} use the in-built \textbf{confirm() JavaScript} function. 
            \item Incorrectly formatted form field values handled gracefully using validation error messages, i.e., \textbf{first name} form field is required.
        \end{itemize}
        \item Paginate \textbf{API} data across several pages with \textbf{next} \& \textbf{previous} buttons or links. You can choose the number of \textbf{API} data per page.
        \item Search \textbf{API} data via a search bar.
        \item User-interface is visually attractive with a coherent graphical theme \& style using \textbf{Material Design}.
        \item Application deployed to \textbf{AWS Amplify}. 
        \item \textbf{Cypress} end-to-end tests that test the register, login and logout functionality. You \textbf{must} declare a \textbf{npm} script in your application's \textbf{package.json} file which automates this process.
\end{itemize}

\subsection*{Code Elegance - Learning Outcome 1 (45\%)}
\begin{itemize}
    \item Idiomatic use of control flow, data structures \& in-built functions.
    \item Sufficient code modularity, i.e., UI split into independent reusable pieces.
    \item Components written as functional, not class.
    \item Adheres to a client-server architecture, i.e., the presentation layer is separate from the business logic.
    \item Function header \& in-line comments explaining complex logic.
    \item Code files are formatted using \textbf{Prettier}. You \textbf{must} declare a \textbf{npm} script in your application's \textbf{package.json} file which automates this process. Rules \textbf{must} include:
    \begin{itemize}
        \item Single quote is set to \textbf{true}.
        \item Semi-colon is set to \textbf{false}.
        \item Tab-width is set to \textbf{2}.
    \end{itemize}
    \item \textbf{Prettier} \& \textbf{Cypress} are installed as development dependencies.
    \begin{itemize}
        \item \textbf{Resource:} \small\href{https://docs.npmjs.com/specifying-dependencies-and-devdependencies-in-a-package-json-file}{https://docs.npmjs.com/specifying-dependencies-and-devdependencies-in-a-package-json-file}
    \end{itemize}
    \item No dead or unused code.
\end{itemize}

\subsection*{Documentation \& Git Usage - Learning Outcome 1 (10\%)}
\begin{itemize}
    \item Provide the following in your repository \textbf{README.md} file:
    \begin{itemize}
        \item URL to the application on \textbf{AWS Amplify}.
        \item How do you setup the environment for development, i.e., after the repository is cloned, what do you need to run the application locally?
        \item How do you deploy the application to \textbf{AWS Amplify}?
        \item How do you run the \textbf{Cypress} tests?
    \end{itemize}
    \item Commit messages \textbf{must}:
    \begin{itemize}
      \item Reflect the context of each functional requirement change. 
      \item Be formatted using the naming conventions outlined in the following:
      \begin{itemize}
        \item \textbf{Resource:} \small\href{https://dev.to/i5han3/git-commit-message-convention-that-you-can-follow-1709}{https://dev.to/i5han3/git-commit-message-convention-that-you-can-follow-1709}
      \end{itemize} 
    \end{itemize}
  \end{itemize}
  
  \subsection*{Additional Information}
  \begin{itemize}
    \item You \textbf{must} commit at least \textbf{five} times per week. By the end of this assessment, you should have at least \textbf{50} commits.
    \item \textbf{Do not} rewrite your \textbf{Git} history. It is important that the course lecturer can see how you worked on your assessment over time. 
  \end{itemize}
\end{document}