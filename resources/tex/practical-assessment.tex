% Author: Grayson Orr
% Course: IN607: Introductory Application Development Concepts

\documentclass{article}
\author{}

\usepackage{graphicx}
\usepackage{wrapfig}
\usepackage{enumerate}
\usepackage{hyperref}
\usepackage[margin = 2.25cm]{geometry}
\usepackage[table]{xcolor}
\usepackage{fancyhdr}
\hypersetup{
  colorlinks = true,
  urlcolor = blue
}
\setlength\parindent{0pt}
\pagestyle{fancy}
\fancyhf{}
\rhead{College of Engineering, Construction and Living Sciences\\Bachelor of Information Technology}
\lfoot{Practical\\Version 1, Semester One, 2021}
\rfoot{\thepage}
 
\begin{document}

\begin{figure}
	\centering
	\includegraphics[width=50mm]{./img/logo.png}
\end{figure}

\title{College of Engineering, Construction and Living Sciences\\Bachelor of Information Technology\\IN607: Introductory Application Development Concepts\\Level 6, Credits 15\\\textbf{Practical}}
\date{}
\maketitle

\section*{Assessment Overview}
In this assessment, you will develop \& deploy an \textbf{API} using \textbf{Laravel} \& \textbf{Heroku}. The \textbf{API's} theme could be on sport, culture, food or something else you are interesting in. Your \textbf{API} data will be stored in a \textbf{MySQL} database for \textbf{development} \& \textbf{Heroku PostgreSQL} database for \textbf{production}. The main purpose of the assessment is to demonstrate your ability to develop a complex \textbf{API} using advanced features such as filtering, sorting \& paging. In addition, marks will be allocated for code elegance, documentation \& \textbf{Git} usage. 

\section*{Learning Outcomes}
At the successful completion of this course, learners will be able to:
\begin{enumerate}
	\item Design \& build usable, secure \& attractive applications with dynamic database functionality following an appropriate software development methodology.
\end{enumerate}

\section*{Assessment Table}
\renewcommand{\arraystretch}{1.5}
\begin{tabular}{|l|l|l|l|l|}
	\hline
	\vtop{\hbox{\strut \textbf{Assessment}}\hbox{\strut \textbf{Activity}}} & \textbf{Weighting} & \vtop{\hbox{\strut \textbf{Learning}}\hbox{\strut \textbf{Outcomes}}} & \vtop{\hbox{\strut \textbf{Assessment}}\hbox{\strut \textbf{Grading Scheme}}} & \vtop{\hbox{\strut \textbf{Completion}}\hbox{\strut \textbf{Requirements}}} \\

	\hline

	\small Practical                                                        & \small 20\%        & \small 1                                                           & \small CRA                                                                    & \small Cumulative                                                           \\ \hline
	\small Project                                                          & \small 80\%        & \small 1                                                        & \small CRA                                                                    & \small Cumulative                                                           \\ \hline
\end{tabular}

\section*{Conditions of Assessment}
You will complete this assessment during your learner managed time, however, there will be availability during the teaching sessions to discuss the requirements \& your progress of this assessment. This assessment will need to be completed by \textbf{Friday, 07 May 2021 at 5:00 PM}.

\section*{Pass Criteria}
This assessment is criterion-referenced (CRA) with a cumulative pass mark of \textbf{50\%} over all assessments in \textbf{IN607: Introductory Application Development Concepts}.

\section*{Authenticity}
All parts of your submitted assessment must be completely your work \& any references must be cited appropriately. Provide your references in a \textbf{README.md} file. Failure to do this will result in a mark of \textbf{zero} for this assessment.

\section*{Policy on Submissions, Extensions, Resubmissions \& Resits}
The school's process concerning submissions, extensions, resubmissions \& resits complies with \textbf{Otago Polytechnic} policies. Learners can view policies on the \textbf{Otago Polytechnic} website located at \href{https://www.op.ac.nz/about-us/governance-and-management/policies}{https://www.op.ac.nz/about-us/governance-and-management/policies}.

\section*{Submissions}
You must submit all program files via \textbf{GitHub Classroom}. Here is the URL to the repository you will use for your submission – \href{https://classroom.github.com/a/ww3bvOnY}{https://classroom.github.com/a/ww3bvOnY}. The latest program files in the \textbf{main} branch will be used to run your application. Late submissions will incur a \textbf{10\% penalty per day}, rolling over at \textbf{5:00 PM}.

\section*{Extensions}
Familiarise yourself with the assessment due date. If you need an extension, contact the course lecturer before the due date. If you require more than a week's extension, a medical certificate or support letter from your manager may be needed.

\section*{Resubmissions}
Learners may be requested to resubmit an assessment following a rework of part/s of the original assessment. Resubmissions are to be completed within a negotiable short time frame \& usually must be completed within the timing of the course to which the assessment relates. Resubmissions will be available to learners who have made a genuine attempt at the first assessment opportunity \& achieved a \textbf{D grade (40-49\%)}. The maximum grade awarded for resubmission will be \textbf{C-}.

\section*{Resits}
Resits \& reassessments are not applicable in \textbf{IN607: Introductory Application Development Concepts}. 

\newpage

\section*{Instructions}
\textbf{Note:} you are not allowed to submit the code snippets provided to you in the teaching sessions, i.e., \textbf{Student} \& \textbf{Institution}. \\

You will need to submit an application \& documentation that meet the following requirements:

\subsection*{Functionality - Learning Outcomes 1 (40\%)}
\begin{itemize}
	\item \textbf{API} can run locally \& display data without modification.
	\item Three \textbf{Models} which have at least create, read, write \& delete functionality. For deletion, if the child table has a foreign key, you must use an on cascade delete, i.e., if data from the parent table is deleted, then data from the child is deleted as well.
	\item Filter, sort \& page \textbf{API} data from two or more \textbf{Models} using query parameters.
	\item Custom validation rules \& messages applied to each \textbf{Model}, i.e., \textbf{first name} is required. Return a validation message if a value for \textbf{first name} is not provided.
	\item HTTP error handling, i.e., if a student does not exist, return a \textbf{404} HTTP status code.
	\item At least \textbf{20} \textbf{API} tests that verify the correctness of the \textbf{API}.
	\item Deployed to \textbf{Heroku}. The application must be usable i.e., a user should be able to interact with your \textbf{API}.
	\item \textbf{API} data is stored in \textbf{MySQL} for \textbf{development} \& \textbf{PostgreSQL} for \textbf{production}.
	\item Each database table is seeded with their own \textbf{JSON} file.
\end{itemize}

\subsection*{Code Elegance - Learning Outcomes 1 (45\%)}
\begin{itemize}
	\item Use of intermediate variables. No method calls as arguments.
	\item Idiomatic use of control flow, data structures \& in-built functions.
	\item Sufficient code modularity, i.e., each \textbf{Model} should have their own \textbf{Controller} class.
	\item Adheres to an \textbf{OO} architecture, i.e., classes, methods \& variables are named appropriately. Methods are assigned to the correct class.
	\item Efficient algorithmic approach, i.e., using the appropriate \textbf{Eloquent} function when querying your \textbf{Models}.
	\item \textbf{API} resource groups named with a plural noun instead of a verb, i.e., \textbf{/api/students} not \textbf{/api/student}.
	\item If necessary, in-line comments explaining complex logic, i.e., an \textbf{Eloquent} function may need additional explanation.
	\item Code files are formatted.
	\item No dead or unused code.
	\item \textbf{Models} contain the appropriate fields, behaviours \& relationships.
	\item Databases configured for development \& production environments.
\end{itemize}

\subsection*{Documentation \& Git Usage - Learning Outcomes 1 (15\%)}
\begin{itemize}
	\item Provide the following in your repository \textbf{README.md} file:
	      \begin{itemize}
		      \item URL to the \textbf{API} on \textbf{Heroku}.
		      \item URL to the \textbf{API} documentation on \textbf{Postman}.
		      \item How do you setup the environment for development, i.e., after the repository is cloned, what do you need to run the \textbf{API} locally?
					\item How do you run the \textbf{API} tests?
					\item How do you deploy the \textbf{API} to \textbf{Heroku}?
	      \end{itemize}
			\end{itemize}
			\begin{itemize}
	\item \textbf{API} documented using \textbf{Postman}.
	\begin{itemize}
		\item \textbf{Resource:} \footnotesize\href{https://learning.postman.com/docs/publishing-your-api/documenting-your-api}{https://learning.postman.com/docs/publishing-your-api/documenting-your-api}
	\end{itemize}
	\item Commit messages must reflect the context of each functional requirement change. \textbf{Do not} rewrite your \textbf{Git} history. It is important that the course lecturer can see how you worked on your assessment over time.
	      \begin{itemize}
		      \item \textbf{Resource:} \footnotesize\href{https://freecodecamp.org/news/writing-good-commit-messages-a-practical-guide}{https://freecodecamp.org/news/writing-good-commit-messages-a-practical-guide}
	      \end{itemize}
\end{itemize}
\end{document}
