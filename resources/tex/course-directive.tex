% Author: Grayson Orr
% Course: ID607001: Introduction Application Development Concepts

\documentclass{article}
\author{}

\usepackage{graphicx}
\usepackage{wrapfig}
\usepackage{enumerate}
\usepackage{hyperref}
\usepackage[margin = 2.25cm]{geometry}
\usepackage[table]{xcolor}
\hypersetup{
  colorlinks = true,
  urlcolor = blue
}
\setlength\parindent{0pt}

\begin{document}

\begin{figure}
	\includegraphics[width=50mm]{../img/logo.png}
\end{figure}

\title{Course Directive\\ID607001: Introductory Application Development Concepts\\Semester One, 2022}
\date{}
\maketitle

\section*{Course Information}
\begin{tabular}{ll}
	Credits:      & 15 Credits                                                             \\
	Prerequisite: & IN/D511001: Programming 2                                                   \\
	Timetable:    & Stream A - Tuesday 1 PM D105b \& Friday 8 AM D105b                     \\
	              & Stream B - Tuesday 3 PM D207 \& Friday 1 PM D207                       \\
	              & Lunchtime Tutorial - Thursday 12 PM to 1 PM D207 (Optional attendance)
\end{tabular}

\section*{Teaching Staff}
\begin{tabular}{ll}
	Name:            & Grayson Orr                             \\
	Position:        & Kaiako \& Second/Third-Year Coordinator \\
	Office Location: & D318                                    \\
	Email Address    & grayson.orr@op.ac.nz                    \\
\end{tabular}

\section*{Course Dates}
\begin{tabular}{ll}
	Term 1:             & Monday 21 February - Thursday 14 April \\
	Mid Semester Break: & Monday 18 April - Friday 29 April      \\
	Term 2:             & Monday 02 May - Thursday 23 June       \\
\end{tabular}

\section*{Public Holidays \& Anniversary Days}
A list of public holidays \& anniversary days can be found here - \href{https://www.op.ac.nz/students/importantdates}{https://www.op.ac.nz/students/importantdates}

\section*{Aims}
To introduce the concepts of application development including algorithms, data structures \& design patterns that are required to use a simple, industry-relevant development framework.

\section*{Learning Outcome}
At the successful completion of this course, learners will be able to:
\begin{enumerate}
	\item Design \& build secure applications with dynamic database functionality following an appropriate software development methodology.
\end{enumerate}

\section*{Assessments}
\renewcommand{\arraystretch}{1.5}
\begin{tabular}{|c|c|c|c|}
	\hline
	\textbf{Assessment}                                 & \textbf{Weighting} & \textbf{Due Date}            & \textbf{Learning Outcomes} \\ \hline
	\small Practical: Node.js REST API Testing Research & \small 20\%        & \small 13-05-2022 (Friday)   & \small 1                   \\ \hline
	\small Project 1: Node.js REST API                  & \small 30\%        & \small 14-04-2022 (Thursday) & \small 1                   \\ \hline
	\small Project 2: React CRUD                        & \small 50\%        & \small 21-06-2022 (Tuesday)  & \small 1                   \\ \hline
\end{tabular}


\section*{Provisional Schedule}

\begin{itemize}
	\item \textbf{Assessment Work} is optional attendance
	\item Course \& teaching surveys will be emailed to you in \textbf{Week 12}
\end{itemize}

\renewcommand{\arraystretch}{1.5}
\begin{tabular}{|c|c|c|c|}
	\hline
	\textbf{Week}                  & \textbf{Date}            & \multicolumn{2}{c|}{\textbf{Topics}}                                                                                             \\ \hline
	\footnotesize 1/Tahi           & \footnotesize 21-02-2022 & \multicolumn{2}{c|}{\footnotesize JavaScript Basics 1 - Declarations, Control Flow, Iterations, Functions, Arrays \& Objects}    \\ \hline
	\footnotesize 2/Rua            & \footnotesize 28-02-2022 & \multicolumn{2}{c|}{\footnotesize JavaScript Basics 2 - Map, Filter, Reduce, Error Handling \& Reading in Data}                  \\ \hline
	\footnotesize 3/Toru           & \footnotesize 07-03-2022 & \multicolumn{2}{c|}{\footnotesize Node.js REST API 1 - Introduction, Express, In-Memory Storage, Controllers, Routes \& Postman} \\ \hline
	\footnotesize 4/Whā            & \footnotesize 14-03-2022 & \multicolumn{2}{c|}{\footnotesize Node.js REST API 2 - MongoDB Atlas, Validation \& Relationships}                               \\ \hline
	\footnotesize 5/Rima           & \footnotesize 21-03-2022 & \multicolumn{2}{c|}{\footnotesize Node.js REST API 3 - JSON Web Tokens \& Heroku}                                                \\ \hline
	\footnotesize 6/Ono            & \footnotesize 28-03-2022 &  \multicolumn{2}{c|}{\footnotesize Node.js REST API 4 - Seeding, Rate-Limits \& Postman Documentation}                            \\ \hline
	\footnotesize 7/Whitu          & \footnotesize 04-04-2022 & \multicolumn{2}{c|}{\footnotesize Project 1: Node.js REST API Assessment Work}                                                   \\ \hline
	\footnotesize 8/Waru           & \footnotesize 11-04-2022 & \multicolumn{2}{c|}{\footnotesize Project 1: Node.js REST API Assessment Work}                                                   \\ \hline
	\rowcolor{yellow} \multicolumn{4}{|c|}{\footnotesize Mid Term Break}                                                                                                                         \\ \hline
	\footnotesize 9/Iwa            & \footnotesize 02-05-2022 & \multicolumn{2}{c|}{\footnotesize React 1 - Introduction \& JSX}                                                                 \\ \hline
	\footnotesize 10/Tekau         & \footnotesize 09-05-2022 & \multicolumn{2}{c|}{\footnotesize React 2 - Components, Axios \& Hooks}                                                          \\ \hline
	\footnotesize 11/Tekau mā tahi & \footnotesize 16-05-2022 & \multicolumn{2}{c|}{\footnotesize React 3 - Reactstrap \& React 4 - Authentication}                                                                           \\ \hline
	\footnotesize 12/Tekau mā rua  & \footnotesize 23-05-2022 & \multicolumn{2}{c|}{\footnotesize React 5 - End-to-End Testing with Cypress}                                           \\ \hline
	\footnotesize 13/Tekau mā toru & \footnotesize 30-05-2022 & \multicolumn{2}{c|}{\footnotesize Project 2: React CRUD Assessment Work}                                                     \\ \hline
	\footnotesize 14/Tekau mā whā  & \footnotesize 06-06-2022 & \multicolumn{2}{c|}{\footnotesize Project 2: React CRUD Assessment Work}                                                         \\ \hline
	\footnotesize 15/Tekau mā rima & \footnotesize 13-06-2022 & \multicolumn{2}{c|}{\footnotesize Project 2: React CRUD Assessment Work}                                                         \\ \hline
	\footnotesize 16/Tekau mā ono  & \footnotesize 20-06-2022 & \multicolumn{2}{c|}{\footnotesize Project 2: React CRUD Assessment  Work}                                                        \\ \hline
\end{tabular}

\section*{Resources}

\subsection*{Software}
This paper will be taught using \textbf{Microsoft Visual Studio Code} \* \textbf{Node.js}. An installer for \textbf{Microsoft Visual Studio Code} \& \textbf{Node.js} are available - \href{https://code.visualstudio.com/download}{https://code.visualstudio.com/download} \& \href{https://nodejs.org/en/download}{https://nodejs.org/en/download}. Please refer any problems with downloads or installers to \textbf{Rob Broadley} in D205a.

\subsection*{Readings}
No textbook is required for this course. URLs to useful resources will be provided in the lecture notes.

\section*{Course Requirements \& Expectations}

\subsection*{Learning Hours}
This course requires \textbf{150 hours} of learning. This time includes \textbf{64 hours} of timetabled class time, \& \textbf{86 hours} of self-directed reading, preparation \& completion of assessments.

\subsection*{Learning \& Teaching Methods}
From \textbf{Week Two} onwards, the lectured course material will be pre-recorded \& available to you via \textbf{Microsoft Teams}. You are \textbf{required} to view the recording prior to attending the class. Class time will consist of discussions \& application development work.  

\subsection*{Criteria for Passing}
To pass this paper, you must achieve a cumulative pass mark of \textbf{50\%} over all assessments. There are no reassessments or resits.

\subsection*{Attendance}
\begin{itemize}
	\item Learners are expected to attend all classes, including lectures \& labs.
	\item If you cannot attend for a few days for any reason, contact the course.
\end{itemize}

\subsection*{Communication}
\textbf{Microsoft Outlook/Teams} are the official communication channels for this course. It is your responsibility to regularly check \textbf{Microsoft Outlook/Teams} \& \href{https://github.com/otago-polytechnic-bit-courses/ID607001-intro-app-dev-concepts}{GitHub} for important course material, including changes to class scheduling or assessment details. Not checking will not be accepted as an excuse.

\subsection*{Snow Days/Polytechnic Closure}
In the event \textbf{Otago Polytechnic | Te Kura Matatini ki Otago} is closed or has a delayed opening because of snow or bad weather, you should not attempt to attend class if it is unsafe to do so. It is possible that the teaching staff will not be able to attend either, so classes will not physically be meeting. However, this does not become a holiday. Rather, the course material will be made available on \href{https://github.com/otago-polytechnic-bit-courses/ID607001-intro-app-dev-concepts}{GitHub} for classes affected by the closure. You are responsible for any course material presented in this manner. Information about closure will be posted on the \textbf{Otago Polytechnic | Te Kura Matatini ki Otago Facebook} page \href{https://www.facebook.com/OtagoPoly}{https://www.facebook.com/OtagoPoly}.

\subsection*{Group Work \& Originality}
Learners in the \textbf{Bachelor of Information Technology} programme are expected to hand in original work. Learners are encouraged to discuss assessments with their fellow learners, however, all assessments are to be completed as individual works unless group work is explicitly required (i.e. if it doesn't say it is group work then it is not group work - even if a group consultation was involved). Failure to submit your original work will be treated as plagiarism.

\subsection*{Referencing}
Appropriate referencing is required for all work. Referencing standards will be specified by the teaching staff.

\subsection*{Plagiarism}
Plagiarism is submitting someone elses work as your own. Plagiarism offences are taken seriously \& an assessment that has been plagiarised may be awarded a zero mark. A definition of plagiarism is in the Student Handbook, available online or at the school office.

\subsection*{Submission Requirements}
All assessments are to be submitted by the time, date, \& method given when the assessment is issued. Failure to meet all requirements will result in a penalty of up to \textbf{10\%} per day (including weekends).

\subsection*{Extensions}
Extensions are only available for unusual circumstances. These must be applied for, \& approved, before the submission date.

\subsection*{Impairment}
In case of sickness contact the teaching staff or \textbf{Head of Information Technology (Michael Holtz)} as soon as possible, preferably before the assessment is due. The policy regarding the granting of a mark that considers impaired performance requires a medical certificate \& a medical practitioner’s signature on a form. You may refer to the guide on impaired performance on the student handbook.

\subsection*{Appeals}
If you are concerned about any aspect of your assessment, approach the teaching staff in the first instance. We support an open-door policy \& aim to resolve issues promptly. Further support is available from the \textbf{Head of Information Technology (Michael Holtz)} \& \textbf{Second/Third-Year Coordinator (Grayson Orr)}. \textbf{Otago Polytechnic | Te Kura Matatini ki Otago} has a formal process for academic appeals if necessary.

\subsection*{Other Documents}
Regulatory documents relating to this course can be found on the \textbf{Otago Polytechnic | Te Kura Matatini ki Otago} website.

\end{document}
