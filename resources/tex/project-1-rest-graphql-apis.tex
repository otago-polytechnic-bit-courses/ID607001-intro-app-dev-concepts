% Author: Grayson Orr
% Course: IN607: Introductory Application Development Concepts

\documentclass{article}
\author{}

\usepackage{graphicx}
\usepackage{wrapfig}
\usepackage{enumerate}
\usepackage{hyperref}
\usepackage[margin = 2.25cm]{geometry}
\usepackage[table]{xcolor}
\usepackage{fancyhdr}
\hypersetup{
  colorlinks = true,
  urlcolor = blue
}
\setlength\parindent{0pt}
\pagestyle{fancy}
\fancyhf{}
\rhead{College of Engineering, Construction and Living Sciences\\Bachelor of Information Technology}
\lfoot{Project 1: REST/GraphQL APIs\\Version 1, Summer School, 2021-2022}
\rfoot{\thepage}
 
\begin{document}

\begin{figure}
    \centering
    \includegraphics[width=50mm]{../img/logo.png}
\end{figure}

\title{College of Engineering, Construction and Living Sciences\\Bachelor of Information Technology\\Year Two - Special Topic\\Level 6, Credits 15\\\textbf{Project 1: REST/GraphQL APIs}}
\date{}
\maketitle

\section*{Assessment Overview}
In this assessment, you will develop two types of \textbf{APIs} using \textbf{REST} \& \textbf{GraphQL}. You will then deploy these \textbf{APIs} to \textbf{AWS Amplify}. You will choose the theme of your \textbf{API}. It could be on sport, culture, food or something else you are interested in. Your \textbf{API} data will be stored in a \textbf{MongoDB Atlas} database. The main purpose of this assessment is to demonstrate your ability to develop \textbf{APIs} using modern technologies as well as high-level concepts such as queries, authentication, relationships, validation, seeding, rate limits, filtering, sorting \& pagination. In addition, marks will be allocated for code elegance, documentation \& \textbf{Git} usage. 

\section*{Learning Outcome}
At the successful completion of this course, learners will be able to:
\begin{enumerate}
    \item Design, create \& deploy microservices using a range of industry-relevant technologies. 
    \item Critically reflect on \& evaluate own learning to identify ways of further personal development.  
\end{enumerate}

\section*{Assessment Table}
\renewcommand{\arraystretch}{1.5}
\begin{tabular}{|l|l|l|l|l|}
    \hline
    \vtop{\hbox{\strut \textbf{Assessment}}\hbox{\strut \textbf{Activity}}} & \textbf{Weighting} & \vtop{\hbox{\strut \textbf{Learning}}\hbox{\strut \textbf{Outcome}}} & \vtop{\hbox{\strut \textbf{Assessment}}\hbox{\strut \textbf{Grading Scheme}}} & \vtop{\hbox{\strut \textbf{Completion}}\hbox{\strut \textbf{Requirements}}} \\

    \hline

    \small Project 1: REST/GraphQL APIs                                                        & \small 40\%        & \small 1                                                           & \small CRA                                                                    & \small Cumulative                                                           \\ \hline
    \small Project 2: React                                                       & \small 40\%        & \small 1                                                        & \small CRA                                                                    & \small Cumulative                                                           \\ \hline
    \small Evaluative Conversation                                                       & \small 20\%        & \small 2                                                        & \small CRA                                                                    & \small Cumulative                                                           \\ \hline
\end{tabular}

\section*{Conditions of Assessment}
You will complete this assessment during your learner-managed time, however, there will be availability during the weekly meetings to discuss the requirements. This assessment will need to be completed by \textbf{Friday, 11 February 2022 at 5:00 PM}.

\section*{Pass Criteria}
This assessment is criterion-referenced (CRA) with a cumulative pass mark of \textbf{50\%} across all assessments in \textbf{Year Two - Special Topic}.

\section*{Authenticity}
All parts of your submitted assessment must be completely your work \& any references must be cited appropriately. Provide your references in a \textbf{README.md} file. Failure to do this will result in a mark of \textbf{zero} for this assessment.

\section*{Policy on Submissions, Extensions, Resubmissions \& Resits}
The school's process concerning submissions, extensions, resubmissions \& resits complies with \textbf{Otago Polytechnic} policies. Learners can view policies on the \textbf{Otago Polytechnic} website located at \href{https://www.op.ac.nz/about-us/governance-and-management/policies}{https://www.op.ac.nz/about-us/governance-and-management/policies}.

\section*{Submissions}
You must submit all program files via \textbf{GitHub}. You will need to create a new repository \& add \textbf{grayson-orr} as a collaborator. The latest program files in the \textbf{main} branch will be used to mark against the \textbf{Functionality} criterion. Please test your \textbf{main} branch application before you submit. Partial marks \textbf{will not} be given for functionality in other branches. Late submissions will incur a \textbf{10\% penalty per day}, rolling over at \textbf{5:00 PM}.

\section*{Extensions}
Familiarise yourself with the assessment due date. If you need an extension, contact the course lecturer before the due date. If you require more than a week's extension, a medical certificate or support letter from your manager may be needed.

\section*{Resubmissions}
Learners may be requested to resubmit an assessment following a rework of part/s of the original assessment. Resubmissions are to be completed within a negotiable short time frame \& usually must be completed within the timing of the course to which the assessment relates. Resubmissions will be available to learners who have made a genuine attempt at the first assessment opportunity \& achieved a \textbf{D grade (40-49\%)}. The maximum grade awarded for resubmission will be \textbf{C-}.

\section*{Resits}
Resits \& reassessments are not applicable in \textbf{Year Two - Special Topic}. 

\newpage

\section*{Instructions}
You will need to submit a \textbf{REST} \& \textbf{GraphQL API}, \& documentation that meet the following requirements:

\subsection*{Functionality - Learning Outcome 1 (45\%)}
\begin{itemize}
    \item \textbf{APIs} are developed using \textbf{Node.js}.
    \item \textbf{APIs} can run locally without modification.
    \item \textbf{REST API}
    \begin{itemize}
      \item Three \textbf{collections} containing at least four \textbf{fields} of data which you can interact with.  
      \item You must have a range of different data types, i.e., all \textbf{fields} of data can not be of type \textbf{string}.
      \item Each \textbf{collection} must have a \textbf{controller} containing \textbf{CRUD} (create, read, write \& delete) functionality. 
      \item Custom validation when creating \& updating a \textbf{document}.
      \item Each \textbf{collection} must be seeded with a \textbf{JSON} file. \textbf{Note:} It should be testing data.
      \item API version set to v1. For example, an endpoint should look like - \textbf{/api/v1/institutions}
      \item Return an appropriate status code \& message when performing \textbf{CRUD} actions. For example, when a \textbf{document} is created, return 200 \& \textbf{document} successfully created. 
      \item Return an appropriate message if a query does not return any data.
      \item Filter \& sort \textbf{API} data on all \textbf{fields} using query parameters. A user should be able to sort \textbf{API} data in ascending \& descending order.
      \item Paginate the \textbf{API} data so that 25 records are displayed per page.
      \item \textbf{POST}, \textbf{PUT} \& \textbf{DELETE} routes are protected using \textbf{JSON Web Tokens (JWT)}.
      \item Set the \textbf{API} rate limit to 25 requests per minute.
      \item \textbf{API} application deployed to \textbf{AWS Amplify}. The application must be usable i.e., a user should be able to perform requests to your \textbf{API}.
      \item \textbf{API} data is stored in a \textbf{MongoDB Atlas} database.
    \end{itemize}
    \item \textbf{GraphQL API}
    \begin{itemize}
      \item Your \textbf{GraphQL API} must use the \textbf{REST API} above.
      \item Provide five schemas that return different \textbf{API} data.
    \end{itemize}
\end{itemize}

\subsection*{Code Elegance - Learning Outcome 1 (45\%)}
\begin{itemize}
    \item Use of intermediate variables. No method calls as arguments.
    \item Idiomatic use of control flow, data structures \& in-built functions.
    \item Functions \& variables are named appropriately.
    \item Efficient algorithmic approach, i.e., using the appropriate function(s) when querying your \textbf{collections}.
    \item \textbf{API} resource groups named with a plural noun instead of a verb, i.e., \textbf{/api/students} not \textbf{/api/student}. 
    \item Function header comments explain each \textbf{CRUD} action in a \textbf{controller}.
    \item In-line comments explaining complex logic, i.e., a line of code that may need additional explanation.
    \item Code files are formatted using \textbf{Prettier}.
    \item No dead or unused code.
    \item Databases configured for production environment, i.e., do not expose your database credentials.
\end{itemize} 

\subsection*{Documentation \& Git Usage - Learning Outcome 1 (10\%)}
\begin{itemize}
    \item Provide the following in your repository \textbf{README.md} file:
    \begin{itemize}
        \item URL to the \textbf{APIs} on \textbf{AWS Amplify}.
        \item How do you setup the environment for development, i.e., after the repository is cloned, what do you need to run the \textbf{APIs} locally?
        \item How do you deploy the \textbf{APIs} to \textbf{AWS Amplify}?
    \end{itemize}
    \item Commit messages \textbf{must}:
    \begin{itemize}
      \item Reflect the context of each functional requirement change. 
      \item Be formatted using the naming conventions outlined in the following:
      \begin{itemize}
        \item \textbf{Resource:} \small\href{https://dev.to/i5han3/git-commit-message-convention-that-you-can-follow-1709}{https://dev.to/i5han3/git-commit-message-convention-that-you-can-follow-1709}
      \end{itemize} 
    \end{itemize}
  \end{itemize}
  
  \subsection*{Additional Information}
  \begin{itemize}
    \item You \textbf{must} commit at least \textbf{five} times per week. By the end of this assessment, you should have at least \textbf{50} commits.
    \item \textbf{Do not} rewrite your \textbf{Git} history. It is important that the course lecturer can see how you worked on your assessment over time. 
  \end{itemize}
\end{document}