% Author: Grayson Orr
% Course: IN607: Introductory Application Development Concepts

\documentclass{article}
\author{}

\usepackage{graphicx}
\usepackage{wrapfig}
\usepackage{enumerate}
\usepackage{hyperref}
\usepackage[margin = 2.25cm]{geometry}
\usepackage[table]{xcolor}
\usepackage{fancyhdr}
\hypersetup{
  colorlinks = true,
  urlcolor = blue
}
\setlength\parindent{0pt}
\pagestyle{fancy}
\fancyhf{}
\rhead{College of Engineering, Construction and Living Sciences\\Bachelor of Information Technology}
\lfoot{Project 1: Laravel API\\Version 2, Semester Two, 2021}
\rfoot{\thepage}
 
\begin{document}

\begin{figure}
	\centering
	\includegraphics[width=50mm]{../img/logo.png}
\end{figure}

\title{College of Engineering, Construction and Living Sciences\\Bachelor of Information Technology\\IN607: Introductory Application Development Concepts\\Level 6, Credits 15\\\textbf{Project 1: Laravel API}}
\date{}
\maketitle

\section*{Assessment Overview}
In this assessment, you will develop an \textbf{API} using \textbf{Laravel} \& deploy it to \textbf{Heroku}. You will choose the theme of your \textbf{API}. This could be on sport, culture, food or something else you are interested in. Your \textbf{API} data will be stored in a \textbf{MySQL} development database \& \textbf{Heroku PostgreSQL} production database. The main purpose of this assessment is to demonstrate your ability to develop an \textbf{API} using concepts taught in class such as queries, relationships, validation, seeders, resources, caching, observers \& rate limits. However, you will be required to independently research and implement more complex concepts such as messaging, filtering, sorting \& pagination. In addition, marks will be allocated for code elegance, documentation \& \textbf{Git} usage. 

\section*{Learning Outcome}
At the successful completion of this course, learners will be able to:
\begin{enumerate}
	\item Design \& build usable, secure \& attractive applications with dynamic database functionality following an appropriate software development methodology.
\end{enumerate}

\section*{Assessment Table}
\renewcommand{\arraystretch}{1.5}
\begin{tabular}{|l|l|l|l|l|}
	\hline
	\vtop{\hbox{\strut \textbf{Assessment}}\hbox{\strut \textbf{Activity}}} & \textbf{Weighting} & \vtop{\hbox{\strut \textbf{Learning}}\hbox{\strut \textbf{Outcome}}} & \vtop{\hbox{\strut \textbf{Assessment}}\hbox{\strut \textbf{Grading Scheme}}} & \vtop{\hbox{\strut \textbf{Completion}}\hbox{\strut \textbf{Requirements}}} \\

	\hline

	\small Practical: API Testing Research                                                      & \small 20\%        & \small 1                                                           & \small CRA                                                                    & \small Cumulative                                                           \\ \hline
	\small Project 1: Laravel API                                                        & \small 30\%        & \small 1                                                        & \small CRA                                                                    & \small Cumulative                                                           \\ \hline
	\small Project 2: React CRUD                                                        & \small 50\%        & \small 1                                                        & \small CRA                                                                    & \small Cumulative                                                           \\ \hline
\end{tabular}

\section*{Conditions of Assessment}
You will complete this assessment during your learner managed time, however, there will be availability during the teaching sessions to discuss the requirements \& your progress of this assessment. This assessment will need to be completed by \textbf{Friday, 17 September 2021 at 5:00 PM}.

\section*{Pass Criteria}
This assessment is criterion-referenced (CRA) with a cumulative pass mark of \textbf{50\%} over all assessments in \textbf{IN607: Introductory Application Development Concepts}.

\section*{Authenticity}
All parts of your submitted assessment must be completely your work \& any references must be cited appropriately. Provide your references in a \textbf{README.md} file. Failure to do this will result in a mark of \textbf{zero} for this assessment.

\section*{Policy on Submissions, Extensions, Resubmissions \& Resits}
The school's process concerning submissions, extensions, resubmissions \& resits complies with \textbf{Otago Polytechnic} policies. Learners can view policies on the \textbf{Otago Polytechnic} website located at \href{https://www.op.ac.nz/about-us/governance-and-management/policies}{https://www.op.ac.nz/about-us/governance-and-management/policies}.

\section*{Submissions}
You must submit all program files via \textbf{GitHub Classroom}. Here is the URL to the repository you will use for your submission – \href{https://classroom.github.com/a/c1Wxock6}{https://classroom.github.com/a/c1Wxock6}. The latest program files in the \textbf{main} branch will be used to mark against the \textbf{Functionality} criterion. Please test your \textbf{main} branch application before you submit. Partial marks \textbf{are not} given for functionality in other branches. Late submissions will incur a \textbf{10\% penalty per day}, rolling over at \textbf{5:00 PM}.

\section*{Extensions}
Familiarise yourself with the assessment due date. If you need an extension, contact the course lecturer before the due date. If you require more than a week's extension, a medical certificate or support letter from your manager may be needed.

\section*{Resubmissions}
Learners may be requested to resubmit an assessment following a rework of part/s of the original assessment. Resubmissions are to be completed within a negotiable short time frame \& usually must be completed within the timing of the course to which the assessment relates. Resubmissions will be available to learners who have made a genuine attempt at the first assessment opportunity \& achieved a \textbf{D grade (40-49\%)}. The maximum grade awarded for resubmission will be \textbf{C-}.

\section*{Resits}
Resits \& reassessments are not applicable in \textbf{IN607: Introductory Application Development Concepts}. 

\newpage

\section*{Instructions}
You will need to submit an application \& documentation that meet the following requirements:

\subsection*{Functionality - Learning Outcome 1 (40\%)}
\begin{itemize}
	\item \textbf{API} application can run locally without modification.
	\item Five \textbf{Models} containing at least four columns of data which you can interact with. 
	\begin{itemize}
    \item You must have a range of different data types, i.e., all your columns of data can not be of type \textbf{string}.
    \item For updating \& deleting, if the child table has a foreign key, you must use an on cascade update \& an on cascade delete. For example, if data in the parent table is updated, then data in the child/children table is updated as well.
  \end{itemize}
	\item \textbf{Controller} for each \textbf{Model} which have \textbf{CRUD} (create, read, write \& delete) functionality. 
	\item Custom validation when creating \& updating a record. 
	\item Seed each table using a \textbf{Seeder} \& \textbf{JSON} file.
	\item \textbf{Independent Research:} Return an appropriate status code \& message when performing CRUD actions. For example, when a record is created, return 200 \& record successfully created. 
  \item \textbf{Independent Research:} Return an appropriate message if a query does not return any data.
	\item Return data except for \textbf{id}, \textbf{created\_at} \& \textbf{updated\_at} using \textbf{API Resources}.
	\item \textbf{Independent Research:} Return filtered, sorted \& paged \textbf{API} data from your \textbf{Models} using query parameters.
	\item Store data in the cache when a \textbf{GET} request is performed. 
	\item Remove data from the cache when a \textbf{POST} request is performed.  
	\item \textbf{POST}, \textbf{PUT} \& \textbf{DELETE} routes are protected using \textbf{Sanctum}.
	\item Set the \textbf{API} rate limit to 25 requests per minute.
	\item \textbf{API} application deployed to \textbf{Heroku}. The application must be usable i.e., a user should be able to perform requests to your \textbf{API}.
	\item \textbf{API} data is stored in a \textbf{MySQL} development database \& \textbf{Heroku PostgreSQL} production database.	
\end{itemize}

\subsection*{Code Elegance - Learning Outcome 1 (45\%)}
\begin{itemize}
	\item Use of intermediate variables. No method calls as arguments.
	\item Idiomatic use of control flow, data structures \& in-built functions.
	\item Adheres to an \textbf{OO} architecture, i.e., classes, methods \& variables are named appropriately.
	\item Efficient algorithmic approach, i.e., using the appropriate \textbf{Eloquent} function when querying your \textbf{Models}.
	\item \textbf{API} resource groups named with a plural noun instead of a verb, i.e., \textbf{/api/students} not \textbf{/api/student}.
	\item In-line comments explaining complex logic, i.e., an \textbf{Eloquent} function that may need additional explanation.
	\item Code files are formatted.
	\item No dead or unused code.
	\item Databases configured for development \& production environments.
\end{itemize}

\subsection*{Documentation \& Git Usage - Learning Outcome 1 (15\%)}
\begin{itemize}
  \item \textbf{API} documented using \textbf{Postman}.
	\item Provide the following in your repository \textbf{README.md} file:
	      \begin{itemize}
		      \item URL to the \textbf{API} application on \textbf{Heroku}.
		      \item URL to the \textbf{API} documentation on \textbf{Postman}.
		      \item How do you setup the environment for development, i.e., after the repository is cloned, what do you need to run the \textbf{API} application locally?
					\item How do you deploy the \textbf{API} application to \textbf{Heroku}?
	      \end{itemize}
			\end{itemize}
			\begin{itemize}
	\item Commit messages must reflect the context of each functional requirement change. \textbf{Do not} rewrite your \textbf{Git} history. It is important that the course lecturer can see how you worked on your assessment over time.
\end{itemize}
\end{document}