% Author: Grayson Orr
% Course: IN607: Introductory Application Development Concept

\documentclass{article}
\author{}

\usepackage{graphicx}
\usepackage{wrapfig}
\usepackage{enumerate}
\usepackage{hyperref}
\usepackage[margin = 2.25cm]{geometry}
\usepackage[table]{xcolor}
\usepackage{fancyhdr}
\hypersetup{
  colorlinks = true, 
  urlcolor = blue
}
\setlength\parindent{0pt}
\pagestyle{fancy}
\fancyhf{}
\rhead{College of Engineering, Construction \& Living Sciences\\Bachelor of Information Technology}
\lfoot{In-Class Activity: PHP Basics 2\\Version 1, Semester Two, 2021}
\rfoot{\thepage}

\begin{document}

\begin{figure}
    \centering
    \includegraphics[width=50mm]{../img/logo.png}
\end{figure}

\title{College of Engineering, Construction \& Living Sciences\\Bachelor of Information Technology\\IN607: Introductory Application Development Concepts\\Level 6, Credits 15\\\textbf{In-Class Activity: PHP Basics 2}}
\date{}
\maketitle
 
\section*{Instructions}
The purpose of this in-class activity to familiarise yourself with classes, inheritance \& interfaces in \textbf{PHP}.

\section*{Code Review}
You must submit all program files via \textbf{GitHub Classroom}. Here is the URL to the repository you will use for your code review – \href{https://classroom.github.com/a/P656imf2}{https://classroom.github.com/a/P656imf2}. Checkout from the \textbf{main} branch to the \textbf{02-in-class-activity} branch by running the command - \textbf{git checkout 02-in-class-activity}. This branch will be your development branch for this activity. Once you have completed this activity, create a pull request \& assign the \textbf{GitHub} user \textbf{grayson-orr} to a reviewer. \textbf{Do not} merge your own pull request.

\subsection*{Problem 1:} 
Create a \textbf{Cat} class with the private attributes name, age \& breed. For each attribute, create a getter \& setter. Also, create a \textbf{\_\_toString()} special method which returns the following:

\begin{verbatim}
  <?php
      My $this->breed's name is $this->name. S/he is $this->age year(s) old.
  ?>
\end{verbatim}

Create two \textbf{Cat} objects called \textbf{cat\_one} \& \textbf{cat\_two}. Using setters, change \textbf{cat\_one's} name to Fido \& age to 10. Again, using a setter, change \textbf{cat\_two's} breed to \textbf{American Bobtail}. For each \textbf{Cat} object, print its string representation.

\begin{verbatim}
  <?php
      // Write your solution here
  ?>
\end{verbatim}

\subsection*{Problem 2:} 
Create a \textbf{SoftwareDeveloper} \& \textbf{AgileCoach} class which inherits from \textbf{Employee} class. \textbf{SoftwareDeveloper} class has one additional class attribute called \textbf{prog\_lang}. \textbf{AgileCoach} also has one additional class attribute called \textbf{employees} \& three class methods which \textbf{add}, \textbf{remove}, \textbf{search} \& \textbf{show\_all} employees managed by \textbf{AgileCoach}. \textbf{Note:} employees is a list of \textbf{SoftwareDeveloper} objects. \\

Use the three \textbf{SoftwareDeveloper} objects \& \textbf{AgileCoach} object provided to display the expected output.

\begin{verbatim}
  <?php
      class Employee {
          protected $first_name;
          protected $last_name;
          protected $salary;

          public function __construct($first_name, $last_name, $salary) {
              $this->first_name = $first_name;
              $this->last_name = $last_name;
              $this->salary = $salary;
          }

          public function __toString() {
              return $this->first_name . " " . $this->last_name;
          }
      }

      $sft_dev_one = new SoftwareDeveloper("Alfredo", "Boyle", 50000, "C#");
      $sft_dev_two = new SoftwareDeveloper("Malik", "Martin", 55000, "JavaScript");
      $sft_dev_three = new SoftwareDeveloper("Livia", "Martin", 75000, "Kotlin");
      $agile_coach = new AgileCoach("Lillian", "Cunningham", 100000, array($sft_dev_one, $sft_dev_two));

      // Write your solution here

      // Expected output:
      // Malik Martin
      // Livia Martin
      // Alfredo Boyle not found
      // Livia Martin found
  ?>
\end{verbatim}

\subsection*{Problem 3:} 
\textbf{Language} class has no class attributes, but a class method called \textbf{good\_morning}. \textbf{Maori}, \textbf{Spanish} \& \textbf{German} class inherit from \textbf{Language} class. When you run the following code, what is happening \& why is it happening? Refactor the code to display the expected output.

\begin{verbatim}
  <?php
      class Language {
          public function good_morning() {
              throw new Exception("good_morning not implemented");
          }
      }    

      class Maori extends Language {
          public function good_morning() {
              echo "Morena";
          }
      }

      class Spanish extends Language {

      }

      class German extends Language {
          public function good_morning() {
              echo "Guten Morgen";
          }
      }
          
      $maori = new Maori();
      $spanish = new Spanish();
      $german = new German();
      $maori->good_morning();
      $spanish->good_morning();
      $german->good_morning();

      // Expected output:
      // Morena
      // Hola
      // Guten Morgen
  ?>
\end{verbatim} 

\subsection*{Problem 4:} 
Implement the \textbf{push()}, \textbf{pop()}, \textbf{peek()}, \textbf{size()} \& \textbf{show\_all()} methods in the \textbf{Stack} class. \textbf{Note:} \textbf{size()} method returns the length of a \textbf{stack} \& \textbf{show\_all()} method returns the items in a \textbf{stack}. \\

You are probably wondering what a \textbf{stack} is. A \textbf{stack} is an \textbf{Abstract Data Type (ADT)} that is used to store elements in a \textbf{Last In First Out (LIFO)} manner. A \textbf{stack} is a collection of elements where the addition of elements is performed at the end \& the removal of elements is performed at the beginning. A \textbf{stack} has two methods: \textbf{push()} \& \textbf{pop()}. The push method adds an element to the \textbf{stack}. The pop method removes the last element added to the \textbf{stack}. \\

Use the \textbf{Stack} object provided to display the expected output. 

\begin{verbatim}
  <?php
      class Stack {
          private $stack;

          public function __construct() {
              $this->stack = array();
          }

          public function push($item) {
              // Write your code here
          }

          public function pop() {
              // Write your code here
          }

          public function peek() {
              // Write your code here
          }

          public function is_empty() {
              // Write your code here
          }

          public function size() {
              // Write your code here
          }

          public function show_all() {
              // Write your code here
          }

          public function __toString() {
              return $this->stack;
          }
      }   
          
      $stack = new Stack();
      $stack->push("Introductory App Dev Concepts");
      $stack->push("Intermediate App Dev Concepts");
      $stack->push("Advanced App Dev Concepts");

      // Write your solution here

      // Expected output:
      // ["Introductory App Dev Concepts", "Intermediate App Dev Concepts"]
      // Intermediate App Dev Concepts is at the top of the stack
      // There are 2 item(s) in the stack
  ?>
\end{verbatim}

\subsection*{Problem 5:} 
Create an interface called \textbf{App}. This interface has three methods, \textbf{login()}, \textbf{register()} \& \textbf{logout()}. The \textbf{login()} method accepts two arguments, email \& password, \& the \textbf{register()} method accepts three arguments, email, password \& username. \\

Create a class called \textbf{Facebook} which implements \textbf{App}. For each method implemented, echo the following:
\begin{verbatim}
  // login() - Logged in with the email - $this->email.
  // register() - Registered with the email and username - $this->email and $this->username.
  // logout() - User logged out.
\end{verbatim}

\begin{verbatim}
  <?php
      // Write your solution here
  ?>
\end{verbatim}

\end{document}
