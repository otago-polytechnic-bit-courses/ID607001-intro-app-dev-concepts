% Author: Grayson Orr
% Course: IN607: Introductory Application Development Concept

\documentclass{article}
\author{}

\usepackage{graphicx}
\usepackage{wrapfig}
\usepackage{enumerate}
\usepackage{hyperref}
\usepackage[margin = 2.25cm]{geometry}
\usepackage[table]{xcolor}
\usepackage{fancyhdr}
\hypersetup{
  colorlinks = true, 
  urlcolor = blue
}
\setlength\parindent{0pt}
\pagestyle{fancy}
\fancyhf{}
\rhead{College of Engineering, Construction \& Living Sciences\\Bachelor of Information Technology}
\lfoot{In-Class Activity: PHP Basics 1 Answers\\Version 2, Semester Two, 2021}
\rfoot{\thepage}

\begin{document}

\begin{figure}
    \centering
    \includegraphics[width=50mm]{../img/logo.png} 
\end{figure}

\title{College of Engineering, Construction \& Living Sciences\\Bachelor of Information Technology\\IN607: Introductory Application Development Concepts\\Level 6, Credits 15\\\textbf{In-Class Activity: PHP Basics 1 Answers}}
\date{}
\maketitle

\subsection*{Problem 1:} 
\begin{verbatim}
  <?php
      $name = "John";
      $age = 55;
      echo "Hello my name is $name & I am $age years old.";
  ?>
\end{verbatim}

\subsection*{Problem 2:} 
\begin{verbatim}
  <?php
      $x = 1957452;
      $y = 2975635;
      $sum = $x + $y;
      echo "The sum of $x & $y is $sum";
  ?>
\end{verbatim}

\subsection*{Problem 3:} 
\begin{verbatim}
  <?php
      $numbers = array(45.3, 67.5, -45.6, 20.34, -33.0, 45.6);
      $average = array_sum($numbers) / count($numbers);
      echo "Average: $average";
  ?>
\end{verbatim}

\subsection*{Problem 4:}
\begin{verbatim}
  <?php
      function fizzBuzz($num) {
          if ($num % 15 == 0) {
              return "FizzBuzz";
          } elseif ($num % 3 == 0) {
              return "Fizz";
          } elseif ($num % 5 == 0) {
              return "Buzz";
          }
          return $num;
      }

      for ($i = 1; $i <= 15; $i += 2) {
          echo fizzBuzz($i) . "<br>";
      }
  ?>
\end{verbatim}

\subsection*{Problem 5:}
\begin{verbatim}
  <?php  
      $numbers = array(21, 19, 68, 55, 42, 12);
      sort($number);
      foreach ($numbers as $num) {
          if ($num % 2 != 0) {
              echo $num . "<br>";
          }
      }
  ?>
\end{verbatim}

\subsection*{Problem 6:}
\begin{verbatim}
  <?php  
      function is_anagram($string_one, $string_two) {
          if (count_chars($string_one, 1) == count_chars($string_two, 1)) {
              echo "true";
          } else {
              return "false";
          }
      }

      echo is_anagram("elvis", "lives") . "<br>";
      echo is_anagram("cat", "sat");
  ?>
\end{verbatim}

\subsection*{Problem 7:}
\begin{verbatim}
  <?php  
      function convert($hours, $mins) {
          $hours = $hours * 3600;
          $mins = $mins * 60;
          return $mins + $hours;
      }

      echo convert(1, 3);
  ?>
\end{verbatim}

\subsection*{Problem 8:}
Write a function called \textbf{palindrome} which accepts a single parameter called \textbf{string}. In the function block, determine whether or not \textbf{string} is a palindrome. The function should return a \textbf{boolean}. 

\begin{verbatim}
  <?php  
      function palindrome($string) {
          $str = preg_replace('/\W/i', '', strtolower($string));
          if (strrev($str) == $str) { 
              return "true"; 
          } else {
              return "false";
          }
      }

      echo palindrome("A man, a plan, a canal - Panama") . "<br>";
      echo palindrome("Hello, World!");
  ?>
\end{verbatim}

\subsection*{Problem 9:}
Write a function called \textbf{is\_five\_letters} which accepts an \textbf{array} of \textbf{strings}. In the function block, return all words that are exactly \textbf{five} letters.

\begin{verbatim}
  <?php  
      function is_five_letters ($string) {
          for($i = 0; $i < count($string); $i++) {
              if(strlen($string[$i]) == 5) {
                  echo $string[$i];
              }
          }
      }

      is_five_letters(["car", "bike", "truck"]);  
  ?>
\end{verbatim}

\subsection*{Problem 10:}
Write a function that accepts an \textbf{integer}. If the \textbf{integer} is prime, return \textbf{true}, otherwise return \textbf{false}. 

\begin{verbatim}
  <?php  
      function is_prime($prime){
          for($p = 2; $p < $prime; $p++) {
              if($prime % $p == 0) {
                  return "false";
              }
          
              if($p >= sqrt($prime)) {
                  return "true";
              }
          }
      }
      
      echo is_prime(11);
      echo is_prime(18);
  ?>
\end{verbatim}

\end{document}
