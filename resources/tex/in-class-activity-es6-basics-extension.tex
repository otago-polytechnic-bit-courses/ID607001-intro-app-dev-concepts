% Author: Grayson Orr
% Course: ID607001: Introductory Application Development Concept

\documentclass{article}
\author{}

\usepackage{graphicx}
\usepackage{wrapfig}
\usepackage{enumerate}
\usepackage{hyperref}
\usepackage[margin = 2.25cm]{geometry}
\usepackage[table]{xcolor}
\usepackage{fancyhdr}
\hypersetup{
  colorlinks = true, 
  urlcolor = blue
}
\setlength\parindent{0pt}
\pagestyle{fancy}
\fancyhf{}
\rhead{College of Engineering, Construction \& Living Sciences\\Bachelor of Information Technology}
\lfoot{In-Class Activity: ES6 Basics Extension\\Version 1, Semester Two, 2021}
\rfoot{\thepage}

\begin{document}

\begin{figure}
  \centering
  \includegraphics[width=50mm]{../img/logo.png}
\end{figure}

\title{College of Engineering, Construction \& Living Sciences\\Bachelor of Information Technology\\ID607001: Introductory Application Development Concepts\\Level 6, Credits 15\\\textbf{In-Class Activity: ES6 Basics Extension}}
\date{}
\maketitle

\section*{Instructions}
The purpose of this in-class activity is to extend yourself with more complex problems. Also, you will articulate various \textbf{JavaScript} concepts.

\section*{Submission}
You must submit all program files via \textbf{GitHub Classroom}. Here is the URL to the repository you will use for your code review – \href{https://classroom.github.com/a/\_6KSahyX}{https://classroom.github.com/a/\_6KSahyX}. If you wish to have your code reviewed, message the course lecturer on \textbf{Microsoft Teams}.

\section*{Getting Started}
Open your repository in \textbf{Visual Studio Code}. Create a new file called \textbf{03-in-class-activity.js}. In \textbf{03-in-class-activity.js}, add the following: 

\begin{verbatim}
  console.log('Hello, World!')
\end{verbatim}

Open a \textbf{terminal} \& run the following command:

\begin{verbatim}
  node 03-in-class-activity.js
\end{verbatim}

If the output is \textbf{Hello, World!}, then you are ready to start coding.

\subsection*{Problem 1:}
You have been given two \textbf{arrays} containing the lecturer's favourite programming languages. Use the following hints to display the expected output:
\begin{itemize}
  \item Add a specified element to the end of a list.
  \item Add all elements of a specified \textbf{array} to the end of a list.
  \item If present, remove a specified element from a \textbf{array}.
  \item Capitalise the element in the 3rd index.
\end{itemize}

\begin{verbatim}
  const progLangsOne = ['C#', 'JavaScript', 'Kotlin', 'OCaml']
  const progLangsTwo = ['C++', 'Go', 'Swift', 'TypeScript']

  // Write your solution here

  // Expected output:
  // [C#, JavaScript, Kotlin, OCAML, Prolog, C++, Swift]
\end{verbatim}

\subsection*{Problem 2:}
Create an \textbf{arrow function} which simulates the \textbf{Rock, Paper, Scissors} game. The \textbf{arrow function} takes the input of two players (rock, paper or scissors), first parameter from the first player, second from the second player. The function returns the result as such:

\begin{itemize}
  \item First player wins
  \item Second player wins
  \item Draw
\end{itemize}

\begin{verbatim}
  // Write your solution here

  console.log(rockPaperScissor('paper', 'rock'))
  console.log(rockPaperScissor('rock', 'paper'))
  console.log(rockPaperScissor('paper', 'paper'))

  // Expected output:
  // First player wins
  // Second player wins
  // Draw
\end{verbatim}

\subsection*{Problem 3:}
You are given the length of a song in minutes. The format is \textbf{mm:ss}, i.e., minutes:seconds or "01:00". Create an \textbf{arrow function} that takes the song's length \& returns it in seconds. \textbf{Note:} If the number of seconds is ≥ 60, return \textbf{false}.

\begin{verbatim}
  // Write your solution here

  console.log(minToSecs('01:00'))
  console.log(minToSecs('13:56'))
  console.log(minToSecs('10:60'))

  // Expected output:
  // 60
  // 836
  // false
\end{verbatim}

\subsection*{Problem 4:}

\subsection*{Problem 5:}

\subsection*{Problem 6:}

\subsection*{Problem 7:}

\subsection*{Problem 8:}

\subsection*{Problem 9:}

\subsection*{Problem 10:}

\end{document}
