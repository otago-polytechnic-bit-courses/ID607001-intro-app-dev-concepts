% Author: Grayson Orr
% Course: IN607: Introductory Application Development Concepts

\documentclass{article}
\author{}

\usepackage{graphicx}
\usepackage{wrapfig}
\usepackage{enumerate}
\usepackage{hyperref}
\usepackage[margin = 2.25cm]{geometry}
\usepackage[table]{xcolor}
\usepackage{fancyhdr}
\hypersetup{
  colorlinks = true,
  urlcolor = blue
}
\setlength\parindent{0pt}
\pagestyle{fancy}
\fancyhf{}
\rhead{College of Engineering, Construction and Living Sciences\\Bachelor of Information Technology}
\lfoot{Practical: Node.js REST API Testing Research\\Version 1, Semester One, 2022}
\rfoot{\thepage}
 
\begin{document}

\begin{figure}
	\centering
	\includegraphics[width=50mm]{../img/logo.png}
\end{figure}

\title{College of Engineering, Construction and Living Sciences\\Bachelor of Information Technology\\IN607: Introductory Application Development Concepts\\Level 6, Credits 15\\\textbf{Practical: API Testing Research}}
\date{}
\maketitle

\section*{Assessment Overview}
In this assessment, you will be given a \textbf{Node.js REST API} to \textbf{API test}. You will be required to independently research \& write at least \textbf{50 API tests} using \textbf{Mocha} \& \textbf{Chai}. These will be used to verify the correctness of the given \textbf{Node.js REST API}. It includes \textbf{CRUD} operations, query parameters, status codes \& the shape of response data. You will also check the coverage of your \textbf{API tests} using \textbf{nyc}. In addition, marks will be allocated for code elegance, documentation \& \textbf{Git} usage. 

\section*{Learning Outcome}
At the successful completion of this course, learners will be able to:
\begin{enumerate}
	\item Design \& build usable, secure \& attractive applications with dynamic database functionality following an appropriate software development methodology.
\end{enumerate}

\section*{Assessment Table}
\renewcommand{\arraystretch}{1.5}
\begin{tabular}{|l|l|l|l|l|}
	\hline
	\vtop{\hbox{\strut \textbf{Assessment}}\hbox{\strut \textbf{Activity}}} & \textbf{Weighting} & \vtop{\hbox{\strut \textbf{Learning}}\hbox{\strut \textbf{Outcome}}} & \vtop{\hbox{\strut \textbf{Assessment}}\hbox{\strut \textbf{Grading Scheme}}} & \vtop{\hbox{\strut \textbf{Completion}}\hbox{\strut \textbf{Requirements}}} \\
		
	\hline
		
	\small Practical: Node.js REST API Testing Research                     & \small 20\%        & \small 1                                                             & \small CRA                                                                    & \small Cumulative                                                           \\ \hline
	\small Project 1: Node.js REST API                                      & \small 30\%        & \small 1                                                             & \small CRA                                                                    & \small Cumulative                                                           \\ \hline
	\small Project 2: React CRUD                                            & \small 50\%        & \small 1                                                             & \small CRA                                                                    & \small Cumulative                                                           \\ \hline
\end{tabular}

\section*{Conditions of Assessment}
You will complete this assessment during your learner-managed time. However, there will be time to discuss the requirements \& your assessment progress during the teaching sessions. This assessment will need to be completed by \textbf{Friday, 13 May 2022 at 4.59 PM}.

\section*{Pass Criteria}
This assessment is criterion-referenced (CRA) with a cumulative pass mark of \textbf{50\%} across all assessments in \textbf{IN607: Introductory Application Development Concepts}.

\section*{Submissions}
You must submit all program files via \textbf{GitHub Classroom}. Here is the URL to the repository you will use for your submission – \href{https://classroom.github.com/a/Anc\_bYhn}{https://classroom.github.com/a/Anc\_bYhn}. The latest program files in the \textbf{main} branch will be used to mark against the \textbf{Functionality} criterion. Please test your \textbf{main} branch application before you submit. Partial marks \textbf{will not} be given for functionality in other branches. Late submissions will incur a \textbf{10\% penalty per day}, rolling over at \textbf{5:00 PM}.

\section*{Authenticity}
All parts of your submitted assessment must be completely your work. If you use code snippets from \textbf{GitHub}, \textbf{StackOverflow} or other online resource, you \textbf{must} reference it appropriately using \textbf{APA 7th edition}. Provide your references in the \textbf{README.md} file in your repository. Failure to do this will result in a mark of \textbf{zero} for this assessment.

\section*{Policy on Submissions, Extensions, Resubmissions \& Resits}
The school's process concerning submissions, extensions, resubmissions \& resits complies with \textbf{Otago Polytechnic} policies. Learners can view policies on the \textbf{Otago Polytechnic} website located at \href{https://www.op.ac.nz/about-us/governance-and-management/policies}{https://www.op.ac.nz/about-us/governance-and-management/policies}.

\section*{Extensions}
Familiarise yourself with the assessment due date. If you need an extension, contact the course lecturer before the due date. If you require more than a \textbf{seven days} extension, a medical certificate or support letter from your manager may be needed.

\section*{Resubmissions}
Learners may be requested to resubmit an assessment following a rework of part/s of the original assessment. Resubmissions are to be completed within a negotiable short time frame \& usually must be completed within the timing of the course to which the assessment relates. Resubmissions will be available to learners who have made a genuine attempt at the first assessment opportunity \& achieved a \textbf{D grade (40-49\%)}. The maximum grade awarded for resubmission will be \textbf{C-}.

\section*{Resits}
Resits \& reassessments are not applicable in \textbf{IN607: Introductory Application Development Concepts}. 

\newpage

\section*{Instructions}
You will need to submit a \textbf{testing suite} \& documentation that meet the following requirements:

\subsection*{Functionality - Learning Outcome 1 (60\%)}
\begin{itemize}
	\item \textbf{API tests} are written using \textbf{Mocha} \& \textbf{Chai}.
	\item At least \textbf{50 API tests} verifying the correctness of the following:
	      \begin{itemize}
	      	\item CRUD (create, read, update \& delete) functionality.
	      	\item Authentication (register, login \& logout) functionality.
	      	\item Validation rules, i.e., checking if field is required, etc.
	      	\item Query parameters, i.e., filtering \& sorting data.
	      	\item Status codes, i.e., checking if a response returns 200, 404, etc.
	      	\item Shape of the data, i.e., does the response data contain a specific column?
	      \end{itemize}
	\item Code coverage using \textbf{nyc}. 
\end{itemize}

\subsection*{Code Elegance - Learning Outcome 1 (30\%)}
\begin{itemize}
	\item Use of intermediate variables. No method calls as arguments.
	\item Idiomatic use of control flow, data structures \& in-built functions.
	\item Sufficient modularity, i.e., setup method at the beginning of each test case.
	\item Functions \& variables are named appropriately.
	\item File header comment explaining the purpose of each \textbf{API test} file.
	\item Code files are formatted using \textbf{Prettier}. You \textbf{need} to declare a \textbf{npm} script in your application's \textbf{package.json} file that automates this process. Rules \textbf{must} include:
	      \begin{itemize}
	      	\item Single quote is set to \textbf{true}.
	      	\item Semi-colon is set to \textbf{false}.
	      	\item Tab-width is set to \textbf{2}.
	      \end{itemize}
	\item \textbf{Prettier} is installed as a development dependency.
	\item No dead or unused code.
	\item Database configured for testing environment.
\end{itemize}

\subsection*{Documentation \& Git Usage - Learning Outcome 1 (10\%)}
\begin{itemize}
	\item Provide the following in your repository \textbf{README.md} file:
	      \begin{itemize}
	      	\item How do you setup the environment for development, i.e., after the repository is cloned, what do you need to run the the \textbf{API tests} locally?
	      	\item How do you run the \textbf{API tests}?
	      \end{itemize}
\end{itemize}
\begin{itemize}
	\item Your \textbf{Git commit messages} should:
	      \begin{itemize}
	      	\item Reflect the context of each functional requirement change. 
	      	\item Be formatted using the naming conventions outlined in the following:
	      	      \begin{itemize}
	      	      	\item \textbf{Resource:} \small\href{https://dev.to/i5han3/git-commit-message-convention-that-you-can-follow-1709}{https://dev.to/i5han3/git-commit-message-convention-that-you-can-follow-1709}
	      	      \end{itemize} 
	      \end{itemize}
\end{itemize}
          
\subsection*{Additional Information}
\begin{itemize}
	\item Attempt to commit at least \textbf{10} times per week. By the end of this assessment, you should have at least \textbf{40} commits.
	\item \textbf{Do not} rewrite your \textbf{Git} history. It is important that the course lecturer can see how you worked on your assessment over time. 
\end{itemize}
\end{document}
