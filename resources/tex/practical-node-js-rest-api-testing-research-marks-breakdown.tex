% Author: Grayson Orr
% Course: ID607001: Introductory Application Development Concepts

\documentclass{article}
\author{}

\usepackage{graphicx}
\usepackage{wrapfig}
\usepackage{enumerate}
\usepackage{hyperref}
\usepackage[margin = 2.25cm]{geometry}
\usepackage[table]{xcolor}
\usepackage{fancyhdr}
\hypersetup{
  colorlinks = true,
  urlcolor = blue
}
\setlength\parindent{0pt}
\pagestyle{fancy}
\fancyhf{}
\rhead{College of Engineering, Construction \& Living Sciences\\Bachelor of Information Technology}
\lfoot{Practical: Node.js REST API Testing Research Marks Breakdown\\Version 1, Semester One, 2022}
\rfoot{\thepage}
 
\begin{document}

\begin{figure}
  \centering
  \includegraphics[width=50mm]{../img/logo.png}
\end{figure}

\title{College of Engineering, Construction \& Living Sciences\\Bachelor of Information Technology\\ID607001: Introductory Application Development Concepts\\Level 6, Credits 15\\\textbf{Practical: Node.js REST API Testing Research Marks Breakdown}}
\date{}
\maketitle

\subsection*{Functionality - Learning Outcome 1 (60\%)}
\begin{itemize}
  \item \textbf{API tests} are written using \textbf{Mocha} \& \textbf{Chai}.
  \item At least \textbf{50 API tests} verifying the correctness of the following:
        \begin{itemize}
          \item CRUD (create, read, update \& delete) operations.
          \item Authentication (register, login \& logout).
          \item Validation rules, i.e., checking if field is required, etc.
          \item Query parameters, i.e., filtering \& sorting data.
          \item Status codes, i.e., checking if a response returns 200, 404, etc.
          \item Shape of the data, i.e., does the response data contain a specific column?
        \end{itemize}
  \item Code coverage using \textbf{nyc}.
\end{itemize}

\subsection*{Code Elegance - Learning Outcome 1 (30\%)}
\begin{itemize}
  \item Use of intermediate variables. No method calls as arguments.
  \item Idiomatic use of control flow, data structures \& in-built functions.
  \item Sufficient modularity, i.e., setup method at the beginning of each test case.
  \item Functions \& variables are named appropriately.
  \item File header comment explaining the purpose of each \textbf{API test} file.
  \item Code files are formatted using \textbf{Prettier}. You \textbf{need} to declare a \textbf{npm} script in your application's \textbf{package.json} file that automates this process. Rules \textbf{must} include:
        \begin{itemize}
          \item Single quote is set to \textbf{true}.
          \item Semi-colon is set to \textbf{false}.
          \item Tab-width is set to \textbf{2}.
        \end{itemize}
  \item \textbf{Prettier} \& \textbf{nyc} are installed as development dependencies.
  \item No dead or unused code.
  \item Database configured for testing environment.
\end{itemize}

\subsection*{Documentation \& Git Usage - Learning Outcome 1 (10\%)}
\begin{itemize}
  \item Provide the following in your repository \textbf{README.md} file:
        \begin{itemize}
          \item How do you setup the environment for development, i.e., after the repository is cloned, what do you need to run the the \textbf{API tests} locally?
          \item How do you run the \textbf{API tests}?
        \end{itemize}
  \item Use of \textbf{Markdown}, i.e., bold text, code blocks, etc.
  \item Correct spelling \& grammar.
  \item Your \textbf{Git commit messages} should:
        \begin{itemize}
          \item Reflect the context of each functional requirement change.
          \item Be formatted using the naming conventions outlined in the following:
                \begin{itemize}
                  \item \textbf{Resource:} \small\href{https://dev.to/i5han3/git-commit-message-convention-that-you-can-follow-1709}{https://dev.to/i5han3/git-commit-message-convention-that-you-can-follow-1709}
                \end{itemize}
        \end{itemize}
\end{itemize}
\end{document}
