% Author: Grayson Orr
% Course: IN607: Introductory Application Development Concepts

\documentclass{article}
\author{}

\usepackage{graphicx}
\usepackage{wrapfig}
\usepackage{enumerate}
\usepackage{hyperref}
\usepackage[margin = 2.25cm]{geometry}
\usepackage[table]{xcolor}
\usepackage{fancyhdr}
\hypersetup{
  colorlinks = true,
  urlcolor = blue
}
\setlength\parindent{0pt}
\pagestyle{fancy}
\fancyhf{}
\rhead{College of Engineering, Construction \& Living Sciences\\Bachelor of Information Technology}
\lfoot{Activity 01: PHP Basics\\Version 1, Semester Two, 2021}
\rfoot{\thepage}

\begin{document}

\begin{figure}
    \centering
    \includegraphics[width=50mm]{../../resources/img/logo.png}
\end{figure}

\title{College of Engineering, Construction \& Living Sciences\\Bachelor of Information Technology\\IN607: Introductory Application Development Concepts\\Level 6, Credits 15\\\textbf{Activity 01: PHP Basics}}
\date{}
\maketitle

\section*{Assessment Overview}
In this assessment, 

\section*{Learning Outcomes}
At the successful completion of this course, learners will be able to: 
\begin{enumerate}
	\item Design \& build usable, secure \& attractive applications with dynamic database functionality following an appropriate software development methodology.
\end{enumerate} 

\section*{Assessment Table}
\renewcommand{\arraystretch}{1.5}
\begin{tabular}{|l|l|l|l|l|}
	\hline      
	\vtop{\hbox{\strut \textbf{Assessment}}\hbox{\strut \textbf{Activity}}} & \textbf{Weighting} & \vtop{\hbox{\strut \textbf{Learning}}\hbox{\strut \textbf{Outcomes}}} & \vtop{\hbox{\strut \textbf{Assessment}}\hbox{\strut \textbf{Grading Scheme}}} & \vtop{\hbox{\strut \textbf{Completion}}\hbox{\strut \textbf{Requirements}}} \\
	                            
	\hline
	                                
	\small Activities                                          & \small 20\%        & \small 1 & \small CRA                                                                    & \small Cumulative                                                           \\ \hline  
	\small Project 1: Student Management System                                                              & \small 50\%        & \small 1                                                    & \small CRA                                                                    & \small Cumulative                                                           \\ \hline 
  \small Project 2: API                                                             & \small 30\%        & \small 1                                                      & \small CRA                                                                    & \small Cumulative                                                           \\ \hline 
\end{tabular}

\section*{Conditions of Assessment}
You will complete this assessment during your learner managed time, however, there will be availability during the teaching sessions to discuss the requirements \& your progress of this assessment. This assessment will need to be completed by \textbf{TBA}. 

\section*{Pass Criteria}
This assessment is criterion-referenced (CRA) with a cumulative pass mark of \textbf{50\%} over all assessments in \textbf{IN607: Introductory Application Development Concepts}.

\section*{Authenticity}
All parts of your submitted assessment must be completely your work \& any references must be cited appropriately including, externally-sourced graphic elements. Provide your references in a \textbf{README.md} file. All media must be royalty free (or legally purchased) for educational use. Failure to do this will result in a mark of \textbf{zero} for this assessment.

\section*{Policy on Submissions, Extensions, Resubmissions \& Resits}
The school's process concerning submissions, extensions, resubmissions \& resits complies with \textbf{Otago Polytechnic} policies. Learners can view policies on the \textbf{Otago Polytechnic} website located at \href{https://www.op.ac.nz/about-us/governance-and-management/policies}{https://www.op.ac.nz/about-us/governance-and-management/policies}.

\section*{Submissions}
You must submit all program files via \textbf{GitHub Classroom}. Here is the URL to the repository you will use for your submission – \href{https://classroom.github.com/a/P656imf2}{https://classroom.github.com/a/P656imf2}. Create a new branch called  \textbf{01-php-basics} from the \textbf{main} branch by running the command - \textbf{git checkout -b 01-php-basics}. This branch will be your development branch for this assessment. Once you have completed this assessment, create a pull request \& assign the \textbf{GitHub} user \textbf{grayson-orr} to a reviewer. \textbf{Do not} merge your own pull request. Late submissions will incur a \textbf{10\% penalty per day}, rolling over at \textbf{5:00 PM}.

\section*{Extensions}
Familiarise yourself with the assessment due date. If you need an extension, contact the course lecturer before the due date. If you require more than a week's extension, a medical certificate or support letter from your manager may be needed.

\section*{Resubmissions}
Learners may be requested to resubmit an assessment following a rework of part/s of the original assessment. Resubmissions are to be completed within a negotiable short time frame \& usually must be completed within the timing of the course to which the assessment relates. Resubmissions will be available to learners who have made a genuine attempt at the first assessment opportunity \& achieved a \textbf{D grade (40-49\%)}. The maximum grade awarded for resubmission will be \textbf{C-}.

\section*{Resits}
Resits \& reassessments are not applicable in \textbf{IN607: Introductory Application Development Concepts}.

\section*{Instructions - Learning Outcomes 1}

\end{document}