% Author: Grayson Orr
% Course: ID607001: Introductory Application Development Concepts

\documentclass{article}
\author{}

\usepackage{graphicx}
\usepackage{wrapfig}
\usepackage{enumerate}
\usepackage{hyperref}
\usepackage[margin = 2.25cm]{geometry}
\usepackage[table]{xcolor}
\usepackage{soul}
\usepackage{fancyhdr}
\hypersetup{
  colorlinks = true,
  urlcolor = blue
}
\setlength\parindent{0pt}
\pagestyle{fancy}
\fancyhf{}
\rhead{College of Engineering, Construction \& Living Sciences\\Bachelor of Information Technology}
\lfoot{Project 1: Node.js REST API\\Version 3, Semester One, 2023}
\rfoot{\thepage} 
 
\begin{document}

\begin{figure}
	\centering
	\includegraphics[width=50mm]{../img/logo.png}
\end{figure}

\title{College of Engineering, Construction \& Living Sciences\\Bachelor of Information Technology\\ID607001: Introductory Application Development Concepts\\Level 6, Credits 15\\\textbf{Project 1: Node.js REST API}}
\date{}
\maketitle

\section*{Assessment Overview}
In this \textbf{individual} assessment, you will develop a \textbf{REST API} using \textbf{Node.js}. You will choose the theme of your \textbf{REST API}. It could be on sport, culture, food or something else you are interested in. Your data will be stored in a \textbf{MySQL} database. The main purpose of this assessment is to demonstrate your ability to develop a \textbf{REST API} using taught concepts such as queries, relationships, validation, etc. In addition, marks will be allocated for code elegance, documentation \& \textbf{Git} usage.

\section*{Learning Outcome}
At the successful completion of this course, learners will be able to:
\begin{enumerate}
	\item Design \& build secure applications with dynamic database functionality following an appropriate software development methodology.
\end{enumerate}

\section*{Assessments}
\renewcommand{\arraystretch}{1.5}
\begin{tabular}{|c|c|c|c|}
	\hline
	\textbf{Assessment}                                 & \textbf{Weighting} & \textbf{Due Date}            & \textbf{Learning Outcomes} \\ \hline
	\small Practical: Node.js REST API Testing & \small 20\%        & \small 05-05-2023 (Friday at 4.59 PM)   & \small 1                   \\ \hline
	\small Project 1: Node.js REST API                  & \small 30\%        & \small \small 05-05-2023 (Friday at 4.59 PM) & \small 1                   \\ \hline
	\small Project 2: React CRUD                        & \small 50\%        & \small 16-06-2023 (Friday at 4.59 PM)  & \small 1                   \\ \hline
\end{tabular}

\section*{Conditions of Assessment}
You will complete this assessment during your learner-managed time. However, there will be time to discuss the requirements \& your assessment progress during the teaching sessions. This assessment will need to be completed by \textbf{Friday, 05 May 2022 at 4.59 PM}.

\section*{Pass Criteria}
This assessment is criterion-referenced (CRA) with a cumulative pass mark of \textbf{50\%} across all assessments in \textbf{ID607001: Introductory Application Development Concepts}.

\section*{Submission}
You must submit all program files via \textbf{GitHub Classroom}. Here is the URL to the repository you will use for your submission – \href{https://classroom.github.com/a/4w4EqOUZ}{https://classroom.github.com/a/4w4EqOUZ}.  Create a \textbf{.gitignore} and add the ignored files in this resource - \href{https://raw.githubusercontent.com/github/gitignore/main/Node.gitignore}{https://raw.githubusercontent.com/github/gitignore/main/Node.gitignore}. The latest program files in the \textbf{master} or \textbf{main} branch will be used to mark against the \textbf{Functionality} criterion. Please test your \textbf{master} or \textbf{main} branch application before you submit. Partial marks \textbf{will not} be given for incomplete functionality. Late submissions will incur a \textbf{10\% penalty per day}, rolling over at \textbf{5:00 PM}.

\section*{Authenticity}
All parts of your submitted assessment \textbf{must} be completely your work. Do your best to complete this assessment without \textbf{ChatGPT}. You need to demonstrate to the course lecturer that you can meet the learning outcome for this assessment. \\
 
 However, if you get stuck, you can use \textbf{ChatGPT} to help you get unstuck, permitting you acknowledge that you have used \textbf{ChatGPT}. In the assessment's repository \textbf{README.md} file, please include what prompt(s) you provided to \textbf{ChatGPT} \& how you used the response(s) to help you with your work. It also applies to code snippets retrieved from \textbf{StackOverflow} \& \textbf{GitHub}. Failure to do this will result in a mark of \textbf{zero} for this assessment.

\section*{Policy on Submissions, Extensions, Resubmissions \& Resits}
The school's process concerning submissions, extensions, resubmissions \& resits complies with \textbf{Otago Polytechnic | Te Pūkenga} policies. Learners can view policies on the \textbf{Otago Polytechnic | Te Pūkenga} website located at \href{https://www.op.ac.nz/about-us/governance-and-management/policies}{https://www.op.ac.nz/about-us/governance-and-management/policies}. 

\section*{Extensions}
Familiarise yourself with the assessment due date. If you need an extension, contact the course lecturer before the due date. If you require more than a week's extension, a medical certificate or support letter from your manager may be needed.

\section*{Resubmissions}
Learners may be requested to resubmit an assessment following a rework of part/s of the original assessment. Resubmissions are to be completed within a negotiable short time frame \& usually \textbf{must} be completed within the timing of the course to which the assessment relates. Resubmissions will be available to learners who have made a genuine attempt at the first assessment opportunity \& achieved a \textbf{D grade (40-49\%)}. The maximum grade awarded for resubmission will be \textbf{C-}.

\section*{Resits}
Resits \& reassessments are not applicable in \textbf{ID607001: Introductory Application Development Concepts}. 

\newpage

\section*{Instructions}
You will need to submit a \textbf{REST API} \& documentation that meet the following requirements: \\

\subsection*{Functionality - Learning Outcome 1 (40\%)}
\begin{itemize}
	\item \textbf{REST API} is developed using \textbf{Node.js}.
	\item \textbf{REST API} can run locally without modification.
	\item \textbf{Five} \textbf{models} containing at least \textbf{three fields} of data which you can interact with.
	\item A range of different data types, i.e., all \textbf{fields} of data can not be of a single type.
	\item \textbf{Three relationships} between \textbf{models}.
	\item Each \textbf{models} has a separate \textbf{controller} \& \textbf{route} file.
	\item A \textbf{controller} \& \textbf{route} file for \textbf{authentication}. The \textbf{controller} file must contain operations for register, login \& logout.
	\item A \textbf{controller} \& \textbf{route} file for each collection except \textbf{User}. Each \textbf{controller} file must contain operations for \textbf{CRUD} (Create, Read one, Read all, Update \& Delete).
	\item \hl{\textbf{Independent Research:}} The \textbf{index route}, i.e., \textbf{localhost:3000/api} must display all of the available \textbf{routes} in the application.
	\item Each \textbf{field} of data has custom validation when creating \& updating a \textbf{document}.
	\item \hl{\textbf{Independent Research:}} \textbf{REST API} version is set to \textbf{v1}. For example, an endpoint should look like \textbf{/api/v1/items}
	\item Return success, i.e., true or false \& data when performing \textbf{authentication} \& \textbf{CRUD} operations.
	\item \hl{\textbf{Independent Research:}} Return a success \& failure message when performing \textbf{authentication} \& \textbf{CRUD} operations, i.e., \textbf{"Successfully logged in"} or \textbf{"Successfully created an institution"}.
	\item \hl{\textbf{Independent Research:}} Filter \& sort \textbf{REST API data} using query parameters. A consumer should be able to filter all \textbf{fields} of data \& sort \textbf{fields} of data in ascending/descending order.
	\item \hl{\textbf{Independent Research:}} Return an appropriate message if a request does not return any \textbf{REST API data}, i.e., empty array.
	\item \hl{\textbf{Independent Research:}} Return an appropriate message if an endpoint does not exist.
	\item \hl{\textbf{Independent Research:}} Paginate the \textbf{REST API data} so that any number of records can be displayed per page. The default number is 10 records per page. 
	\item \textbf{GET}, \textbf{POST}, \textbf{PUT} \& \textbf{DELETE} routes are protected using \textbf{JSON Web Tokens (JWT)}.
	\item \textbf{REST API} rate limit is set to 50 requests per minute. You must display the following message if the user exceeds the 50 requests per minute - \textbf{"You have exceeded the number of requests per minute: 50. Please try again later."}
	\item Secure \textbf{HTTP} headers using \textbf{Helmet}. 
	\item \textbf{REST API} is deployed to \textbf{Heroku}. The \textbf{REST API} \textbf{should} be usable i.e., a consumer should be able to perform operations on your \textbf{REST API}.
	\item \textbf{REST API data} is stored in a \textbf{MongoDB Atlas} database.
\end{itemize}

\subsection*{Code Elegance - Learning Outcome 1 (40\%)}
\begin{itemize}
	\item Use of intermediate variables. No method calls as arguments.
	\item Idiomatic use of control flow, data structures \& in-built functions.
	\item Sufficient modularity.
	\item Functions \& variables are named appropriately.
	\item Efficient algorithmic approach, i.e., using the appropriate function(s) when querying your \textbf{collections}.
	\item \textbf{REST API} resource groups named with a plural noun instead of a noun or verb, i.e., \textbf{/api/v1/items} not \textbf{/api/v1/item}.
	\item File header comments using \textbf{JSDoc}. You \textbf{need} to explain the purpose of each \textbf{controller} \& \textbf{route} file.
	\item In-line comments using \textbf{JSDoc}. You \textbf{need} to explain complex logic that is not obvious.
	\item \hl{\textbf{Independent Research:}} Code files are formatted using \textbf{Prettier} \& a \textbf{.prettierrc} file. You \textbf{need} to declare a \textbf{npm} script in your application's \textbf{package.json} file which automates this process. Rules \textbf{should} include:
	      \begin{itemize}
		      \item Single quote is set to \textbf{true}.
		      \item Semi-colon is set to \textbf{false}.
		      \item Tab-width is set to \textbf{2}.
	      \end{itemize}
	\item \hl{\textbf{Independent Research:}} \textbf{Prettier} is installed as a development dependency.
	\item Declare a \textbf{npm} script in your application's \textbf{package.json} file which seeds the \textbf{collections}.
	\item No dead or unused code.
	\item Database configured for the development \& production environments.
	\item Application's environment variables are stored in a \textbf{.env} file.
	\begin{itemize}
		\item Create \textbf{example.env} file containing all of the application's environment variables' key. 
		\item Do not include the environment variables' value.  
	\end{itemize}  
\end{itemize}

\subsection*{Documentation \& Git Usage - Learning Outcome 1 (20\%)}
\begin{itemize}
	\item \hl{\textbf{Independent Research:}} Project board to help you organise \& prioritise your work. 
	\item \textbf{GitHub} repository contains a \textbf{Node.js .gitignore}.
	\item \textbf{REST API} is documented using \textbf{Postman}.
	\begin{itemize}
		\item You \textbf{should} provide an example for each route. However, you \textbf{should} provide \textbf{one} example of filtering, sorting \& paging. 
		\item Each example \textbf{should} contain a description, request \& response.
	\end{itemize}
	\item Provide the following in your repository \textbf{README.md} file:
	\begin{itemize}
		\item URL to the documented \textbf{REST API} on \textbf{Postman}.
		\item URL to the \textbf{REST API} on \textbf{Heroku}.
		\item How do you setup the development environment, i.e., after the repository is cloned, what do you need to do before you run the \textbf{REST API}?
		\item How do you deploy the \textbf{REST API} to \textbf{Heroku}?
		\item How do you format the code using \textbf{Prettier}?
	\end{itemize}
	\item Use of \textbf{Markdown}, i.e., bold text, code blocks, etc.
	\item Correct spelling \& grammar. 
	\item Your \textbf{Git commit messages} should:
	\begin{itemize}
		\item Reflect the context of each functional requirement change.
		\item Be formatted using the naming conventions outlined in the following:
			\begin{itemize}
				\item \textbf{Resource:} \small\href{https://dev.to/i5han3/git-commit-message-convention-that-you-can-follow-1709}{https://dev.to/i5han3/git-commit-message-convention-that-you-can-follow-1709}
			\end{itemize}
	\end{itemize}
\end{itemize}

\subsection*{Additional Information}
\begin{itemize}
    \item Attempt to commit at least \textbf{10} times per week.
    \item \textbf{Do not} rewrite your \textbf{Git} history. It is important that the course lecturer can see how you worked on your assessment over time. 
\end{itemize} 
\end{document}