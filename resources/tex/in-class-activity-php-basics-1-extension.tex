% Author: Grayson Orr
% Course: IN607: Introductory Application Development Concept

\documentclass{article}
\author{}

\usepackage{graphicx}
\usepackage{wrapfig}
\usepackage{enumerate}
\usepackage{hyperref}
\usepackage[margin = 2.25cm]{geometry}
\usepackage[table]{xcolor}
\usepackage{fancyhdr}
\hypersetup{
  colorlinks = true, 
  urlcolor = blue
}
\setlength\parindent{0pt}
\pagestyle{fancy}
\fancyhf{}
\rhead{College of Engineering, Construction \& Living Sciences\\Bachelor of Information Technology}
\lfoot{In-Class Activity: PHP Basics 1 Extension\\Version 1, Semester Two, 2021}
\rfoot{\thepage}

\begin{document}

\begin{figure}
    \centering
    \includegraphics[width=50mm]{../img/logo.png}
\end{figure}

\title{College of Engineering, Construction \& Living Sciences\\Bachelor of Information Technology\\IN607: Introductory Application Development Concepts\\Level 6, Credits 15\\\textbf{In-Class Activity: PHP Basics 1 Extension}}
\date{}
\maketitle
 
\section*{Instructions}
The purpose of this in-class activity to extend your knowledge. These problems are difficult \& will require you to understand \& use commonly used in-built \textbf{PHP} functions.

\section*{Code Review}
You must submit all program files via \textbf{GitHub Classroom}. Here is the URL to the repository you will use for your code review – \href{https://classroom.github.com/a/P656imf2}{https://classroom.github.com/a/P656imf2}. Checkout from the \textbf{main} branch to the \textbf{01-in-class-activity-ext} branch by running the command - \textbf{git checkout 01-in-class-activity-ext}. This branch will be your development branch for this activity. Once you have completed this activity, create a pull request \& assign the \textbf{GitHub} user \textbf{grayson-orr} to a reviewer. \textbf{Do not} merge your own pull request.

\subsection*{Problem 1:} 
Write a function called \textbf{find\_breed} which accepts an unsorted array of \textbf{strings} called \textbf{breeds}. Your code needs to search \textbf{breeds} for the name "Afghan Hound" \& return the location in the array using the \textbf{array\_search} function. If "Afghan Hound" is not in \textbf{breeds}, return -1.

\begin{verbatim}
  <?php
    // Write your solution here

    // Expected output:
    // find_breed(["Beagle", "Dalmatian", "Afghan Hound"]); => 2
    // find_breed(["Dalmatian", "Beagle", "Golden Retriever"]); => -1
  ?>
\end{verbatim}

\subsection*{Problem 2:} 
Write a function called \textbf{remove\_vowels} which accepts a \textbf{string} \textbf{word} \& returns a new \textbf{string} with all vowels removed using the \textbf{preg\_replace} function.

\begin{verbatim}
  <?php
    // Write your solution here

    // Expected output:
    // remove_vowels("Hello, World!"); => Hll, Wrld!
  ?>
\end{verbatim}

\subsection*{Problem 3:} 
Write a function called \textbf{missing\_num} which accepts an unsorted array of \textbf{integers} called \textbf{nums} \& return the missing number using the \textbf{array\_sum} function.

\begin{verbatim}
  <?php
    // Write your solution here

    // Expected output:
    // missing_num([10, 3, 4, 8, 1, 7, 6, 9, 2]); => 5
  ?>
\end{verbatim}

\subsection*{Problem 4:}
Write a function called \textbf{file\_extensions} which accepts an array of \textbf{strings} called \textbf{files} \& returns their extension names using the \textbf{explode} function.

\begin{verbatim}
  <?php
    // Write your solution here

    // Expected output:
    // file_extensions(["index.html", "index.js"]); => ["html", "js"]
  ?>
\end{verbatim}

\end{document}
