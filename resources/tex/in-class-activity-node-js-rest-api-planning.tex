% Author: Grayson Orr
% Course: ID607001: Introductory Application Development Concept

\documentclass{article}
\author{}

\usepackage{graphicx}
\usepackage{wrapfig}
\usepackage{enumerate}
\usepackage{hyperref}
\usepackage[margin = 2.25cm]{geometry}
\usepackage[table]{xcolor}
\usepackage{fancyhdr}
\hypersetup{
  colorlinks = true, 
  urlcolor = blue
}
\setlength\parindent{0pt}
\pagestyle{fancy}
\fancyhf{}
\rhead{College of Engineering, Construction \& Living Sciences\\Bachelor of Information Technology}
\lfoot{In-Class Activity: Project 1: Node.js REST API Preparation \\Version 1, Semester Two, 2021}
\rfoot{\thepage}

\begin{document}

\begin{figure}
  \centering
  \includegraphics[width=50mm]{../img/logo.png}
\end{figure}

\title{College of Engineering, Construction \& Living Sciences\\Bachelor of Information Technology\\ID607001: Introductory Application Development Concepts\\Level 6, Credits 15\\\textbf{In-Class Activity: Project 1: Node.js REST API Planning}}
\date{}
\maketitle

\section*{Instructions}
The main purpose of this in-class activity is to plan your \textbf{Project 1: Node.js REST API}. In addition, you will explore how to automatically restart your simple \textbf{API's} server using \textbf{Nodemon} and get an institution by its id.

\section*{Code Review}
You must submit all program files via \textbf{GitHub Classroom}. Here is the URL to the repository you will use for your code review – \href{https://classroom.github.com/a/P656imf2}{https://classroom.github.com/a/P656imf2}. Checkout from the \textbf{main} branch to the \textbf{03-in-class-activity} branch by running the command - \textbf{git checkout 03-in-class-activity}. This branch will be your development branch for this activity. Once you have completed this activity, create a pull request \& assign the \textbf{GitHub} user \textbf{grayson-orr} to a reviewer. \textbf{Do not} merge your pull request.

\section*{Getting Started}
Open your repository in \textbf{Visual Studio Code}. Create a simple \textbf{API} as described in the \href{https://github.com/otago-polytechnic-bit-courses/ID607001-intro-app-dev-concepts/blob/master/lecture-notes/03-node-js-rest-api-1.md}{lecture notes}.

\subsection*{Nodemon}
\textbf{Nodemon} is a tool that helps you develop \textbf{Node.js} applications by automatically restarting the application when a file change is detected. It does not require additional changes to the application's code to get started. To use \textbf{Nodemon}, install it as a development dependency. In \textbf{package.json}, replace the \textbf{start} script value \textbf{node app.js} with \textbf{nodemon app.js}.

\subsection*{Get an institution by its id}
In the \href{https://github.com/otago-polytechnic-bit-courses/ID607001-intro-app-dev-concepts/blob/master/lecture-notes/03-node-js-rest-api-1.md}{lecture notes}, you looked at how to create, read, update and delete an institution. Extend your \textbf{API's} functionality by creating a route and controller function that enables a user to get an institution by its id. \textbf{Hint:} The approach is similar to updating and deleting an institution.

\subsection*{Project 1: Node.js REST API planning}
You will be starting your \textbf{Project 1: Node.js REST API} assessment next week. Before you start, you need to decide your \textbf{API's} theme and the data you are going to display to the user. You need at least \textbf{five} collections (\textbf{user collection} is included) with at least \textbf{three fields} of data. Also, show the relationships between \textbf{collections}. \textbf{Note:} You need at least \textbf{two relationships} between \textbf{collections}. You can display this anyway you wish, i.e., UML, text, etc. As long as it is clear to the \textbf{course lecturer} when reviewing.

\end{document}