% Author: Grayson Orr
% Course: IN607: Introductory Application Development Concept

\documentclass{article}
\author{}

\usepackage{graphicx}
\usepackage{wrapfig}
\usepackage{enumerate}
\usepackage{hyperref}
\usepackage[margin = 2.25cm]{geometry}
\usepackage[table]{xcolor}
\usepackage{fancyhdr}
\hypersetup{
  colorlinks = true, 
  urlcolor = blue
}
\setlength\parindent{0pt}
\pagestyle{fancy}
\fancyhf{}
\rhead{College of Engineering, Construction \& Living Sciences\\Bachelor of Information Technology}
\lfoot{In-Class Activity: ES6 Basics 1 Extension\\Version 1, Semester One, 2022}
\rfoot{\thepage}

\begin{document}

\begin{figure}
    \centering
    \includegraphics[width=50mm]{../img/logo.png}
\end{figure}

\title{College of Engineering, Construction \& Living Sciences\\Bachelor of Information Technology\\IN607: Introductory Application Development Concepts\\Level 6, Credits 15\\\textbf{In-Class Activity: ES6 Basics 1 Extension}}
\date{}
\maketitle
 
\section*{Instructions}
The purpose of this in-class activity to extend your knowledge. These problems are difficult \& will require you to understand \& use commonly used in-built \textbf{PHP} functions.

\section*{Code Review}
You must submit all program files via \textbf{GitHub Classroom}. Here is the URL to the repository you will use for your code review – \href{https://classroom.github.com/a/_6KSahyX}{https://classroom.github.com/a/_6KSahyX}. Checkout from the \textbf{main} branch to the \textbf{01-in-class-activity-ext} branch by running the command - \textbf{git checkout 01-in-class-activity-ext}. This branch will be your development branch for this activity. Once you have completed this activity, create a pull request \& assign the \textbf{GitHub} user \textbf{grayson-orr} to a reviewer. \textbf{Do not} merge your own pull request.

\subsection*{Problem 1:} 
Write an \textbf{arrow function} called \textbf{findBreed} which accepts an unsorted array of \textbf{strings} called \textbf{breeds}. Your code needs to search \textbf{breeds} for "Afghan Hound" \& return its location in the array, i.e., index. If "Afghan Hound" is not in \textbf{breeds}, return -1.

\begin{verbatim}
// Write your solution here

const breeds = ["Afghan Hound", /** Add your breeds here */]
console.log(findBreed(breeds)) 

// Expected output:
// 1

// Make sure to test if Afghan Hound is not found
\end{verbatim}

\subsection*{Problem 2:} 
Write an \textbf{arrow function} called \textbf{removeVowels} which accepts a \textbf{string} called \textbf{word} \& returns a new \textbf{string} with all vowels removed.

\begin{verbatim}
// Write your solution here

const word = // Add your word here

console.log(removeVowels(word))
\end{verbatim}

\subsection*{Problem 3:} 
Write an \textbf{arrow function} function called \textbf{missingNum} which accepts an unsorted array of \textbf{integers} called \textbf{nums} \& return the missing number.

\begin{verbatim}
// Write your solution here

const nums = [10, 3, 4, 8, 1, 7, 6, 9, 5]
console.log(missingNum(nums))

// Expected output:
// 2
\end{verbatim}

\subsection*{Problem 4:}
Write an \textbf{arrow function} called \textbf{fileExtensions} which accepts an array of \textbf{strings} called \textbf{files} \& returns their extension names.

\begin{verbatim}
// Write your solution here

const files = ["index.html", "main.js", "sample.txt", "data.json"]
console.log(fileExtensions(files))

// Expected output:
// html
// js
// txt
// json
\end{verbatim}

\end{document}
