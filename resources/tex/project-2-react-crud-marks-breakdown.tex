% Author: Grayson Orr
% Course: ID607001: Introductory Application Development Concepts

\documentclass{article}
\author{}

\usepackage{graphicx}
\usepackage{wrapfig}
\usepackage{enumerate}
\usepackage{hyperref}
\usepackage[margin = 2.25cm]{geometry}
\usepackage[table]{xcolor}
\usepackage{fancyhdr}
\hypersetup{
  colorlinks = true,
  urlcolor = blue
}
\setlength\parindent{0pt}
\pagestyle{fancy}
\fancyhf{}
\rhead{College of Engineering, Construction \& Living Sciences\\Bachelor of Information Technology}
\lfoot{Project 2: React CRUD Marks Breakdown\\Version 3, Semester One, 2022}
\rfoot{\thepage}
 
\begin{document}

\begin{figure}
  \centering
  \includegraphics[width=50mm]{../img/logo.png}
\end{figure}

\title{College of Engineering, Construction \& Living Sciences\\Bachelor of Information Technology\\ID607001: Introductory Application Development Concepts\\Level 6, Credits 15\\\textbf{Project 2: React CRUD Marks Breakdown}}
\date{}
\maketitle

\subsection*{Functionality - Learning Outcome 1 (40\%)}
\begin{itemize}
  \item Authentication
        \begin{itemize}
          \item \textbf{Independent Research:} Register a new user via a form.
          \item Login an existing user via a form.
          \item Log out of the application.
        \end{itemize}
  \item CRUD
        \begin{itemize}
          \item Request \textbf{REST API data} from at least three \textbf{API} resource groups using \textbf{Axios}.
          \item Create new \textbf{REST API data} via a form. You can display the form on the page or in a modal.
          \item View \textbf{REST API data} in a table.
          \item \textbf{Independent Research:} View \textbf{REST API data} in a table using an id. For example, \textbf{/institutions/1} would return \textbf{REST API data} for that specific \textbf{Institutions} object.
          \item \textbf{Independent Research:} Update \textbf{REST API data} via a form. Similar to creating \textbf{REST API data}, you can display the form on the page or in a modal.
          \item \textbf{Independent Research:} Delete \textbf{REST API data}. Prompt the user for deletion. You \textbf{can} use the in-built \textbf{confirm() JavaScript} function.
          \item \textbf{Independent Research:} Incorrectly formatted form field values handled gracefully using validation error messages, i.e., \textbf{first name} form field is required.
        \end{itemize}
  \item \textbf{Independent Research:} Paginate \textbf{REST API data} across several pages with \textbf{next} \& \textbf{previous} buttons or links. You can choose the number of \textbf{REST API data} per page.
  \item \textbf{Independent Research:} Search \textbf{REST API data} via a search bar.
  \item User-interface is visually attractive with a coherent graphical theme \& style using \textbf{Reactstrap}.
  \item Application deployed to \textbf{Heroku}.
  \item End-to-end \textbf{Cypress} tests that ensures the register, login \& logout functionality is working as expected. You \textbf{need} to declare a \textbf{npm} script in your application's \textbf{package.json} file that automates this process.
\end{itemize}

\subsection*{Code Elegance - Learning Outcome 1 (45\%)}
\begin{itemize}
  \item Use of intermediate variables. No method calls as arguments.
  \item Idiomatic use of control flow, data structures \& in-built functions.
  \item Sufficient modularity, i.e., UI split into independent reusable pieces.
  \item Functions \& variables are named appropriately.
  \item Components written as functional, not class.
  \item Adheres to a client-server architecture, i.e., the frontend is separate from the backend.
  \item File header comments using \textbf{JSDoc}. You \textbf{need} to explain the purpose of each \textbf{component} file.
  \item In-line comments using \textbf{JSDoc}. You \textbf{need} to explain complex logic.
  \item Code files are formatted using \textbf{Prettier} \& a \textbf{.prettierrc} file. You \textbf{need} to declare a \textbf{npm} script in your application's \textbf{package.json} file which automates this process. Rules \textbf{should} include:
        \begin{itemize}
          \item Single quote is set to \textbf{true}.
          \item Semi-colon is set to \textbf{false}.
          \item Tab-width is set to \textbf{2}.
        \end{itemize}
  \item \textbf{Prettier} \& \textbf{Cypress} are installed as development dependencies.
  \item No dead or unused code.
\end{itemize}

\subsection*{Documentation \& Git Usage - Learning Outcome 1 (15\%)}
\begin{itemize}
  \item Project board to help you organise \& prioritise your work.
  \item Provide the following in your repository \textbf{README.md} file:
        \begin{itemize}
          \item URL to the application on \textbf{Heroku}.
          \item How do you setup the environment for development, i.e., after the repository is cloned, what do you need to run the application locally?
          \item How do you run the end-to-end \textbf{Cypress} tests?
          \item How do you deploy the \textbf{React} application to \textbf{Heroku}?
        \end{itemize}
  \item Use of \textbf{Markdown}, i.e., bold text, code blocks, etc.
  \item Correct spelling \& grammar.
  \item Your \textbf{Git commit messages} should:
        \begin{itemize}
          \item Reflect the context of each functional requirement change.
          \item Be formatted using the naming conventions outlined in the following:
                \begin{itemize}
                  \item \textbf{Resource:} \small\href{https://dev.to/i5han3/git-commit-message-convention-that-you-can-follow-1709}{https://dev.to/i5han3/git-commit-message-convention-that-you-can-follow-1709}
                \end{itemize}
        \end{itemize}
\end{itemize}
\end{document}