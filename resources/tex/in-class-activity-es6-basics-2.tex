% Author: Grayson Orr
% Course: ID607001: Introductory Application Development Concept

\documentclass{article}
\author{}

\usepackage{graphicx}
\usepackage{wrapfig}
\usepackage{enumerate}
\usepackage{hyperref}
\usepackage[margin = 2.25cm]{geometry}
\usepackage[table]{xcolor}
\usepackage{fancyhdr}
\hypersetup{
  colorlinks = true, 
  urlcolor = blue
}
\setlength\parindent{0pt}
\pagestyle{fancy}
\fancyhf{}
\rhead{College of Engineering, Construction \& Living Sciences\\Bachelor of Information Technology}
\lfoot{In-Class Activity: ES6 Basics 2\\Version 1, Semester Two, 2021}
\rfoot{\thepage}

\begin{document}

\begin{figure}
  \centering
  \includegraphics[width=50mm]{../img/logo.png}
\end{figure}

\title{College of Engineering, Construction \& Living Sciences\\Bachelor of Information Technology\\ID607001: Introductory Application Development Concepts\\Level 6, Credits 15\\\textbf{In-Class Activity: ES6 Basics 2}}
\date{}
\maketitle

\section*{Instructions}
The purpose of this in-class activity is to familiarise yourself with more complex constructs such as \textbf{map}, \textbf{filter} \& \textbf{reduce}. Also, you will look at how to read \& process data from a local file. The following eight questions will require a little more thought than the previous in-class activity.

\section*{Submission}
You must submit all program files via \textbf{GitHub Classroom}. Here is the URL to the repository you will use for your code review – \href{https://classroom.github.com/a/\_6KSahyX}{https://classroom.github.com/a/\_6KSahyX}. If you wish to have your code reviewed, message the course lecturer on \textbf{Microsoft Teams}.

\section*{Getting Started}
Open your repository in \textbf{Visual Studio Code}. Create a new file called \textbf{02-in-class-activity.js}. In \textbf{02-in-class-activity.js}, add the following: 

\begin{verbatim}
  console.log('Hello, World!')
\end{verbatim}

Open a \textbf{terminal} \& run the following command:

\begin{verbatim}
  node 02-in-class-activity.js
\end{verbatim}

If the output is \textbf{Hello, World!}, then you are ready to start coding.

\subsection*{Problem 1:}
For each element in \textbf{nums}, calculate its power of two using \& return as an \textbf{array} using the \textbf{map} function.

\begin{verbatim}
  const nums = [2, 4, 6, 8, 10]

  const powOfTwo = // Write your solution here
  console.log(powOfTwo)

  // Expected output:
  // [4, 16, 36, 64, 100]
\end{verbatim}

\subsection*{Problem 2:}
For each element in \textbf{temps}, convert its value from fahrenheit to celsius and return as an \textbf{array} using the \textbf{map} function. Round each value to the nearest two decimal places using the \textbf{Math.round} function.

\begin{verbatim}
  const temps = [65, 45, 25, 5]

  const fahToCel = // Write your solution here
  console.log(fahToCel)

  // Expected output:
  // [18.33, 7.22, -3.89, -15.0]
\end{verbatim}

\subsection*{Problem 3:}
Using the \textbf{filter} function, return countries that have a population of less than 1000000000 (one billion).

\begin{verbatim}
  const countries = [
      { name: 'Brazil', population: 213445417 },
      { name: 'China', population: 1339330514 },
      { name: 'India', population: 1352642280 },
      { name: 'Russia', population: 142320790 },
      { name: 'United States of America', population: 332475723 }
  ]

  const countriesWithPopLessThanOneBil = // Write your solution here
  console.log(countriesWithPopLessThanOneBil)

  // Expected output:
  // [
  //     { name: 'Brazil', population: 213445417 }, 
  //     { name: 'Russia', population: 142320790 }, 
  //     { name: 'United States of America', population: 332475723 }
  // ]
\end{verbatim}

\subsection*{Problem 4:}
Using the \textbf{filter} function, return animals that are native to New Zealand.

\begin{verbatim}
  const animals = [
      { name: "Cassowary", native_country: "Australia" },
      { name: "Kiwi", native_country: "New Zealand" },
      { name: "Little Blue Penguin", native_country: "New Zealand" },
      { name: "Bald Eagle", native_country: "United States of America" }
  ]

  const nativeAnimals = // Write your solution here
  console.log(nativeAnimals)

  // Expected output:
  // [
  //     { name: 'Kiwi', native_country: 'New Zealand' },
  //     { name: 'Little Blue Penguin', native_country: 'New Zealand' }
  // ]
\end{verbatim}

\subsection*{Problem 5:}
Using the \textbf{reduce} function, return the total \textbf{price} for the given \textbf{groceries} \textbf{array} of \textbf{objects}.

\begin{verbatim}
  const groceries = [
      { name: 'Chicken', price: 10 },
      { name: 'Butter', price: 5 },
      { name: 'Lettuce', price: 2 },
      { name: 'Steak', price: 20 },
  ]

  const groceriesTotal = // Write your solution here
  console.log(groceriesTotal)

  // Expected output:
  // 37
\end{verbatim}

\subsection*{Problem 6:}
Using the \textbf{reduce} function, return an \textbf{object} where the \textbf{key} is the name of the ice cream flavour, i.e., chocolate \& the \textbf{value} is an \textbf{integer} that represents the total count for that flavour, i.e., 3.

\begin{verbatim}
  const iceCreamFlavours = [
      'vanilla', 'chocolate', 'strawberry', 
      'vanilla', 'mango', 'vanilla', 
      'chocolate', 'strawberry', 'mango', 
      'orange', 'chocolate'
  ]

  const iceCreamFlavourCount = // Write your solution here
  console.log(iceCreamFlavourCount)

  // Expected output:
  // { vanilla: 3, chocolate: 3, strawberry: 2, mongo: 2, orange: 1 }
\end{verbatim}

\subsection*{Problem 7:}
Using the \textbf{readFile} function, read \textbf{nursery-rhyme.txt} located in the \textbf{in-class activities} directory. For each word in \textbf{nursery-rhyme.txt}, convert it to \textbf{lowercase} using the \textbf{map} function.

\begin{verbatim}
  // Expected output:
  // [
  //     'old',       'macdonald',  'had',
  //     'a',         'farm,',      'e-i-e-i-o!',
  //     'and',       'on',         'his',
  //     'farm',      'he',         'had',
  //     'a',         'cow,',       'e-i-e-i-o!',
  //     'with',      'a',          'moo-moo',
  //     'here',      'and',        'a',
  //     'moo-moo',   'there,',     'here',
  //     'a',         'moo,',       'there',
  //     'a',         'moo,',       'everywhere',
  //     'a',         'moo-moo,',   'old',
  //     'macdonald', 'had',        'a',
  //     'farm,',     'e-i-e-i-o!'
  // ]
\end{verbatim}

\subsection*{Problem 8:}
Using the \textbf{readFile} function, read \textbf{users.json} located in the \textbf{in-class activities} directory. Using the \textbf{filter} function, return \textbf{users} who are \textbf{Senior Lecturers}.

\begin{verbatim}
  // [
  //     {
  //         first_name: 'Faisal',
  //         last_name: 'Hassan',
  //         position: 'Senior Lecturer'
  //     },
  //     {
  //         first_name: 'Joy',
  //         last_name: 'Gasson',
  //         position: 'Senior Lecturer'
  //     }
  // [
\end{verbatim}

\end{document}
