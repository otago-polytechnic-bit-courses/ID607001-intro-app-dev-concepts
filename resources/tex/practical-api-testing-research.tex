% Author: Grayson Orr
% Course: IN607: Introductory Application Development Concepts

\documentclass{article}
\author{}

\usepackage{graphicx}
\usepackage{wrapfig}
\usepackage{enumerate}
\usepackage{hyperref}
\usepackage[margin = 2.25cm]{geometry}
\usepackage[table]{xcolor}
\usepackage{fancyhdr}
\hypersetup{
  colorlinks = true,
  urlcolor = blue
}
\setlength\parindent{0pt}
\pagestyle{fancy}
\fancyhf{}
\rhead{College of Engineering, Construction and Living Sciences\\Bachelor of Information Technology}
\lfoot{Practical: API Testing Research\\Version 1, Semester Two, 2021}
\rfoot{\thepage}
 
\begin{document}

\begin{figure}
	\centering
	\includegraphics[width=50mm]{../img/logo.png}
\end{figure}

\title{College of Engineering, Construction and Living Sciences\\Bachelor of Information Technology\\IN607: Introductory Application Development Concepts\\Level 6, Credits 15\\\textbf{Practical: API Testing Research}}
\date{}
\maketitle

\section*{Assessment Overview}
In this assessment, you will be given a \textbf{Laravel API} application to test. You will be required to independently research \& write at least 40 tests using \textbf{PHPUnit} which verify the correctness of the given application. This includes \textbf{CRUD} functionality, query parameters, status codes \& shape of response data. You will also check the code coverage of your tests using \textbf{php-code-coverage}. In addition, marks will be allocated for code elegance, documentation \& \textbf{Git} usage. 

\section*{Learning Outcomes}
At the successful completion of this course, learners will be able to:
\begin{enumerate}
	\item Design \& build usable, secure \& attractive applications with dynamic database functionality following an appropriate software development methodology.
\end{enumerate}

\section*{Assessment Table}
\renewcommand{\arraystretch}{1.5}
\begin{tabular}{|l|l|l|l|l|}
	\hline
	\vtop{\hbox{\strut \textbf{Assessment}}\hbox{\strut \textbf{Activity}}} & \textbf{Weighting} & \vtop{\hbox{\strut \textbf{Learning}}\hbox{\strut \textbf{Outcomes}}} & \vtop{\hbox{\strut \textbf{Assessment}}\hbox{\strut \textbf{Grading Scheme}}} & \vtop{\hbox{\strut \textbf{Completion}}\hbox{\strut \textbf{Requirements}}} \\

	\hline

	\small Practical: API Testing Research                                                      & \small 20\%        & \small 1                                                           & \small CRA                                                                    & \small Cumulative                                                           \\ \hline
	\small Project 1: Laravel API                                                        & \small 30\%        & \small 1                                                        & \small CRA                                                                    & \small Cumulative                                                           \\ \hline
	\small Project 2: React CRUD                                                        & \small 50\%        & \small 1                                                        & \small CRA                                                                    & \small Cumulative                                                           \\ \hline
\end{tabular}

\section*{Conditions of Assessment}
You will complete this assessment during your learner managed time, however, there will be availability during the teaching sessions to discuss the requirements \& your progress of this assessment. This assessment will need to be completed by \textbf{Wednesday, 20 October 2021 at 5:00 PM}.

\section*{Pass Criteria}
This assessment is criterion-referenced (CRA) with a cumulative pass mark of \textbf{50\%} over all assessments in \textbf{IN607: Introductory Application Development Concepts}.

\section*{Authenticity}
All parts of your submitted assessment must be completely your work \& any references must be cited appropriately. Provide your references in a \textbf{README.md} file. Failure to do this will result in a mark of \textbf{zero} for this assessment.

\section*{Policy on Submissions, Extensions, Resubmissions \& Resits}
The school's process concerning submissions, extensions, resubmissions \& resits complies with \textbf{Otago Polytechnic} policies. Learners can view policies on the \textbf{Otago Polytechnic} website located at \href{https://www.op.ac.nz/about-us/governance-and-management/policies}{https://www.op.ac.nz/about-us/governance-and-management/policies}.

\section*{Submissions}
You must submit all program files via \textbf{GitHub Classroom}. Here is the URL to the repository you will use for your submission – \href{https://classroom.github.com/a/0kYlKqW8}{https://classroom.github.com/a/0kYlKqW8}. The latest program files in the \textbf{main} branch will be used to run your application. Late submissions will incur a \textbf{10\% penalty per day}, rolling over at \textbf{5:00 PM}.

\section*{Extensions}
Familiarise yourself with the assessment due date. If you need an extension, contact the course lecturer before the due date. If you require more than a week's extension, a medical certificate or support letter from your manager may be needed.

\section*{Resubmissions}
Learners may be requested to resubmit an assessment following a rework of part/s of the original assessment. Resubmissions are to be completed within a negotiable short time frame \& usually must be completed within the timing of the course to which the assessment relates. Resubmissions will be available to learners who have made a genuine attempt at the first assessment opportunity \& achieved a \textbf{D grade (40-49\%)}. The maximum grade awarded for resubmission will be \textbf{C-}.

\section*{Resits}
Resits \& reassessments are not applicable in \textbf{IN607: Introductory Application Development Concepts}. 

\newpage

\section*{Instructions}
You will need to submit an application \& documentation that meet the following requirements:

\subsection*{Functionality - Learning Outcomes 1 (60\%)}
\begin{itemize}
	\item At least 40 tests verifying the correctness of the following:
	\begin{itemize}
    \item CRUD (create, read, update \& delete) functionality.
    \item Query parameters, i.e., filtering \& sorting data.
    \item Status codes, i.e., checking if a response returns 200, 404, etc.
    \item Shape of the data, i.e., does the response data contain a specific column?
  \end{itemize}
  \item Code covered using \textbf{php-code-coverage}.
\end{itemize}

\subsection*{Code Elegance - Learning Outcomes 1 (30\%)}
\begin{itemize}
	\item Use of intermediate variables. No method calls as arguments.
	\item Idiomatic use of control flow, data structures \& in-built functions.
	\item Adheres to an \textbf{OO} architecture, i.e., classes, methods \& variables are named appropriately.
	\item Code files are formatted.
	\item No dead or unused code.
	\item Databases configured for testing environment.
\end{itemize}

\subsection*{Documentation \& Git Usage - Learning Outcomes 1 (10\%)}
\begin{itemize}
	\item Provide the following in your repository \textbf{README.md} file:
	      \begin{itemize}
		      \item How do you setup the environment for development, i.e., after the repository is cloned, what do you need to run the the tests locally?
					\item How do you run the tests?
	      \end{itemize}
			\end{itemize}
			\begin{itemize}
	\item Commit messages must reflect the context of each functional requirement change. \textbf{Do not} rewrite your \textbf{Git} history. It is important that the course lecturer can see how you worked on your assessment over time.
\end{itemize}
\end{document}