% Author: Grayson Orr
% Course: ID607001: Introductory Application Development Concept

\documentclass{article}
\author{}

\usepackage{graphicx}
\usepackage{wrapfig}
\usepackage{enumerate}
\usepackage{hyperref}
\usepackage[margin = 2.25cm]{geometry}
\usepackage[table]{xcolor}
\usepackage{fancyhdr}
\hypersetup{
  colorlinks = true, 
  urlcolor = blue
}
\setlength\parindent{0pt}
\pagestyle{fancy}
\fancyhf{}
\rhead{College of Engineering, Construction \& Living Sciences\\Bachelor of Information Technology}
\lfoot{In-Class Activity: ES6 Basics 2\\Version 1, Semester Two, 2021}
\rfoot{\thepage}

\begin{document}

\begin{figure}
    \centering
    \includegraphics[width=50mm]{../img/logo.png}
\end{figure}

\title{College of Engineering, Construction \& Living Sciences\\Bachelor of Information Technology\\ID607001: Introductory Application Development Concepts\\Level 6, Credits 15\\\textbf{In-Class Activity: ES6 Basics 2}}
\date{}
\maketitle
 
\section*{Instructions}
The purpose of this in-class activity is to 

\section*{Code Review}
You must submit all program files via \textbf{GitHub Classroom}. Here is the URL to the repository you will use for your code review – \href{https://classroom.github.com/a/P656imf2}{https://classroom.github.com/a/P656imf2}. Checkout from the \textbf{main} branch to the \textbf{03-in-class-activity} branch by running the command - \textbf{git checkout 03-in-class-activity}. This branch will be your development branch for this activity. Once you have completed this activity, create a pull request \& assign the \textbf{GitHub} user \textbf{grayson-orr} to a reviewer. \textbf{Do not} merge your pull request.

\section*{Getting Started}
Open your repository in \textbf{Visual Studio Code}. Create a simple \textbf{API} as described in the \href{lecture notes}{https://github.com/otago-polytechnic-bit-courses/ID607001-intro-app-dev-concepts/blob/master/lecture-notes/03-node-js-rest-api-1.md}.

\subsection*{Problem 1:} 

\subsection*{Problem 2:} 

\end{document}