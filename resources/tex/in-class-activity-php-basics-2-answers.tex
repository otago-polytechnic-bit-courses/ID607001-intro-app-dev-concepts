% Author: Grayson Orr
% Course: IN607: Introductory Application Development Concept

\documentclass{article}
\author{}

\usepackage{graphicx}
\usepackage{wrapfig}
\usepackage{enumerate}
\usepackage{hyperref}
\usepackage[margin = 2.25cm]{geometry}
\usepackage[table]{xcolor}
\usepackage{fancyhdr}
\hypersetup{
  colorlinks = true, 
  urlcolor = blue
}
\setlength\parindent{0pt}
\pagestyle{fancy}
\fancyhf{}
\rhead{College of Engineering, Construction \& Living Sciences\\Bachelor of Information Technology}
\lfoot{In-Class Activity: PHP Basics 2 Answers\\Version 1, Semester Two, 2021}
\rfoot{\thepage}

\begin{document}

\begin{figure}
    \centering
    \includegraphics[width=50mm]{../img/logo.png}
\end{figure}

\title{College of Engineering, Construction \& Living Sciences\\Bachelor of Information Technology\\IN607: Introductory Application Development Concepts\\Level 6, Credits 15\\\textbf{In-Class Activity: PHP Basics 2 Answers}}
\date{}
\maketitle

\subsection*{Problem 1:} 
\begin{verbatim}
  <?php
      class Cat {
          private $name;
          private $age;
          private $breed;

          public function __construct($name, $age, $breed) {
              $this->name = $name;
              $this->age = $age;
              $this->breed = $breed;
          }
                  
          public function set_name($name) {
              $this->name = $name;
          }
        
          public function get_name() {
              return $this->name;
          }
        
          public function set_age($age) {
              $this->age = $age;
          }
        
          public function get_age() {
            return $this->age;
          }

          public function set_breed($breed) {
              $this->breed = $breed;
          }
        
          public function get_breed() {
            return $this->breed;
          }

          public function __toString() {
            return "My " . $this->breed . " 's name is " 
                . $this->name. ". S/he is " . $this->age . " year(s) old.";        
          }
      }
      
      $cat_one = new Cat("Ralph", 12, "Abyssinian");
      $cat_two = new Cat("Leo", 9, "Birman");
      $cat_one->set_name("Fido");
      $cat_one->set_age(10);
      $cat_two->set_breed("American Bobtail");
      echo $cat_one;
      echo $cat_two;
  ?>
\end{verbatim}

\subsection*{Problem 2:} 
\begin{verbatim}
  <?php
      class Employee {
          protected $first_name;
          protected $last_name;
          protected $salary;

          public function __construct($first_name, $last_name, $salary) {
              $this->first_name = $first_name;
              $this->last_name = $last_name;
              $this->salary = $salary;
          }

          public function __toString() {
              return $this->first_name . " " . $this->last_name;
          }
      }

      class SoftwareDeveloper extends Employee {
          public function __construct($first_name, $last_name, $salary, $prog_lang) {
              $this->first_name = $first_name;
              $this->last_name = $last_name;
              $this->salary = $salary;
              $this->prog_lang = $prog_lang;
          }
      } 

      class AgileCoach extends Employee {
          public function __construct($first_name, $last_name, $salary, $employees) {
              $this->first_name = $first_name;
              $this->last_name = $last_name;
              $this->salary = $salary;
              $this->employees = $employees;
          }

          public function add($employee) {
              array_push($this->employees, $employee);
          }

          public function remove($employee) {
              $key = array_search($employee, $this->employees);
              if ($key !== false) {
                  unset($this->employees[$key]);
              }
          }

          public function search($employee) {
              $key = array_search($employee, $this->employees);
              if ($key === false) {
                  echo $employee . " not found" . "<br>";
              } else {
                  echo $employee . " found" . "<br>";
              }
          }

          public function show_all() {
              foreach($this->employees as $employee) {
                  echo $employee . "<br>";
              }
          }
      } 

      $sft_dev_one = new SoftwareDeveloper("Alfredo", "Boyle", 50000, "C#");
      $sft_dev_two = new SoftwareDeveloper("Malik", "Martin", 55000, "JavaScript");
      $sft_dev_three = new SoftwareDeveloper("Livia", "Martin", 75000, 'Kotlin');
      $agile_coach = new AgileCoach("Lillian", "Cunningham", 100000, array($sft_dev_one, $sft_dev_two));
      $agile_coach->add($sft_dev_three);
      $agile_coach->remove($sft_dev_one);
      $agile_coach->show_all();
      $agile_coach->search($sft_dev_one);
      $agile_coach->search($sft_dev_three);
  ?>
\end{verbatim}

\subsection*{Problem 3:} 
\begin{verbatim}
    <?php
    class Language {
        public function good_morning() {
            throw new Exception("good_morning not implemented");
        }
    }    

    class Maori extends Language {
        public function good_morning() {
            echo "Morena";
        }
    }

    class Spanish extends Language {
        public function good_morning() {
            echo "Hola";
        }
    }

    class German extends Language {
        public function good_morning() {
            echo "Guten Morgen";
        }
    }
        
    $maori = new Maori();
    $spanish = new Spanish();
    $german = new German();
    $maori->good_morning();
    $spanish->good_morning();
    $german->good_morning();
?>
\end{verbatim}

\subsection*{Problem 4:} 

This is an okay solution, but we can limit the size of our \textbf{stack}. Also, there is an absence of error checking, in particular, when pushing \& popping to \& from the \textbf{stack}. 

\begin{verbatim}
  <?php
      class Stack {
          private $stack;
          private $limit;

          public function __construct() {
              $this->stack = array();
              $this->limit = 5; // Limit the number of elements to five
          }

          public function push($item) {
              // Okay solution
              array_unshift($this->stack, $item); 

              // Better solution
              if (count($this->stack) < $this->limit) {
                  array_unshift($this->stack, $item);
              } else {
                  throw new RunTimeException("The stack is full. You can not add an item.");
              }
          }

          public function pop() {
              // Okay solution
              return array_shift($this->stack);

              // Better solution
              if ($this->is_empty()) {
                  throw new RunTimeException("The stack is empty. You can not remove a item.");
              } else {
                  return array_shift($this->stack); 
              }
          }

          public function peek() {
              return current($this->stack);
          }

          public function is_empty() {
              return empty($this->stack);
          }

          public function size() {
              return count($this->stack);
          }

          public function show_all() {
              foreach($this->stack as $item) {
                  echo $item . "<br>";
              }
          }

          public function __toString() {
              return $this->stack;
          }
      }   
          
      $stack = new Stack();
      $stack->push("Introductory App Dev Concepts");
      $stack->push("Intermediate App Dev Concepts");
      $stack->push("Advanced App Dev Concepts");
      $stack->pop();
      $stack->show_all();
      echo $stack->peek() . " is at the top of the stack" . "<br>";
      echo "There are " . $stack->size() . " item(s) in the stack"
  ?>
\end{verbatim}

\subsection*{Problem 5:} 

\begin{verbatim}
  <?php
      interface App {
          public function login($email, $password);
          public function register($email, $password, $username); 
          public function logout(); 
      }

      class Facebook implements App {
          public function login($email, $password) {
              echo "Logged in with the email - $email.";
          }

          public function register($email, $password, $username) {
              echo "Registered with the email and username - $email and $username.";
          }

          public function logout() {
              echo "User logged out.";
          }
      }

      $fb_user = new Facebook();
      $fb_user->register("john.doe@gmail.com", "P@ssw0rd123", "john.doe") . "<br>";
      $fb_user->login("john.doe@gmail.com", "P@ssw0rd123") . "<br>";
      $fb_user->register() . "<br>";
      $fb_user->logout() . "<br>"; 
  ?>
\end{verbatim}

\end{document}
