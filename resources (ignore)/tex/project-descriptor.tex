% Author: Grayson Orr
% Course: ID607001: Introductory Application Development Concepts

\documentclass{article}
\author{}
 
\usepackage{fontspec}
\setmainfont{Arial}

\usepackage{graphicx}
\usepackage{wrapfig}
\usepackage{enumerate}
\usepackage{hyperref}
\usepackage[margin = 2.25cm]{geometry}
\usepackage[table]{xcolor}
\usepackage{soul}
\usepackage{fancyhdr}
\hypersetup{
  colorlinks = true,
  urlcolor = blue
} 
\setlength\parindent{0pt}
\pagestyle{fancy}
\fancyhf{}
\rhead{College of Engineering, Construction and Living Sciences\\Bachelor of Information Technology}
\lfoot{Project\\Version 1, Semester One, 2024}
\rfoot{\thepage}  
 
\begin{document}

\begin{figure}
	\centering
	\includegraphics[width=50mm]{../img/logo.png}
\end{figure}

\title{College of Engineering, Construction and Living Sciences\\Bachelor of Information Technology\\ID607001: Introductory Application Development Concepts\\Level 6, Credits 15\\\textbf{Project}}
\date{}
\maketitle 

\section*{Assessment Overview}
In this \textbf{individual} assessment, you will develop two \textbf{REST APIs} using \textbf{Express} and \textbf{Node.js}, and deploy them as a \textbf{web service} on \textbf{Render}. Your data will be stored in a \textbf{PostgreSQL} database on \textbf{Render}. In addition, marks will be allocated for code quality and best practices, documentation and \textbf{Git} usage. 

\section*{Learning Outcome}
At the successful completion of this course, learners will be able to:
\begin{enumerate}
	\item Design and build secure applications with dynamic database functionality following an appropriate software development methodology.
\end{enumerate}

\section*{Assessments}
\renewcommand{\arraystretch}{1.5}
\begin{tabular}{|c|c|c|c|}
	\hline
	\textbf{Assessment}                                 & \textbf{Weighting} & \textbf{Due Date}            & \textbf{Learning Outcome} \\ \hline
	\small Practical & \small 20\%        & \small 13-11-2024 (Wednesday at 4.59 PM)   & \small 1                   \\ \hline
	\small Project                 & \small 80\%        & \small 13-11-2024 (Wednesday at 4.59 PM) \small  & \small 1                   \\ \hline
\end{tabular}


\section*{Conditions of Assessment}
You will complete this assessment during your learner-managed time. However, there will be time during class to discuss the requirements and your progress on this assessment. This assessment will need to be completed by \textbf{Wednesday, 13 November 2024} at \textbf{4.59 PM}. 

\section*{Pass Criteria}
This assessment is criterion-referenced (CRA) with a cumulative pass mark of \textbf{50\%} across all assessments in \textbf{ID607001: Introductory Application Development Concepts}.

\section*{Submission}
You \textbf{must} submit all application files via \textbf{GitHub Classroom}. Here is the URL to the repository you will use for your submission – \href{https://classroom.github.com/a/wlzE5yYo}{https://classroom.github.com/a/wlzE5yYo}. If you do not have not one, create a \textbf{.gitignore} and add the ignored files in this resource - \href{https://raw.githubusercontent.com/github/gitignore/main/Node.gitignore}{https://raw.githubusercontent.com/github/gitignore/main/Node.gitignore}. The latest application files in the \textbf{main} branch will be used to mark against the \textbf{Functionality} criterion. Please test before you submit. Partial marks \textbf{will not} be given for incomplete functionality. Late submissions will incur a \textbf{10\% penalty per day}, rolling over at \textbf{5:00 PM}.

\section*{Authenticity}
All parts of your submitted assessment \textbf{must} be completely your work. Do your best to complete this assessment without using an \textbf{AI generative tool}. You need to demonstrate to the course lecturer that you can meet the learning outcome for this assessment. \\
 
 However, if you get stuck, you can use an \textbf{AI generative tool} to help you get unstuck, permitting you to acknowledge that you have used it. In the assessment's repository \textbf{README.md} file, please include what prompt(s) you provided to the \textbf{AI generative tool} and how you used the response(s) to help you with your work. It also applies to code snippets retrieved from \textbf{StackOverflow} and \textbf{GitHub}. \\
 
 Failure to do this may result in a mark of \textbf{zero} for this assessment.

\section*{Policy on Submissions, Extensions, Resubmissions and Resits}
The school's process concerning submissions, extensions, resubmissions and resits complies with \textbf{Otago Polytechnic | Te Pūkenga} policies. Learners can view policies on the \textbf{Otago Polytechnic | Te Pūkenga} website located at \href{https://www.op.ac.nz/about-us/governance-and-management/policies}{https://www.op.ac.nz/about-us/governance-and-management/policies}. 

\section*{Extensions}
Familiarise yourself with the assessment due date. Extensions will \textbf{only} be granted if you are unable to complete the assessment by the due date because of \textbf{unforeseen circumstances outside your control}. The length of the extension granted will depend on the circumstances and \textbf{must} be negotiated with the course lecturer before the assessment due date. A medical certificate or support letter may be needed. Extensions will not be granted for poor time management or pressure of other assessments.

\section*{Resits}
Resits and reassessments are not applicable in \textbf{ID607001: Introductory Application Development Concepts}. 

\newpage

\section*{Instructions}

\subsection*{Functionality - Learning Outcome 1 (50\%)}
\begin{itemize} 
	\item \textbf{Your choice REST API (20\%):}
	\begin{itemize}
		\item Developed using \textbf{Node.js}.
		\item Can run in development and production without modification.
		\item \textbf{Five} \textbf{models}. Each \textbf{model} contains a \textbf{minimum} of \textbf{three} \textbf{fields} excluding the \textbf{id}, \textbf{createdAt} and \textbf{updatedAt} \textbf{fields}.
		\item A range of different data types, i.e., all \textbf{fields} in a \textbf{model} can not be of a single type.
		\item \textbf{Five relationships} between \textbf{models}.
		\item \textbf{One} \textbf{model} has an \textbf{enum} \textbf{field}. 
		\item A \textbf{repository}, \textbf{controller} and \textbf{route} file for each \textbf{model}. Each \textbf{controller} file needs to contain operations for \textbf{POST}, \textbf{GET all}, \textbf{GET one}, \textbf{PUT} and \textbf{DELETE}.
		\item Return an appropriate success or failure message, and status code when performing the operations, i.e., \textbf{"Successfully created an institution"} or \textbf{"No institutions found"}, and \textbf{200} or \textbf{404}.
		\item \textbf{Filter} and \textbf{sort} your data using \textbf{query parameters}. All \textbf{fields} should be filterable and sortable (in ascending and descending order).
		\item \textbf{Paginate} your data using \textbf{query parameters}. The default number of data per page is 25.
		\item An endpoint for Swagger documentation. Each route needs to be documented. 
		\item Return an appropriate message if an endpoint does not exist.
		\item When creating and updating, validate each \textbf{field} using \textbf{Joi}. 
		\item Store your data in a \textbf{PostgreSQL} database on \textbf{Render}.
		\item Deploy your \textbf{REST API} as a \textbf{web service} on \textbf{Render}.
	\end{itemize}
	\item \textbf{OpenTDB REST API:}
		\begin{itemize}
			\item Developed using \textbf{Node.js}.
			\item Can run in development and production without modification.
		\end{itemize}

	\item \textbf{Scripts:}
	\begin{itemize}
		\item Run your \textbf{REST API} and \textbf{CRUD application} locally.
		\item Create and apply a migration using \textbf{Prisma}.
		\item Reset your database using \textbf{Prisma}.
		\item Open \textbf{Prisma Studio}. 
		\item Format your code using \textbf{Prettier}.
	\end{itemize}
\end{itemize}

\subsection*{Code Quality and Best Practices - Learning Outcome 1 (45\%)}
\begin{itemize}
	\item A \textbf{Node.js} \textbf{.gitignore} file is used.
	\item Environment variables' key is stored in the \textbf{.env.example} file. 
  	\item Appropriate naming of files, variables, functions and resource groups.
  	\begin{itemize}
	\item Resource groups are named with a plural noun instead of a noun or verb, i.e., \textbf{/api/items} not \textbf{/api/item}.
  \end{itemize}
	\item Idiomatic use of control flow, data structures and in-built functions.
	\item Efficient algorithmic approach.
	\item Sufficient modularity.
	\item Each \textbf{controller}, \textbf{route} and \textbf{component} file has a \textbf{JSDoc} header comment located at the top of the file.
	\item Code is formatted using \textbf{Prettier}.
	\item \textbf{Prettier} is installed as a \textbf{development dependency}.	
	\item No dead or unused code. 
\end{itemize}

\subsection*{Documentation and Git Usage - Learning Outcome 1 (5\%)}
\begin{itemize}
	\item A \textbf{GitHub} project board or issues to help you organise and prioritise your development work. The course lecturer needs to see consistent use of the \textbf{GitHub} project board or issues for the duration of the assessment.
    \item Provide the following in your repository \textbf{README.md} file:
    \begin{itemize} 
	\item A URL to your \textbf{REST API} as a \textbf{web service} on \textbf{Render}.
	\item A URL to your published \textbf{REST API} documentation. Each route needs to be documented. Include a description, example request and example response.
      \item How do you setup the environments, i.e., after the repository is cloned?
	  \item How do you run your \textbf{REST API} and \textbf{CRUD application} locally?
	  \item How do you create and apply a migration?  
	  \item How do you reset your database?
	  \item How do you open \textbf{Prisma Studio}?
	  \item How do you format your code?
    \end{itemize}
    \item Use of \textbf{Markdown}, i.e., headings, bold text, code blocks, etc.
    \item Correct spelling and grammar.
    \item Your \textbf{Git commit messages} should:
    \begin{itemize}
      \item Reflect the context of each functional requirement change.
      \item Be formatted using an appropriate naming convention style.
    \end{itemize}	
\end{itemize} 

\subsection*{Additional Information}
\begin{itemize}
    \item \textbf{Do not} rewrite your \textbf{Git} history. It is important that the course lecturer can see how you worked on your assessment over time. 
    \item You need to show the course lecturer the initial \textbf{GitHub} project board or issues before you start your development work. Following this, you need to show the course lecturer your \textbf{GitHub} project board or issues at the end of each week.
\end{itemize} 
\end{document}