% Author: Grayson Orr
% Course: ID607001: Introductory Application Development Concepts

\documentclass{article}
\author{}

\usepackage{graphicx}
\usepackage{wrapfig}
\usepackage{enumerate}
\usepackage{hyperref}
\usepackage[margin = 2.25cm]{geometry}
\usepackage[table]{xcolor}
\usepackage{soul}
\usepackage{fancyhdr}
\hypersetup{
  colorlinks = true,
  urlcolor = blue
}
\setlength\parindent{0pt}
\pagestyle{fancy}
\fancyhf{}
\rhead{College of Engineering, Construction \& Living Sciences\\Bachelor of Information Technology}
\lfoot{Project 1: Node.js REST API\\Version 3, Semester One, 2023}
\rfoot{\thepage}  
 
\begin{document}

\begin{figure}
	\centering
	\includegraphics[width=50mm]{../img/logo.png}
\end{figure}

\title{College of Engineering, Construction \& Living Sciences\\Bachelor of Information Technology\\ID607001: Introductory Application Development Concepts\\Level 6, Credits 15\\\textbf{Project 1: Node.js REST API}}
\date{}
\maketitle

\section*{Assessment Overview}
In this \textbf{individual} assessment, you will develop a \textbf{REST API} using \textbf{Node.js}. You will choose the theme of your \textbf{REST API}. It could be on sport, culture, food or something else you are interested in. Your data will be stored in a \textbf{SQLite} database. The main purpose of this assessment is to demonstrate your ability to develop a \textbf{REST API} using taught concepts such as queries, relationships, validation, etc. In addition, marks will be allocated for code elegance, documentation \& \textbf{Git} usage.

\section*{Learning Outcome}
At the successful completion of this course, learners will be able to:
\begin{enumerate}
	\item Design \& build secure applications with dynamic database functionality following an appropriate software development methodology.
\end{enumerate}

\section*{Assessments}
\renewcommand{\arraystretch}{1.5}
\begin{tabular}{|c|c|c|c|}
	\hline
	\textbf{Assessment}                                 & \textbf{Weighting} & \textbf{Due Date}            & \textbf{Learning Outcomes} \\ \hline
	\small Practical: Node.js REST API Testing & \small 20\%        & \small 05-05-2023 (Friday at 4.59 PM)   & \small 1                   \\ \hline
	\small Project 1: Node.js REST API                  & \small 30\%        & \small \small 05-05-2023 (Friday at 4.59 PM) & \small 1                   \\ \hline
	\small Project 2: React CRUD                        & \small 50\%        & \small 16-06-2023 (Friday at 4.59 PM)  & \small 1                   \\ \hline
\end{tabular}

\section*{Conditions of Assessment}
You will complete this assessment during your learner-managed time. However, there will be time to discuss the requirements \& your assessment progress during the teaching sessions. This assessment will need to be completed by \textbf{Friday, 05 May 2023 at 4.59 PM}.

\section*{Pass Criteria}
This assessment is criterion-referenced (CRA) with a cumulative pass mark of \textbf{50\%} across all assessments in \textbf{ID607001: Introductory Application Development Concepts}.

\section*{Submission}
You must submit all program files via \textbf{GitHub Classroom}. Here is the URL to the repository you will use for your submission – \href{https://classroom.github.com/a/4w4EqOUZ}{https://classroom.github.com/a/4w4EqOUZ}.  Create a \textbf{.gitignore} and add the ignored files in this resource - \href{https://raw.githubusercontent.com/github/gitignore/main/Node.gitignore}{https://raw.githubusercontent.com/github/gitignore/main/Node.gitignore}. The latest program files in the \textbf{master} or \textbf{main} branch will be used to mark against the \textbf{Functionality} criterion. Please test your \textbf{master} or \textbf{main} branch application before you submit. Partial marks \textbf{will not} be given for incomplete functionality. Late submissions will incur a \textbf{10\% penalty per day}, rolling over at \textbf{5:00 PM}.

\section*{Authenticity}
All parts of your submitted assessment \textbf{must} be completely your work. Do your best to complete this assessment without \textbf{ChatGPT}. You need to demonstrate to the course lecturer that you can meet the learning outcome for this assessment. \\
 
 However, if you get stuck, you can use \textbf{ChatGPT} to help you get unstuck, permitting you acknowledge that you have used \textbf{ChatGPT}. In the assessment's repository \textbf{README.md} file, please include what prompt(s) you provided to \textbf{ChatGPT} \& how you used the response(s) to help you with your work. It also applies to code snippets retrieved from \textbf{StackOverflow} \& \textbf{GitHub}. Failure to do this will result in a mark of \textbf{zero} for this assessment.

\section*{Policy on Submissions, Extensions, Resubmissions \& Resits}
The school's process concerning submissions, extensions, resubmissions \& resits complies with \textbf{Otago Polytechnic | Te Pūkenga} policies. Learners can view policies on the \textbf{Otago Polytechnic | Te Pūkenga} website located at \href{https://www.op.ac.nz/about-us/governance-and-management/policies}{https://www.op.ac.nz/about-us/governance-and-management/policies}. 

\section*{Extensions}
Familiarise yourself with the assessment due date. If you need an extension, contact the course lecturer before the due date. If you require more than a week's extension, a medical certificate or support letter from your manager may be needed.

\section*{Resubmissions}
Learners may be requested to resubmit an assessment following a rework of part/s of the original assessment. Resubmissions are to be completed within a negotiable short time frame \& usually \textbf{must} be completed within the timing of the course to which the assessment relates. Resubmissions will be available to learners who have made a genuine attempt at the first assessment opportunity \& achieved a \textbf{D grade (40-49\%)}. The maximum grade awarded for resubmission will be \textbf{C-}.

\section*{Resits}
Resits \& reassessments are not applicable in \textbf{ID607001: Introductory Application Development Concepts}. 

\newpage

\section*{Instructions}
You will need to submit a \textbf{REST API} \& documentation that meet the following requirements: \\

\subsection*{Functionality - Learning Outcome 1 (40\%)}
\begin{itemize} 
	\item \textbf{REST API:}
	\begin{itemize}
	\item Developed using \textbf{Node.js}.
	\item Can run locally without modification.
	\item \textbf{Six} \textbf{models} containing at least \textbf{three column values} which you can interact with.
	\item A range of different data types, i.e., all \textbf{column values} can not be of a single type.
	\item \textbf{Three relationships} between \textbf{models}.
	\item A \textbf{controller} \& \textbf{route} file for each \textbf{model}. Each \textbf{controller} file must contain operations for \textbf{CRUD} (Create, Read one, Read all, Update \& Delete).
	\item The \textbf{index route}, i.e., \textbf{localhost:3000/api} must display all of the available \textbf{routes} in the application.
	\item Using \textbf{Joi}, each \textbf{column value} has custom validation when creating \& updating a \textbf{document}.
	\item Version is set to \textbf{v1}. For example, an endpoint should look like \textbf{/api/v1/items}
	\item Return a success \& failure message when performing \textbf{CRUD} operations, i.e., \textbf{"Successfully created an institution"}.
	\item Filter \& sort using query parameters. A consumer should be able to filter all \textbf{column values} \& sort \textbf{column values} in ascending/descending order.
	\item Return an appropriate message if an endpoint does not exist.
	\item Paginate the data so that any number of records can be displayed per page. The default number is 10 records per page. 
	\item Rate limit is set to 50 requests per minute. You must display the following message if the user exceeds the 50 requests per minute - \textbf{"You have exceeded the number of requests per minute: 50. Please try again later."}
	\item Data is stored in a \textbf{SQLite} database.
\end{itemize}
\item NPM Scripts
\begin{itemize}
	\item Opening \textbf{Prisma Studio}.
	\item Creating a migration using \textbf{Prisma}. 
	\item Formatting your code using \textbf{Prettier}.
\end{itemize}
\end{itemize}

\subsection*{Code Elegance - Learning Outcome 1 (45\%)}
\begin{itemize}
	\item Environment variables' key is stored in the \textbf{env.example} file. 
  \item Database configured for the development environment.
  \begin{itemize}
    \item Create a new database called \textbf{dev.db}. 
  \end{itemize}
  \item Appropriate naming of variables, functions \& resource groups.
  \begin{itemize}
	\item Resource groups are named with a plural noun instead of a noun or verb, i.e., \textbf{/api/v1/items} not \textbf{/api/v1/item}.
  \end{itemize}
	\item Idiomatic use of control flow, data structures \& in-built functions.
  \item Efficient algorithmic approach.
  \item Sufficient modularity.
  \item Each \textbf{controller} \& \textbf{route} file \textbf{must} have a header comment located immediately before the \textbf{import} statements.
\item In-line comments where required.
  \item Code is formatted using \textbf{Prettier}.
  \item \textbf{Prettier} is installed as \textbf{development dependencies}.	
\item No dead or unused code. 
\end{itemize}

\subsection*{Documentation \& Git Usage - Learning Outcome 1 (15\%)}
\begin{itemize}
    \item Provide the following in your repository \textbf{README.md} file:
    \begin{itemize} 
      \item How do you setup the development environment, i.e., after the repository is cloned?
	  \item How do you open \textbf{Prisma Studio}?
	  \item How do you create a migration? 
	  \item How do you format your code?
    \end{itemize}
    \item Use of \textbf{Markdown}, i.e., headings, bold text, code blocks, etc.
    \item Correct spelling \& grammar.
    \item Your \textbf{Git commit messages} should:
    \begin{itemize}
      \item Reflect the context of each functional requirement change.
      \item Be formatted using an appropriate naming convention style.
    \end{itemize}	
\end{itemize}

\subsection*{Additional Information}
\begin{itemize}
    \item \textbf{Do not} rewrite your \textbf{Git} history. It is important that the course lecturer can see how you worked on your assessment over time. 
\end{itemize} 
\end{document}