% Author: Grayson Orr
% Course: ID607001: Introductory Application Development Concepts

\documentclass{article}
\author{}

\usepackage{graphicx}
\usepackage{wrapfig}
\usepackage{enumerate}
\usepackage{hyperref}
\usepackage[margin = 2.25cm]{geometry}
\usepackage[table]{xcolor}
\usepackage{soul}
\usepackage{fancyhdr}
\hypersetup{
  colorlinks = true,
  urlcolor = blue
} 
\setlength\parindent{0pt}
\pagestyle{fancy}
\fancyhf{}
\rhead{College of Engineering, Construction and Living Sciences\\Bachelor of Information Technology}
\lfoot{Project 1: Node.js REST API\\Version 4, Semester One, 2023}
\rfoot{\thepage}  
 
\begin{document}

\begin{figure}
	\centering
	\includegraphics[width=50mm]{../img/logo.png}
\end{figure}

\title{College of Engineering, Construction and Living Sciences\\Bachelor of Information Technology\\ID607001: Introductory Application Development Concepts\\Level 6, Credits 15\\\textbf{Project 1: Node.js REST API}}
\date{}
\maketitle

\section*{Assessment Overview}
In this \textbf{individual} assessment, you will develop a \textbf{REST API} using \textbf{Node.js} and deploy it as a \textbf{web service} on \textbf{Render}. You will choose the theme of your \textbf{REST API}. It could be on sport, culture, food or something else you are interested in. Your data will be stored in a \textbf{PostgreSQL} database on \textbf{Render}. The main purpose of this assessment is to demonstrate your ability to develop a \textbf{REST API} using taught concepts such as queries, relationships, validation, etc. In addition, marks will be allocated for code elegance, documentation and \textbf{Git} usage.

\section*{Learning Outcome}
At the successful completion of this course, learners will be able to:
\begin{enumerate}
	\item Design and build secure applications with dynamic database functionality following an appropriate software development methodology.
\end{enumerate}

\section*{Assessments}
\renewcommand{\arraystretch}{1.5}
\begin{tabular}{|c|c|c|c|}
	\hline
	\textbf{Assessment}                                 & \textbf{Weighting} & \textbf{Due Date}            & \textbf{Learning Outcomes} \\ \hline
	\small Practical: Node.js REST API Testing & \small 20\%        & \small 11-09-2023 (Monday at 04.59 PM)   & \small 1                   \\ \hline
	\small Project 1: Node.js REST API                  & \small 40\%        & \small \small 11-09-2023 (Monday at 04.59 PM) & \small 1                   \\ \hline
	\small Project 2: React CRUD                        & \small 40\%        & \small 13-11-2023 (Monday at 04.59 PM)  & \small 1                   \\ \hline
\end{tabular}

\section*{Conditions of Assessment}
You will complete this assessment during your learner-managed time. However, there will be time to discuss the requirements and your assessment progress during the teaching sessions. This assessment will need to be completed by \textbf{Monday, 11 September 2023 at 4.59 PM}.

\section*{Pass Criteria}
This assessment is criterion-referenced (CRA) with a cumulative pass mark of \textbf{50\%} across all assessments in \textbf{ID607001: Introductory Application Development Concepts}.

\section*{Submission}
You must submit all program files via \textbf{GitHub Classroom}. Here is the URL to the repository you will use for your submission – \href{https://classroom.github.com/a/wJ4pC7Y7}{https://classroom.github.com/a/wJ4pC7Y7}. Create a \textbf{.gitignore} and add the ignored files in this resource - \href{https://raw.githubusercontent.com/github/gitignore/main/Node.gitignore}{https://raw.githubusercontent.com/github/gitignore/main/Node.gitignore}. The latest program files in the \textbf{master} or \textbf{main} branch will be used to mark against the \textbf{Functionality} criterion. Please test your \textbf{master} or \textbf{main} branch application before you submit. Partial marks \textbf{will not} be given for incomplete functionality. Late submissions will incur a \textbf{10\% penalty per day}, rolling over at \textbf{5:00 PM}.

\section*{Authenticity}
All parts of your submitted assessment \textbf{must} be completely your work. Do your best to complete this assessment without using a \textbf{AI generative tool}. You need to demonstrate to the course lecturer that you can meet the learning outcome for this assessment. \\
 
 However, if you get stuck, you can use a \textbf{AI generative tool} to help you get unstuck, permitting you acknowledge that you have used \textbf{AI generative tool}. In the assessment's repository \textbf{README.md} file, please include what prompt(s) you provided to the \textbf{AI generative tool} and how you used the response(s) to help you with your work. It also applies to code snippets retrieved from \textbf{StackOverflow} and \textbf{GitHub}. \\
 
 Failure to do this may result in a mark of \textbf{zero} for this assessment.

\section*{Policy on Submissions, Extensions, Resubmissions and Resits}
The school's process concerning submissions, extensions, resubmissions and resits complies with \textbf{Otago Polytechnic | Te Pūkenga} policies. Learners can view policies on the \textbf{Otago Polytechnic | Te Pūkenga} website located at \href{https://www.op.ac.nz/about-us/governance-and-management/policies}{https://www.op.ac.nz/about-us/governance-and-management/policies}. 

\section*{Extensions}
Familiarise yourself with the assessment due date. Extensions will \textbf{only} be granted if you are unable to complete the assessment by the due date because of \textbf{unforeseen circumstances outside your control}. The length of the extension granted will depend on the circumstances and must be negotiated with the course lecturer before the assessment due date. A medical certificate or support letter may be needed.

\section*{Resubmissions}
Learners may be requested to resubmit an assessment following a rework of part/s of the original assessment. Resubmissions are to be completed within a negotiable short time frame and usually \textbf{must} be completed within the timing of the course to which the assessment relates. Resubmissions will be available to learners who have made a genuine attempt at the first assessment opportunity and achieved a \textbf{D grade (40-49\%)}. The maximum grade awarded for resubmission will be \textbf{C-}.

\section*{Resits}
Resits and reassessments are not applicable in \textbf{ID607001: Introductory Application Development Concepts}. 

\newpage

\section*{Instructions}
You will need to submit a \textbf{REST API} and documentation that meet the following requirements: \\

\subsection*{Functionality - Learning Outcome 1 (50\%)}
\begin{itemize} 
	\item \textbf{REST API:}
	\begin{itemize}
	\item Developed using \textbf{Node.js}.
	\item Can run locally without modification.
	\item A \textbf{maximum} of \textbf{eight} \textbf{models}. Each containing a \textbf{minimum} of \textbf{four} \textbf{fields} excluding the \textbf{id}, \textbf{createdAt} and \textbf{updatedAt} \textbf{fields}.
	\item A range of different data types, i.e., all \textbf{fields} in a \textbf{model} can not be of a single type.
	\item A minimum of \textbf{five relationships} between \textbf{models}.
	\item At least \textbf{one} \textbf{model} must have an \textbf{enum} \textbf{field}. 
	\item A \textbf{controller} and \textbf{route} file for each \textbf{model}. Each \textbf{controller} file must contain operations for \textbf{POST}, \textbf{GET all}, \textbf{GET one}, \textbf{PUT} and \textbf{DELETE}.
	\item Return an appropriate success or failure message, and status code when performing the operations, i.e., \textbf{"Successfully created an institution"} or \textbf{"No institutions found"}.
	\item The \textbf{index route}, i.e., \textbf{https://localhost:3000/api/} must display all existing \textbf{routes}.
	\item When creating and updating, validate each \textbf{field} using \textbf{Joi}. 
	\item Store your data in a \textbf{PostgreSQL} database on \textbf{Render}.
	\item Deploy your \textbf{REST API} as a \textbf{web service} on \textbf{Render}.
	\item \textbf{Independent Research Tasks:} 
	\end{itemize}
	\begin{itemize}
		\item \textbf{Filter} and \textbf{sort} your data using \textbf{query parameters}. All \textbf{fields} should be filterable and sortable (in ascending and descending order).
		\item \textbf{Paginate} your data using \textbf{query parameters}. The default number of data per page is 25.
		\item Return an appropriate message if an endpoint does not exist.
		\item Limit the number of \textbf{API requests} per minute to 100. You must display the following message if the user exceeds the 100 \textbf{API requests} per minute - \textbf{"You have exceeded the number of requests per minute: 100. Please try again later."}
	\end{itemize}
	\item \textbf{Scripts:}
	\begin{itemize}
		\item Run your \textbf{REST API} locally.
		\item Create and apply a migration using \textbf{Prisma}.
		\item Reset your database using \textbf{Prisma}.
		\item Open \textbf{Prisma Studio}. 
		\item Format your code using \textbf{Prettier}.
	\end{itemize}
\end{itemize}

\subsection*{Code Elegance - Learning Outcome 1 (40\%)}
\begin{itemize}
	\item A \textbf{Node.js} \textbf{.gitignore} file is used.
	\item Environment variables' key is stored in the \textbf{env.example} file. 
  \item Appropriate naming of files, variables, functions and resource groups.
  \begin{itemize}
	\item Resource groups are named with a plural noun instead of a noun or verb, i.e., \textbf{/api/v1/items} not \textbf{/api/v1/item}.
  \end{itemize}
	\item Idiomatic use of control flow, data structures and in-built functions.
  \item Efficient algorithmic approach.
  \item Sufficient modularity.
  \item Each \textbf{controller} and \textbf{route} file \textbf{must} have a \textbf{JSDoc} header comment located at the top of the file.
\item In-line comments where required. It should be for code that needs further explanation, i.e., the \textbf{independent research tasks}.
  \item Code is formatted using \textbf{Prettier}.
  \item \textbf{Prettier} is installed as a \textbf{development dependency}.	
\item No dead or unused code. 
\end{itemize}

\subsection*{Documentation and Git Usage - Learning Outcome 1 (10\%)}
\begin{itemize}
    \item Provide the following in your repository \textbf{README.md} file:
    \begin{itemize} 
      \item How do you setup the environment, i.e., after the repository is cloned?
	  \item How do you run your \textbf{REST API} locally?
	  \item How do you create and apply a migration?  
	  \item How do you reset your database?
	  \item How do you open \textbf{Prisma Studio}?
	  \item How do you format your code?
	  \item An Entity Relationship Diagram (ERD) of your database.
	  \item A URL to your \textbf{REST API} as a \textbf{web service} on \textbf{Render}.
	  \item A URL to your published \textbf{REST API} documentation. 
    \end{itemize}
    \item Use of \textbf{Markdown}, i.e., headings, bold text, code blocks, etc.
    \item Correct spelling and grammar.
    \item Your \textbf{Git commit messages} should:
    \begin{itemize}
      \item Reflect the context of each functional requirement change.
      \item Be formatted using an appropriate naming convention style.
    \end{itemize}	
\end{itemize}

\subsection*{Additional Information}
\begin{itemize}
    \item \textbf{Do not} rewrite your \textbf{Git} history. It is important that the course lecturer can see how you worked on your assessment over time. 
\end{itemize} 
\end{document}