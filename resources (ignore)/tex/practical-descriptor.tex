% Author: Grayson Orr
% Course: ID607001: Introductory Application Development Concepts

\documentclass{article}
\author{}

\usepackage{graphicx}
\usepackage{wrapfig}
\usepackage{enumerate}
\usepackage{hyperref}
\usepackage[margin = 2.25cm]{geometry}
\usepackage[table]{xcolor}
\usepackage{fancyhdr}
\hypersetup{
  colorlinks = true,
  urlcolor = blue
}
\setlength\parindent{0pt}
\pagestyle{fancy}
\fancyhf{}
\rhead{College of Engineering, Construction and Living Sciences\\Bachelor of Information Technology}
\lfoot{Practical \\Version 1, Semester One, 2024}
\rfoot{\thepage}
 
\begin{document}

\begin{figure}
    \centering
    \includegraphics[width=50mm]{../img/logo.png}
\end{figure}

\title{College of Engineering, Construction and Living Sciences\\Bachelor of Information Technology\\ID607001: Introductory Application Development Concepts\\Level 6, Credits 15\\\textbf{Practical}}
\date{}
\maketitle

\section*{Assessment Overview}
In this \textbf{individual} assessment, you will test the \textbf{REST API} you created in the \textbf{Project} assessment. In addition, marks will be allocated for code quality and best practices, documentation and \textbf{Git} usage.  

\section*{Learning Outcome}
At the successful completion of this course, learners will be able to:
\begin{enumerate}
	\item Design and build secure applications with dynamic database functionality following an appropriate software development methodology.
\end{enumerate}

\section*{Assessments}
\renewcommand{\arraystretch}{1.5}
\begin{tabular}{|c|c|c|c|}
	\hline
	\textbf{Assessment}                                 & \textbf{Weighting} & \textbf{Due Date}            & \textbf{Learning Outcome} \\ \hline
	\small Practical & \small 20\%        & \small 21-06-2024 (Friday at 4.59 PM)   & \small 1                   \\ \hline
	\small Project                 & \small 80\%        & \small 21-06-2024 (Friday at 4.59 PM) \small  & \small 1                   \\ \hline
\end{tabular}

\section*{Conditions of Assessment}
You will complete this assessment during your learner-managed time. However, there will be time during class to discuss the requirements and your progress on this assessment. This assessment will need to be completed by \textbf{Friday, 21 June 2024} at \textbf{4.59 PM}. 

\section*{Pass Criteria}
This assessment is criterion-referenced (CRA) with a cumulative pass mark of \textbf{50\%} across all assessments in \textbf{ID607001: Introductory Application Development Concepts}.

\section*{Submission}
You \textbf{must} submit all application files via \textbf{GitHub Classroom}. Here is the URL to the repository you will use for your submission – \href{https://classroom.github.com/a/wlzE5yYo}{https://classroom.github.com/a/wlzE5yYo}. If you do not have not one, create a \textbf{.gitignore} and add the ignored files in this resource - \href{https://raw.githubusercontent.com/github/gitignore/main/Node.gitignore}{https://raw.githubusercontent.com/github/gitignore/main/Node.gitignore}. Create a branch called \textbf{practical}. The latest application files in the \textbf{practical} branch will be used to mark against the \textbf{Functionality} criterion. Please test before you submit. Partial marks \textbf{will not} be given for incomplete functionality. Late submissions will incur a \textbf{10\% penalty per day}, rolling over at \textbf{5:00 PM}.

\section*{Authenticity}
All parts of your submitted assessment \textbf{must} be completely your work. Do your best to complete this assessment without using an \textbf{AI generative tool}. You need to demonstrate to the course lecturer that you can meet the learning outcome for this assessment. \\
 
 However, if you get stuck, you can use an \textbf{AI generative tool} to help you get unstuck, permitting you to acknowledge that you have used it. In the assessment's repository \textbf{README.md} file, please include what prompt(s) you provided to the \textbf{AI generative tool} and how you used the response(s) to help you with your work. It also applies to code snippets retrieved from \textbf{StackOverflow} and \textbf{GitHub}. \\
 
 Failure to do this may result in a mark of \textbf{zero} for this assessment.

\section*{Policy on Submissions, Extensions, Resubmissions and Resits}
The school's process concerning submissions, extensions, resubmissions and resits complies with \textbf{Otago Polytechnic | Te Pūkenga} policies. Learners can view policies on the \textbf{Otago Polytechnic | Te Pūkenga} website located at \href{https://www.op.ac.nz/about-us/governance-and-management/policies}{https://www.op.ac.nz/about-us/governance-and-management/policies}. 

\section*{Extensions}
Familiarise yourself with the assessment due date. Extensions will \textbf{only} be granted if you are unable to complete the assessment by the due date because of \textbf{unforeseen circumstances outside your control}. The length of the extension granted will depend on the circumstances and \textbf{must} be negotiated with the course lecturer before the assessment due date. A medical certificate or support letter may be needed. Extensions will not be granted for poor time management or pressure of other assessments.

\section*{Resits}
Resits and reassessments are not applicable in \textbf{ID607001: Introductory Application Development Concepts}. 

\newpage

\section*{Instructions}
You will need to submit a \textbf{suite of API tests} and \textbf{documentation} that meet the following requirements:

\subsection*{Functionality - Learning Outcome 1 (50\%)}
\begin{itemize}
  \item \textbf{Testing:}
	\begin{itemize}
    \item \textbf{API tests} are written using \textbf{Mocha} and \textbf{Chai}.
    \item \textbf{API tests} verifying the correctness for the following:
      \begin{itemize}
        \item \textbf{GET one}, \textbf{GET all}, \textbf{POST}, \textbf{PUT} and \textbf{DELETE} operations. (30 tests).
        \item \textbf{Index route} displaying all existing \textbf{routes}. (one test).
        \item A \textbf{route} that does not exist. (one test).
        \item Validation for \textbf{POST} and \textbf{PUT} operations. (12 tests).
      \end{itemize}
	\end{itemize}

	\item \textbf{Scripts:}
	\begin{itemize}
    \item Seed your database with \textbf{Prisma}.
		\item Run your \textbf{API tests} using \textbf{Mocha}.
	\end{itemize}
\end{itemize}

\subsection*{Code Quality and Best Practices - Learning Outcome 1 (45\%)}
\begin{itemize}
  \item Appropriate naming of files, variables and functions.
	\item Idiomatic use of control flow, data structures and in-built functions.
  \item Efficient algorithmic approach.
  \item Sufficient modularity.
	\item Each \textbf{test} file has a \textbf{JSDoc} header comment located at the top of the file.
	\item Code is formatted using \textbf{Prettier}.
	\item \textbf{Mocha} and \textbf{Chai} are installed as \textbf{development dependencies}.	
  \item No dead or unused code.
\end{itemize} 

\subsection*{Documentation and Git Usage - Learning Outcome 1 (5\%)}
\begin{itemize}
    \item Provide the following in your repository \textbf{README.md} file:
    \begin{itemize} 
      \item How to seed your database with \textbf{Prisma}?
      \item How do you run your \textbf{API tests}?
    \end{itemize}
    \item Use of \textbf{Markdown}, i.e., headings, bold text, code blocks, etc.
    \item Correct spelling and grammar.
    \item Your \textbf{Git commit messages} should:
    \begin{itemize}
      \item Reflect the context of each functional requirement change.
      \item Be formatted using an appropriate naming convention style.
    \end{itemize}	
\end{itemize}
          
\subsection*{Additional Information}
\begin{itemize}
  \item You do not need to test the \textbf{filtering}, \textbf{sorting} and \textbf{pagination} operations.
    \item \textbf{Do not} rewrite your \textbf{Git} history. It is important that the course lecturer can see how you worked on your assessment over time. 
\end{itemize} 

\end{document}
