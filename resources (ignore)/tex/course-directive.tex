% Author: Grayson Orr
% Course: ID607001: Introduction Application Development Concepts

\documentclass{article}
\author{}

\usepackage{graphicx}
\usepackage{wrapfig}
\usepackage{enumerate}
\usepackage{hyperref}
\usepackage[margin = 2.25cm]{geometry}
\usepackage[table]{xcolor}
\hypersetup{
  colorlinks = true,
  urlcolor = blue
} 
\setlength\parindent{0pt}

\begin{document}

\begin{figure}
	\centering
	\includegraphics[width=50mm]{../img/logo.png}
\end{figure}

\title{Course Directive\\ID607001: Introductory Application Development Concepts\\Semester One, 2024}
\date{}
\maketitle 

\section*{Course Information}
\begin{tabular}{ll}
	Level:        & 6 \\
	Credits:      & 15                                                             \\
	Prerequisite: & ID511001: Programming 2                                                   \\
	Timetable:    & Rōpū Kōwhai: Wednesday 10.00 AM D207 and Friday 1.00 PM D201  \\
\end{tabular}

\section*{Teaching Staff}
\begin{tabular}{ll}
	Name:            & Grayson Orr                           \\
	Position:        & Senior Lecturer and Second/Third-Year Coordinator\\
	Office Location: & D309                                 \\
	Email Address    & grayson.orr@op.ac.nz                    \\
\end{tabular}

\section*{Course Dates}
\begin{tabular}{ll}
	Term 1:           & Monday 26 February - Friday 12 April  \\
	Mid Semester Break: &  Monday 15 April  - Friday 26 April     \\
	Term 2:             & Monday 29 April - Thursday 27 June       \\
\end{tabular}

\section*{Public Holidays and Anniversary Days}
A list of public holidays and anniversary days can be found here - \href{https://www.op.ac.nz/students/importantdates}{https://www.op.ac.nz/students/importantdates}

\section*{Aims}
To introduce the concepts of application development including algorithms, data structures and design patterns that are required to use a simple, industry-relevant development framework.

\section*{Learning Outcome}
At the successful completion of this course, learners will be able to:
\begin{enumerate}
	\item Design and build secure applications with dynamic database functionality following an appropriate software development methodology.
\end{enumerate}

\section*{Assessments}
\renewcommand{\arraystretch}{1.5}
\begin{tabular}{|c|c|c|c|}
	\hline
	\textbf{Assessment}                                 & \textbf{Weighting} & \textbf{Due Date}            & \textbf{Learning Outcome} \\ \hline
	\small Practical & \small 20\%        & \small 21-06-2024 (Friday at 4.59 PM)   & \small 1                   \\ \hline
	\small Project                 & \small 80\%        & \small 21-06-2024 (Friday at 4.59 PM) \small  & \small 1                   \\ \hline
\end{tabular}

\section*{Provisional Schedule}
\renewcommand{\arraystretch}{1.5}
\begin{tabular}{|c|c|c|c|}
	\hline
	\textbf{Week}                  & \textbf{Date Starting}            & \multicolumn{2}{c|}{\textbf{Topics}}                                                                                             \\ \hline
	\footnotesize 1/Tahi           & \footnotesize 26-02-2024 & \multicolumn{2}{c|}{\footnotesize GitHub and JavaScript}    \\ \hline
	\footnotesize 2/Rua            & \footnotesize 04-03-2024 & \multicolumn{2}{c|}{\footnotesize Express, Postman and Deployment}                   \\ \hline
	\footnotesize 3/Toru           & \footnotesize 11-03-2024 & \multicolumn{2}{c|}{\footnotesize PostgreSQL, ORM and Relationships} \\ \hline
	\footnotesize 4/Whā            & \footnotesize 18-03-2024 & \multicolumn{2}{c|}{\footnotesize Validation, Pagination, Filtering and Sorting}                               \\ \hline
	\footnotesize 5/Rima           & \footnotesize 25-03-2024 & \multicolumn{2}{c|}{\footnotesize Seeding and API Testing}                                                \\ \hline
	\footnotesize 6/Ono            & \footnotesize 01-04-2024  & \multicolumn{2}{c|}{\footnotesize JSDoc and Postman Documentation}                                                   \\ \hline
	\footnotesize 7/Whitu          & \footnotesize 08-04-2024 &  \multicolumn{2}{c|}{\footnotesize Assessment Work}                            \\ \hline
	\rowcolor{yellow} \multicolumn{4}{|c|}{\footnotesize Mid Term Break}                                                                                                                         \\ \hline
	\footnotesize 8/Waru   & \footnotesize 29-04-2024 & \multicolumn{2}{c|}{\footnotesize Assessment Work}                                                   \\ \hline
	\footnotesize 9/Iwa            & \footnotesize 06-05-2024 & \multicolumn{2}{c|}{\footnotesize React 1}                                                                 \\ \hline
	\footnotesize 10/Tekau         & \footnotesize 13-05-2024 & \multicolumn{2}{c|}{\footnotesize React 2}                                                                 \\ \hline
	\footnotesize 11/Tekau mā tahi & \footnotesize 20-05-2024 & \multicolumn{2}{c|}{\footnotesize React 3}                                                                 \\ \hline
	\footnotesize 12/Tekau mā rua  & \footnotesize 27-05-2024 & \multicolumn{2}{c|}{\footnotesize React 4}                                                                 \\ \hline
	\footnotesize 13/Tekau mā toru & \footnotesize 03-06-2024 & \multicolumn{2}{c|}{\footnotesize Assessment Work}                                                     \\ \hline
	\footnotesize 14/Tekau mā whā  & \footnotesize 10-06-2024 & \multicolumn{2}{c|}{\footnotesize Assessment Work} \\ \hline 
	\footnotesize 15/Tekau mā rima & \footnotesize 17-06-2024 & \multicolumn{2}{c|}{\footnotesize Assessment Work}                                                       \\ \hline
	\footnotesize 16/Tekau mā ono  & \footnotesize 24-06-2024 & \multicolumn{2}{c|}{\footnotesize Assessment Marking}                                                         \\ \hline
\end{tabular}

\section*{Resources}

\subsection*{Software}
This paper will be taught using \textbf{Microsoft Visual Studio Code} and \textbf{Node.js}. An installer for \textbf{Microsoft Visual Studio Code} and \textbf{Node.js} are available - \href{https://code.visualstudio.com/download}{https://code.visualstudio.com/download} and \href{https://nodejs.org/en/download}{https://nodejs.org/en/download}. Please refer any problems with downloads or installers to \textbf{Rob Broadley} in D205a.

\subsection*{Readings}
No textbook is required for this course. URLs to useful resources will be provided in the lecture notes. 

\section*{Course Requirements and Expectations}

\subsection*{Learning Hours}
This course requires \textbf{150 hours} of learning. This time includes \textbf{64 hours} of timetabled class time, and \textbf{86 hours} of self-directed reading, preparation and completion of assessments.  

\subsection*{Criteria for Passing}
To pass this paper, you must achieve a cumulative pass mark of \textbf{50\%} over all assessments. There are no reassessments or resits.

\subsection*{Attendance}
\begin{itemize}
	\item Learners are expected to attend all classes, including lectures and labs.
	\item If you cannot attend for a few days for any reason, contact the course.
\end{itemize}

\subsection*{Communication}
\textbf{Microsoft Outlook/Teams} are the official communication channels for this course. It is your responsibility to regularly check \textbf{Microsoft Outlook/Teams} and \href{https://github.com/otago-polytechnic-bit-courses/ID607001-intro-app-dev-concepts}{GitHub} for important course material, including changes to class scheduling or assessment details. Not checking will not be accepted as an excuse.

\subsection*{Snow Days/Polytechnic Closure}
In the event \textbf{Otago Polytechnic | Te Pūkenga} is closed or has a delayed opening because of snow or bad weather, you should not attempt to attend class if it is unsafe to do so. It is possible that the course lecturer will not be able to attend either, so classes will not physically be meeting. However, this does not become a holiday. Rather, the course material will be made available on \href{https://github.com/otago-polytechnic-bit-courses/ID607001-intro-app-dev-concepts}{GitHub} for classes affected by the closure. You are responsible for any course material presented in this manner. Information about closure will be posted on the \textbf{Otago Polytechnic | Te Pūkenga Facebook} page \href{https://www.facebook.com/OtagoPoly}{https://www.facebook.com/OtagoPoly}.

\subsection*{Group Work and Originality}
Learners in the \textbf{Bachelor of Information Technology} programme are expected to hand in original work. Learners are encouraged to discuss assessments with their fellow learners, however, all assessments are to be completed as individual works unless group work is explicitly required (i.e. if it doesn't say it is group work then it is not group work - even if a group consultation was involved). Failure to submit your original work will be treated as plagiarism.

\subsection*{ChatGPT}
Learning to use \textbf{Artificial Intelligence tools} like \textbf{ChatGPT} is an important skill. While \textbf{ChatGPT} is a powerful tool, you \textbf{must} be aware of the following:

\begin{itemize}
    \item If you provide \textbf{ChatGPT} with a prompt that is not refined enough, it may generate a not-so-useful response
    \item Do not trust \textbf{ChatGPT's} responses blindly. You \textbf{must} still use your judgement and may need to do additional research to determine if the response is correct
    \item Acknowledge that you are using \textbf{ChatGPT}. In the assessment's repository \textbf{README.md} file, please include what prompt(s) you provided to \textbf{ChatGPT} and how you used the response(s) to help you with your work
\end{itemize}

\subsection*{Referencing}
Appropriate referencing is required for all work. Referencing standards will be specified by the course lecturer.

\subsection*{Plagiarism}
Plagiarism is submitting someone elses work as your own. Plagiarism offences are taken seriously and an assessment that has been plagiarised may be awarded a zero mark. A definition of plagiarism is in the Student Handbook, available online or at the school office.

\subsection*{Submission Requirements}
All assessments are to be submitted by the time, date, and method given when the assessment is issued. Failure to meet all requirements will result in a penalty of up to \textbf{10\%} per day (including weekends).

\subsection*{Extensions}
Familiarise yourself with the assessment due dates. Extensions will \textbf{only} be granted if you are unable to complete the assessment by the due date because of \textbf{unforeseen circumstances outside your control}. The length of the extension granted will depend on the circumstances and \textbf{must} be negotiated with the course lecturer before the assessment due date. A medical certificate or support letter may be needed. Extensions will not be granted for poor time management or pressure of other assessments.

\subsection*{Impairment}
In case of sickness contact the course lecturer or \textbf{Head of Information Technology (Michael Holtz)} as soon as possible, preferably before the assessment is due. The policy regarding the granting of a mark that considers impaired performance requires a medical certificate and a medical practitioner’s signature on a form. You may refer to the guide on impaired performance on the student handbook.

\subsection*{Appeals}
If you are concerned about any aspect of your assessment, approach the course lecturer in the first instance. We support an open-door policy and aim to resolve issues promptly. Further support is available from the \textbf{Head of Information Technology (Michael Holtz)} and \textbf{Second/Third-Year Coordinator (Grayson Orr)}. \textbf{Otago Polytechnic | Te Pūkenga} has a formal process for academic appeals if necessary.

\subsection*{Other Documents}
Regulatory documents relating to this course can be found on the \textbf{Otago Polytechnic | Te Pūkenga} website.

\end{document}