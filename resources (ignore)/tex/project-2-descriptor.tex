% Author: Grayson Orr
% Course: ID607001: Introductory Application Development Concepts

\documentclass{article}
\author{}

\usepackage{graphicx}
\usepackage{wrapfig}
\usepackage{enumerate}
\usepackage{hyperref}
\usepackage[margin = 2.25cm]{geometry}
\usepackage[table]{xcolor}
\usepackage{soul}
\usepackage{fancyhdr}
\hypersetup{
  colorlinks = true,
  urlcolor = blue
} 
\setlength\parindent{0pt}
\pagestyle{fancy}
\fancyhf{}
\rhead{College of Engineering, Construction and Living Sciences\\Bachelor of Information Technology}
\lfoot{Project 2: React CRUD\\Version 5, Semester Two, 2023}
\rfoot{\thepage}
 
\begin{document} 

\begin{figure}
	\centering
	\includegraphics[width=50mm]{../img/logo.png}
\end{figure}

\title{College of Engineering, Construction and Living Sciences\\Bachelor of Information Technology\\ID607001: Introductory Application Development Concepts\\Level 6, Credits 15\\\textbf{Project 2: React CRUD}}
\date{}
\maketitle

\section*{Assessment Overview}
In this \textbf{individual} assessment, you will develop a \textbf{CRUD application} using \textbf{React}. This application will consume the \textbf{REST API} you developed in the \textbf{Project 1: Node.js REST API} assessment. The main purpose of this assessment is not just to build a full-stack application, rather to demonstrate an ability to decouple the presentation layer (\textbf{frontend}) from the business logic (\textbf{backend}). In addition, marks will be allocated for code elegance, documentation and \textbf{Git} usage. 

\section*{Learning Outcome}
At the successful completion of this course, learners will be able to:
\begin{enumerate}
	\item Design and build secure applications with dynamic database functionality following an appropriate software development methodology.
\end{enumerate}

\section*{Assessments}
\renewcommand{\arraystretch}{1.5}
\begin{tabular}{|c|c|c|c|}
	\hline
	\textbf{Assessment}                                 & \textbf{Weighting} & \textbf{Due Date}            & \textbf{Learning Outcome} \\ \hline
	\small Practical: Node.js REST API Testing & \small 20\%        & \small 11-09-2023 (Monday at 04.59 PM)   & \small 1                   \\ \hline
	\small Project 1: Node.js REST API                  & \small 40\%        & \small \small 11-09-2023 (Monday at 04.59 PM) & \small 1                   \\ \hline
	\small Project 2: React CRUD                        & \small 40\%        & \small 13-11-2023 (Monday at 04.59 PM)  & \small 1                   \\ \hline
\end{tabular}

\section*{Conditions of Assessment}
You will complete majority of this assessment during your learner-managed time. However, there will be time to discuss the requirements and your assessment progress during the teaching sessions. This assessment will need to be completed by \textbf{Monday, 13 November 2023 at 4.59 PM}.

\section*{Pass Criteria}
This assessment is criterion-referenced (CRA) with a cumulative pass mark of \textbf{50\%} across all assessments in \textbf{ID607001: Introductory Application Development Concepts}.

\section*{Submission}
You must submit all program files via \textbf{GitHub Classroom}. Here is the URL to the repository you will use for your submission – \href{https://classroom.github.com/a/wJ4pC7Y7}{https://classroom.github.com/a/wJ4pC7Y7}. Create a \textbf{.gitignore} and add the ignored files in this resource - \href{https://raw.githubusercontent.com/github/gitignore/main/Node.gitignore}{https://raw.githubusercontent.com/github/gitignore/main/Node.gitignore}. The latest program files in the \textbf{master} or \textbf{main} branch will be used to mark against the \textbf{Functionality} criterion. Please test your \textbf{master} or \textbf{main} branch application before you submit. Partial marks \textbf{will not} be given for incomplete functionality. Late submissions will incur a \textbf{10\% penalty per day}, rolling over at \textbf{5:00 PM}.

\section*{Authenticity}
All parts of your submitted assessment \textbf{must} be completely your work. Do your best to complete this assessment without using an \textbf{AI generative tool}. You need to demonstrate to the course lecturer that you can meet the learning outcome for this assessment. \\
 
 However, if you get stuck, you can use an \textbf{AI generative tool} to help you get unstuck, permitting you acknowledge that you have used it. In the assessment's repository \textbf{README.md} file, please include what prompt(s) you provided to the \textbf{AI generative tool} and how you used the response(s) to help you with your work. It also applies to code snippets retrieved from \textbf{StackOverflow} and \textbf{GitHub}. \\
 
 Failure to do this may result in a mark of \textbf{zero} for this assessment.

\section*{Policy on Submissions, Extensions, Resubmissions and Resits}
The school's process concerning submissions, extensions, resubmissions and resits complies with \textbf{Otago Polytechnic | Te Pūkenga} policies. Learners can view policies on the \textbf{Otago Polytechnic | Te Pūkenga} website located at \href{https://www.op.ac.nz/about-us/governance-and-management/policies}{https://www.op.ac.nz/about-us/governance-and-management/policies}. 

\section*{Extensions}
Familiarise yourself with the assessment due date. Extensions will \textbf{only} be granted if you are unable to complete the assessment by the due date because of \textbf{unforeseen circumstances outside your control}. The length of the extension granted will depend on the circumstances and must be negotiated with the course lecturer before the assessment due date. A medical certificate or support letter may be needed. Extensions will not be granted for poor time management or pressure of other assessments.

\section*{Resubmissions}
Learners may be requested to resubmit an assessment following a rework of part/s of the original assessment. Resubmissions are to be completed within a negotiable short time frame and usually \textbf{must} be completed within the timing of the course to which the assessment relates. Resubmissions will be available to learners who have made a genuine attempt at the first assessment opportunity and achieved a \textbf{D grade (40-49\%)}. The maximum grade awarded for resubmission will be \textbf{C-}.

\section*{Resits}
Resits and reassessments are not applicable in \textbf{ID607001: Introductory Application Development Concepts}. 
\newpage

\section*{Instructions}
You will need to submit a \textbf{CRUD application} and \textbf{documentation} that meet the following requirements: 

\subsection*{Functionality - Learning Outcome 1 (50\%)}
\begin{itemize}
	\item \textbf{CRUD Application:}
		\begin{itemize}
		\item Request \textbf{REST API} data from at four three \textbf{API} resource groups using \textbf{Axios}.
		\item Create new \textbf{REST API} data via a button and form. 
		\item View \textbf{REST API} data in a table.
		\item Update \textbf{REST API} data via a button and form. 
		\item Delete \textbf{REST API} data via a button.
		Prompt the user for deletion. You \textbf{can} use the in-built \textbf{confirm() JavaScript} function. 
		\item Incorrectly formatted form field values handled gracefully using validation error messages.
		\item UI is visually attractive with a coherent graphical theme and style.	
	\end{itemize}
	\item \textbf{Scripts:}
	\begin{itemize}
		\item Formatting your code using \textbf{Prettier}.
	\end{itemize}
\end{itemize}

\subsection*{Code Elegance - Learning Outcome 1 (40\%)}
\begin{itemize}
	\item A \textbf{Node.js} \textbf{.gitignore} file is used.
  \item Appropriate naming of files, variables, functions and components.
	\item Idiomatic use of control flow, data structures and in-built functions.
  \item Efficient algorithmic approach.
  \item Sufficient modularity.
  \item Each \textbf{component} file \textbf{must} have a \textbf{JSDoc} header comment located immediately before the \textbf{import} statements.
\item In-line comments where required. It should be for code that needs further explanation.
  \item Code is formatted using \textbf{Prettier}.
  \item \textbf{Prettier} is installed as a \textbf{development dependency}.	
\item No dead or unused code. 
\end{itemize}

\subsection*{Documentation and Git Usage - Learning Outcome 1 (10\%)}
\begin{itemize}
	\item \textbf{GitHub} project board to help you organise and prioritise your work. 
    \item Provide the following in your repository \textbf{README.md} file:
    \begin{itemize} 
      \item How do you setup the environment, i.e., after the repository is cloned?
      \item How do you format your code?
    \end{itemize}
    \item Use of \textbf{Markdown}, i.e., headings, bold text, code blocks, etc.
    \item Correct spelling and grammar.
    \item Your \textbf{Git commit messages} should:
    \begin{itemize}
      \item Reflect the context of each functional requirement change.
      \item Be formatted using an appropriate naming convention style.
    \end{itemize}	
\end{itemize}
          
\subsection*{Additional Information}
\begin{itemize}
    \item \textbf{Do not} rewrite your \textbf{Git} history. It is important that the course lecturer can see how you worked on your assessment over time. 
\end{itemize} 
\end{document}