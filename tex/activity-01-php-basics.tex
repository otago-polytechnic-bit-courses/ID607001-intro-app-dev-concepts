% Author: Grayson Orr
% Course: IN607: Introductory Application Development Concept

\documentclass{article}
\author{}

\usepackage{graphicx}
\usepackage{wrapfig}
\usepackage{enumerate}
\usepackage{hyperref}
\usepackage[margin = 2.25cm]{geometry}
\usepackage[table]{xcolor}
\usepackage{fancyhdr}
\hypersetup{
  colorlinks = true, 
  urlcolor = blue
}
\setlength\parindent{0pt}
\pagestyle{fancy}
\fancyhf{}
\rhead{College of Engineering, Construction \& Living Sciences\\Bachelor of Information Technology}
\lfoot{Activity 01: PHP Basics\\Version 1, Semester One, 2021}
\rfoot{\thepage}

\begin{document}

\begin{figure}
    \centering
    \includegraphics[width=50mm]{./img/logo.png}
\end{figure}

\title{College of Engineering, Construction \& Living Sciences\\Bachelor of Information Technology\\IN607: Introductory Application Development Concepts\\Level 6, Credits 15\\\textbf{Activity 01: PHP Basics}}
\date{}
\maketitle

\section*{Code Review}
You must submit all program files via \textbf{GitHub Classroom}. Here is the URL to the repository you will use for your code review – \href{https://classroom.github.com/a/fqBug5Kt}{https://classroom.github.com/a/fqBug5Kt}. Checkout from the \textbf{main} branch to the \textbf{01-activity} branch by running the command - \textbf{git checkout 01-activity}. This branch will be your development branch for this activity. Once you have completed this activity, create a pull request \& assign the \textbf{GitHub} user \textbf{grayson-orr} to a reviewer. \textbf{Do not} merge your own pull request.

\section*{Part One}

\subsection*{Problem 1:} 
Declare two variables called \textbf{name} \& \textbf{age} with the values John \& 55. Use the two variables to display the expected output.

\begin{verbatim}
  <?php
  // Write your solution here

  // Expected output:
  // Hello my name is John & I am 55 years old.
  ?>
\end{verbatim}

\subsection*{Problem 2:} Calculate the \textbf{sum} of the given \textbf{integers} \& display the expected output.

\begin{verbatim}
  <?php
  $x = 1957452;
  $y = 2975635;

  // Write your solution here

  // Expected output:
  // The sum of 1957452 & 2975635 is 4933087
  ?>
\end{verbatim}

\subsection*{Problem 3:} 
Calculate the \textbf{average} of the given \textbf{array} of \textbf{doubles} \& display the expected output.

\begin{verbatim}
  <?php
  $numbers = array(45.3, 67.5, -45.6, 20.34, -33.0, 45.6)

  // Write your solution here

  // Expected output:
  // Average: 16.69 
  ?>
}
\end{verbatim}

\subsection*{Problem 4:}
Write a function called \textbf{fizzBuzz} which accepts an \textbf{integer} called \textbf{num}. If \textbf{num} is a multiple of three, return \textbf{Fizz}, if \textbf{num} is a multiple of five, return \textbf{Buzz} \& if \textbf{num} is a multiple of three \& five, return \textbf{FizzBuzz}. Call the \textbf{fizzBuzz} function in the for loop to display the expected output.

\begin{verbatim}
  <?php
  // Write your fizzBuzz function here
  
  for ($i = 1; $i <= 15; $i+=2) {
    // Write your solution here
  }

  // Expected output:
  // 1
  // Fizz
  // Buzz
  // 7
  // Fizz
  // 11
  // 13
  // FizzBuzz
  ?>
\end{verbatim}

\subsection*{Problem 5:}
You have been given an \textbf{array} of \textbf{floats} or \textbf{doubles}. Display \textbf{only} the odd numbers in the \textbf{array}. Sort from lowest to highest.

\begin{verbatim}
  <?php  
  $numbers = array(21, 19, 68, 55, 42, 12) 
  
  // Write your solution here

  // Expected output:
  // 19
  // 21
  // 55
  ?>
\end{verbatim}

\section*{Part Two}

\subsection*{Problem 6:}
Write a function called \textbf{is\_anagram} which accepts two parameters called \textbf{string\_one} \& \textbf{string\_two}. In the function block, write some code that checks whether or not \textbf{string\_one} \& \textbf{string\_two} are an anagram. An anagram is a word or phrase that made by arranging the letters of another word or phrase in a different order. If you are still unsure what an anagram is, here is an example:

\begin{verbatim}
  Input: is_anagram("elvis", "lives")
  Output: true

  Input: is_anagram("cat", "sat")
  Output : false
\end{verbatim}

Call the \textbf{is\_anagram} function to display the expected output.

\begin{verbatim}
  <?php  
  // Write your solution here

  // Expected output:
  // true
  // false
  ?>
\end{verbatim}

\subsection*{Problem 7:}
Write a function called \textbf{convert} which accepts two parameters \textbf{hours} \& \textbf{minutes} (both integers). In the function block, write some code that converts both \textbf{hours} \& \textbf{minutes} to seconds, then adds them together.

\begin{verbatim}
  <?php  
  // Write your solution here

  convert(1, 3)

  // Expected output:
  // 3780
  ?>
\end{verbatim}

\subsection*{Problem 8:}
Write a function called \textbf{factorial} which accepts a single parameter called \textbf{num}. In the function block, write some code that returns the factorial of \textbf{num}. Assume all inputs are greater than or equal to 0. 

\begin{verbatim}
  <?php  
  // Write your solution here

  factorial(3)
  factorial(5)

  // Expected output:
  // 6
  // 120
  ?>
\end{verbatim}

\subsection*{Problem 9:}
Write a function called \textbf{palindrome} which has a parameter called \textbf{string}. In the function block, determine whether or not \textbf{string} is a palindrome. The function should return a \textbf{boolean}.

\begin{verbatim}
  <?php  
  // Write your solution here

  palindrome("A man, a plan, a canal - Panama")
  palindrome("Hello, World!")

  // Expected output:
  // true
  // false
  ?>
\end{verbatim}
 
\subsection*{Problem 10:}
Write a function called \textbf{is\_five\_letters} which accepts an \textbf{array} of \textbf{strings}. In the function block, return all words that are exactly \textbf{five} letters.

\begin{verbatim}
  <?php  
  // Write your solution here

  is_five_letters(["car", "bike", "truck"])

  // Expected output:
  // ["truck"]
  ?>
\end{verbatim}

\subsection*{Problem 11:}
Write a function called \textbf{remove\_one} which removes all occurrences of the number \textbf{one} in an \textbf{array}. 

\begin{verbatim}
  <?php  
  // Write your solution here

  remove_one([1, 1, 1, 1, 1])
  remove_one([1, 2, 3, 4, 1])

  // Expected output:
  // []
  // [2, 3, 4]
  ?>
\end{verbatim}

\subsection*{Problem 12:}
It is my birthday in a couple months, so you will need to save your pennies to buy me a present. Write a function called \textbf{is\_my\_birthday} that accepts a \textbf{DateTime} object \& returns \textbf{true} if it is the 8th of April, otherwise return \textbf{false}.

\begin{verbatim}
  <?php  
  // Write your solution here

  is_my_birthday(new DateTime("1995-04-08"))
  is_my_birthday(new DateTime("2015-12-13"))

  // Expected output:
  // true
  // false
  ?>
\end{verbatim}

\subsection*{Problem 13:}

Write a function that accepts an \textbf{integer} called \textbf{num}. If \textbf{num} is prime, return \textbf{true}, otherwise return \textbf{false}. 

\begin{verbatim}
  <?php  
  // Write your solution here

  is_prime(11)
  is_prime(18)

  // Expected output:
  // true
  // false
  ?>
\end{verbatim}

\subsection*{Problem 14:}
Write a function that splits a \textbf{string} into separate alpha \& numeric values. Return the values in an \textbf{array}.

\begin{verbatim}
  <?php  
  // Write your solution here

  splitCode("IN607")

  // Expected output:
  // ["IN", 607]
  ?>
\end{verbatim} 

\end{document}