% Author: Grayson Orr
% Course: IN607: Introductory Application Development Concept

\documentclass{article}
\author{}

\usepackage{graphicx}
\usepackage{wrapfig}
\usepackage{enumerate}
\usepackage{hyperref}
\usepackage[margin = 2.25cm]{geometry}
\usepackage[table]{xcolor}
\usepackage{fancyhdr}
\hypersetup{
  colorlinks = true,
  urlcolor = blue
}
\setlength\parindent{0pt}
\pagestyle{fancy}
\fancyhf{}
\rhead{College of Engineering, Construction \& Living Sciences\\Bachelor of Information Technology}
\lfoot{Practical 01: PHP Basics\\Version 1, Semester One, 2021}
\rfoot{\thepage}

\begin{document}

\begin{figure}
    \centering
    \includegraphics[width=50mm]{./img/logo.png}
\end{figure}

\title{College of Engineering, Construction \& Living Sciences\\Bachelor of Information Technology\\IN607: Introductory Application Development Concepts\\Level 6, Credits 15\\\textbf{Practical 01: PHP Basics}}
\date{}
\maketitle

\section*{Assessment Overview}
In this assessment, you will solve 15 coding problems using \textbf{PHP} in \textbf{Repl.it}. 

\section*{Learning Outcomes}
At the successful completion of this course, learners will be able to: 
\begin{enumerate}
	\item Design \& build usable, secure \& attractive applications with dynamic database functionality following an appropriate software development methodology.
\end{enumerate} 

\section*{Assessment Table}
\renewcommand{\arraystretch}{1.5}
\begin{tabular}{|l|l|l|l|l|}
	\hline      
	\vtop{\hbox{\strut \textbf{Assessment}}\hbox{\strut \textbf{Activity}}} & \textbf{Weighting} & \vtop{\hbox{\strut \textbf{Learning}}\hbox{\strut \textbf{Outcomes}}} & \vtop{\hbox{\strut \textbf{Assessment}}\hbox{\strut \textbf{Grading Scheme}}} & \vtop{\hbox{\strut \textbf{Completion}}\hbox{\strut \textbf{Requirements}}} \\
	                            
	\hline
	                                
	\small Practical                                          & \small 20\%        & \small 1                                                         & \small CRA                                                                    & \small Cumulative                                                           \\ \hline  
	\small Project                                                             & \small 80\%        & \small 1                                                       & \small CRA                                                                    & \small Cumulative                                                           \\ \hline 
\end{tabular}

\section*{Conditions of Assessment}
You will complete this assessment during your learner managed time, however, there will be availability during the teaching sessions to discuss the requirements \& your progress of this assessment.

\section*{Pass Criteria}
This assessment is criterion-referenced (CRA) with a cumulative pass mark of \textbf{50\%} over all assessments in \textbf{IN607: Introductory Application Development Concepts}.

\section*{Authenticity}
All parts of your submitted assessment must be completely your work \& any references must be cited appropriately including, externally-sourced graphic elements. Provide your references in a \textbf{README.md} file. All media must be royalty free (or legally purchased) for educational use. Failure to do this will result in a mark of \textbf{zero} for this assessment.

\section*{Policy on Submissions, Extensions, Resubmissions \& Resits}
The school's process concerning submissions, extensions, resubmissions \& resits complies with \textbf{Otago Polytechnic} policies. Learners can view policies on the \textbf{Otago Polytechnic} website located at \href{https://www.op.ac.nz/about-us/governance-and-management/policies}{https://www.op.ac.nz/about-us/governance-and-management/policies}.

\section*{Submissions}
You must submit all program files via \textbf{GitHub Classroom}. Here is the URL to the repository you will use for your submission – \href{https://classroom.github.com/a/fqBug5Kt}. 

\section*{Instructions - Learning Outcomes 1}

\subsection*{Problem 1:} 
Declare two variables called \textbf{name} \& \textbf{age} with the values John \& 55. Use the two variables to display the expected output.

\begin{verbatim}
  <?php
  // Write your solution here

  // Expected output:
  // Hello my name is John & I am 55 years old.
  ?>
\end{verbatim}

\subsection*{Problem 2:} Calculate the \textbf{sum} of the given integers \& display the expected output.

\begin{verbatim}
  <?php
  $x = 1957452;
  $y = 2975635;

  // Write your solution here

  // Expected output:
  // The sum of 1957452 & 2975635 is 4933087
  ?>
\end{verbatim}

\subsection*{Problem 3:} 
Calculate the \textbf{average} of the given \textbf{double array} \& display the expected output.

\begin{verbatim}
  <?php
  $numbers = array(45.3, 67.5, -45.6, 20.34, -33.0, 45.6)

  // Write your solution here

  // Expected output:
  // Average: 16.69 
  ?>
}
\end{verbatim}

\subsection*{Problem 4:}
Write a function called \textbf{fizzBuzz} which has a parameter called \textbf{num}. If \textbf{num} is a multiple of three, return \textbf{Fizz}, if \textbf{num} is a multiple of five, return \textbf{Buzz} \& if \textbf{num} is a multiple of three \& five, return \textbf{FizzBuzz}. Call the \textbf{fizzBuzz} function in the for loop to display the expected output.

\begin{verbatim}
  <?php
  // Write your fizzBuzz function here
  
  for ($i = 1; $i <= 15; $i+=2) {
    // Write your solution here
  }

  // Expected output:
  // 1
  // Fizz
  // Buzz
  // 7
  // Fizz
  // 11
  // 13
  // FizzBuzz
  ?>
\end{verbatim}

\subsection*{Problem 5:}
You have been given an array of floats or doubles. Display \textbf{only} the odd numbers in the array. Sort from lowest to highest.

\begin{verbatim}
  <?php  
  $numbers = array(21, 19, 68, 55, 42, 12) 
  
  // Write your solution here

  // Expected output:
  // 19
  // 21,
  // 55
  ?>
\end{verbatim}

\subsection*{Problem 6:}
Write a function called \textbf{is\_anagram} which has two parameters called \textbf{string\_one} \& \textbf{string\_two}. In the function block, write some code that checks whether or not \textbf{string\_one} \& \textbf{string\_two} are an anagram. An anagram is a word or phrase that made by arranging the letters of another word or phrase in a different order. If you are still unsure what an anagram is, here is an example:

\begin{verbatim}
  Input: is\_anagram("elvis", "lives")
  Output: true

  Input: is\_anagram("cat", "sat")
  Output : false
\end{verbatim}

Call the \textbf{is\_anagram} function to display the expected output.

\begin{verbatim}
  <?php  
  // Write your solution here

  // Expected output:
  // true
  // false
  ?>
\end{verbatim}

\subsection*{Problem 7:}
Write a function called \textbf{convert} which has two parameters called \textbf{hours} \& \textbf{minutes}. In the function block, write some code that converts both \textbf{hours} \& \textbf{minutes} to seconds, then adds them together.

\begin{verbatim}
  <?php  
  // Write your solution here

  convert(1, 3)

  // Expected output:
  // 3780
  ?>
\end{verbatim}

\subsection*{Problem 8:}
Write a function called \textbf{factorial} which has a parameter called \textbf{num}. In the function block, write some code that returns the factorial of \textbf{num}. Assume all inputs are ≥ 0. 

\begin{verbatim}
  <?php  
  // Write your solution here

  factorial(3)
  factorial(5)

  // Expected output:
  // 6
  // 120
  ?>
\end{verbatim}

\subsection*{Problem 9:}
Write a function called \textbf{palindrome} which has a parameter called \textbf{string}. In the function block, determine whether or not \textbf{string} is a palindrome. The function should return a boolean, i.e., true or false.

\textbf{Additional constraints:}
\begin{itemize}
	\item Check for a case insensitive input.
  \item Special characters \& spaces should be ignored.
\end{itemize} 

\begin{verbatim}
  <?php  
  // Write your solution here

  palindrome("A man, a plan, a canal – Panama")
  palindrome("Hello, World!")

  // Expected output:
  // true
  // false
  ?>
\end{verbatim}

\subsection*{Problem 10:}
Write a function called \textbf{is\_five\_letters} which has a parameter called \textbf{array}. In the function block, return the words that are exactly \textbf{five} letters.

\begin{verbatim}
  <?php  
  // Write your solution here

  is_five_letters(["car", "bike", "truck"])

  // Expected output:
  // ["truck"]
  ?>
\end{verbatim}

\subsection*{Problem 11:}
Write a function called \textbf{remove\_one} which removes all occurrences of the number \textbf{1 (one)} in an array. 
\begin{verbatim}
  <?php  
  // Write your solution here

  remove\_one([1, 1, 1, 1, 1])
  remove\_one([1, 2, 3, 4, 1])

  // Expected output:
  // []
  // [2, 3, 4]
  ?>
\end{verbatim}

\subsection*{Problem 12:}

https://edabit.com/challenge/7dF5QoA3Tg8xn4A2u

\subsection*{Problem 13:}

https://edabit.com/challenge/Xqi73dZ8kLDegRcwQ

\subsection*{Problem 14:}

https://edabit.com/challenge/Q5mYvvjj8HmzhQeCN

\subsection*{Problem 15:}

https://edabit.com/challenge/cX4ibRJ2amc992NNM


\end{document}